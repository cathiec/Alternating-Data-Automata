\section{Emptiness Checking for First Order Alternating Automata}
\label{app:algorithm}

The execution of Algorithm \ref{alg:impact} consists of three phases,
corresponding to the \textsc{Close}, \textsc{Refine} and
\textsc{Expand} of the original \impact\ procedure \cite{McMillan06}.

\begin{algorithm}[t!]
\begin{algorithmic}[0] \State {\bf input}: a first order
  alternating automaton $\A = \tuple{\Sigma,X,Q,\iota,F,\Delta}$
  
  \State {\bf output}: $\top$ if $L(\A)=\emptyset$, or word $w \in
  L(\A)$, otherwise

  \State {\bf data structures}: $\worklist$ and unfolding tree
  $\mathcal{U} = \tuple{N,E,\rootNode,U,\lhd}$,
  where: \begin{compactitem}
  \item $N$ is a set of nodes,
  \item $E \subseteq N \times \Sigma \times N$ is a set of edges labeled by input events,
  \item $U : N \rightarrow \posforms(Q,\emptyset)$ is a labeling of
    nodes with positive sentences
  \item $\lhd \subseteq N \times N$ is a coverage relation,
  \end{compactitem}  

  \State {\bf initially} $\worklist = \set{\rootNode}$ and
  $N=E=U=\lhd=\emptyset$.
\end{algorithmic}
\begin{algorithmic}[1]   
  \While{$\worklist \neq \emptyset$}
  \label{ln:impact-while}

  \State dequeue $n$ from $\worklist$ 
  \label{ln:impact-dequeue}

  \State $N \leftarrow N \cup \set{n}$

  \State let $\alpha(n)$ be $a_1, \ldots, a_k$

  \If{$\substform{\alpha}(\stamp{X}{1}, \ldots, \stamp{X}{k})$ is satisfiable} 
  \Comment{counterexample is feasible}

  \State get model $\nu$ of $\substform{\alpha}(\stamp{X}{1}, \ldots, \stamp{X}{k})$

  \State {\bf return} $w = (a_1,\nu(\stamp{X}{1})) \ldots (a_k,\nu(\stamp{X}{k}))$
  \label{ln:impact-nonempty}
  \Comment{$w \in L(\A)$ by construction}

  \Else \Comment{spurious counterexample}

  \State let $(I_0,\ldots,I_k)$ be a GLI for $\alpha$
  \label{ln:refine-begin}

  \State $b \leftarrow \bot$

  \For{$i=0,\ldots,k$}

  \If{$U(n_i) \not\models I_i$}

  \State $\mathit{Uncover} \leftarrow \set{ m \in N \mid (m,n_i) \in \lhd }$
  
  \State $\lhd \leftarrow \lhd \setminus \set{(m,n_i) \mid m \in \mathit{Uncover}}$
  \Comment{uncover the nodes covered by $n_i$}

  \For{$m \in \mathit{Uncover}$ such that $m$ is a leaf of $\mathcal{U}$}

  \State enqueue $m$ into $\worklist$
  \Comment{reactivate uncovered leaves}
  
  \EndFor
  
  \State $U(n_i) \leftarrow U(n_i) \wedge J_i$
  \Comment{strenghten the label of $n_i$ (Lemma \ref{lemma:refinement})}

  \If{$\neg b$}

  \State $b \leftarrow \Call{Close}{n_i}$
  \label{ln:refine-end}

  \EndIf % $\neg b$

  \EndIf % $\Lambda(n_i) \not\models I_i$

  \EndFor 

  \EndIf % $\mu(n)$ is satisfiable

  \If{$n$ is not covered}

  \For{$a \in \Sigma$} 
  \Comment{expand $n$}
  \label{ln:expand-begin}

  \State let $s$ be a fresh node and $e = (n,a,s)$ be a new edge

  \State $E \leftarrow E \cup \set{e}$
  \label{ln:edge-insert}

  \State $U \leftarrow U \cup \set{(s,\top)}$

  \State enqueue $s$ into $\worklist$
  \label{ln:expand-end}

  \EndFor

  \EndIf % $n$ is not covered

  \EndWhile  

  \State {\bf return} $\top$
\end{algorithmic}
\caption{\impact-based Semi-algorithm for First Order Alternating Automata}
\label{alg:impact}
\end{algorithm}

\begin{algorithm}[htb]
\begin{algorithmic}[1]
\Function{Close}{$x$} {\bf returns} $\Bool$

\For{$y \in N$ such that $\alpha(y) \prec^* \alpha(x)$}

\If{$U(x) \models U(y)$}

\State $\lhd \leftarrow \left[\lhd \setminus \set{ (p,q) \in \lhd \mid
  \text{$q$ is $x$ or a successor of $x$}}\right] \cup \set{(x,y)}$
\label{ln:close-uncover}

\State {\bf return} $\top$ 

\EndIf

\EndFor

\State {\bf return} $\bot$

\EndFunction
\end{algorithmic}
\caption{The \textsc{Close} function from Algorithm \ref{alg:impact}}
\end{algorithm}

Let $n$ be a node removed from the worklist at line
\ref{ln:impact-dequeue} and let $\alpha(n)$ be the input sequence
labeling the path from the root node to $n$. If
$\substform{\alpha(n)}$ is satisfiable, the sequence $\alpha(n)$ is
feasible, in which case a model of $\substform{\alpha(n)}$ is obtained
and a word $w \in L(\A)$ is returned. Otherwise, $\alpha(n)$ is an
infeasible input sequence and the procedure enters the refinement
phase (lines \ref{ln:refine-begin}-\ref{ln:refine-end}). The GLI for
$\alpha(n)$ is used to strenghten the labels of all the ancestors of
$n$, by conjoining the formulae of the interpolant, changed according
to Lemma \ref{lemma:refinement}, to the existing labels.

In this process, the nodes on the path between $\rootNode$ and $n$,
including $n$, might become eligible for coverage, therefore we
attempt to close each ancestor of $n$ that is impacted by the
refinement (line \ref{ln:refine-end}). Observe that, in this case the
call to $\Call{Close}{}$ must uncover each node which is covered by a
successor of $n$ (line \ref{ln:close-uncover} of the $\Call{Close}{}$
function). This is required because, due to the over-approximation of
the sets of reachable configurations, the covering relation is not
transitive, as explained in \cite{McMillan06}. If $\Call{Close}{}$
adds a covering edge $(n_i,m)$ to $\lhd$, it does not have to be
called for the successors of $n_i$ on this path, which is handled via
the boolean flag $b$. Finally, if $n$ is still uncovered (it has not
been previously covered during the refinement phase) we expand $n$
(lines \ref{ln:expand-begin}-\ref{ln:expand-end}) by creating a new
node for each successor $s$ via the input event $a \in \Sigma$ and
inserting it into the worklist.

\section{Proofs of the Technical Results in the Paper}
\label{app:proofs}

\subsection{Proof of Lemma \ref{lemma:path-formula}}

``$\subseteq$'' Let $\I$ be a minimal interpretation such that
$\I,\data{w} \models \pathform{\event{w}}$. We show that there exists
an execution $\T$ of $\A$ over $w$ such that $\I = \I_\T$, by
induction on $n\geq 0$. For $n=0$, we have $w=\epsilon$ and
$\pathform{\event{w}} = \stamp{\iota}{0}$. Because $\iota$ is a
sentence, the valuation $\data{w}$ is not important in $\I,\data{w}
\models \stamp{\iota}{0}$ and, moreover, since $\I$ is minimal, we
have $\I \in \minsem{\stamp{\iota}{0}}$. We define the interpretation
$\J(q)=\I(\stamp{q}{0})$, for all $q \in Q$. Then $\cube{\J}$ is an
execution of $\A$ over $\epsilon$ and $\I = \I_{\cube{\J}}$ is
immediate. For the inductive case $n>0$, we assume that
$w=u\cdot(a_n,\nu_n)$ for a word $u$. Let $\J$ be the interpretation
defined as $\I$ for all $\stamp{q}{i}$, with $q \in Q$ and $i \in
[n-1]$, and $\emptyset$ everywhere else. Then $\J,u_\Data \models
\pathform{u_\Sigma}$ and $\J$ is moreover minimal. By the induction
hypothesis, there exists an execution $\G$ of $\A$ over $u$, such that
$\J = \I_\G$. Consider a leaf of a tree $T \in \G$, labeled with a
configuration $q(d_1,\ldots,d_{\#(q)})$ and let $\forall y_1 \ldots
\forall y_{\#(q)} ~.~ \stamp{q}{n-1}(\vec{y}) \rightarrow
\stamp{\psi}{n}$ be the subformula of $\pathform{\event{w}}$
corresponding to the application(s) of the transition rule $q(\vec{y})
\arrow{a_n}{} \psi$ at the $(n-1)$-th step. Let $\nu = \data{w}[y_1
  \leftarrow d_1,\ldots,y_{\#(q)} \leftarrow d_{\#(q)}]$. Because $\I,
\data{w} \models \forall y_1 \ldots \forall y_{\#(q)} ~.~
\stamp{q}{n-1}(\vec{y}) \rightarrow \stamp{\psi}{n}$, we have $\I \in
\sem{\stamp{\psi}{n}}_{\nu}$ and let $\K$ be one of the minimal
interpretations such that $\K \subseteq \I$ and $\K \in
\sem{\stamp{\psi}{n}}_{\nu}$. It is not hard to see that $\K$ exists
and is unique, otherwise we could take the pointwise intersection of
two or more such interpretations. We define the interpretation
$\overline{\K}(q) = \overline{\K}(\stamp{q}{n})$ for all $q \in Q$. We
have that $\overline{\K} \in \minsem{\psi}_{\nu}$ --- if
$\overline{\K}$ was not minimal, $\K$ was not minimal to start with,
contradiction. Then we extend the execution $\G$ by appending to each
node labeled with a configuration $q(d_1, \ldots, d_{\#(q)})$ the cube
$\cube{\overline{\K}}$. By repeating this step for all leaves of a
tree in $\G$, we obtain an execution of $\A$ over $w$.

``$\supseteq$'' Let $\T$ be an execution of $\A$ over $w$. We show
that $\I_{\T}$ is a minimal interpretation such that $\I_{\T},
\data{w} \models \pathform{\event{w}}$, by induction on $n \geq
0$. For $n=0$, $\T$ is a cube from $\cube{\minsem{\iota}}$, by
definition. Then $\I_{\T} \models \stamp{\iota}{0}$ and moreover, it
is a minimal such interpretation. For the inductive case $n > 0$, let
$w=u\cdot(a_n,\nu_n)$ for a word $u$. Let $\G$ be the restriction of
$\T$ to $u$. Consequently, $\I_\G$ is the restriction of $\I_\T$ to
$\stamp{Q}{\leq n-1}$. By the inductive hypothesis, $\I_{\G}$ is a
minimal interpretation such that \(\I_{\G}, u_{\Data} \models
\pathform{u_\Sigma}\). Since $\I_{\T}(\stamp{q}{n}) =
\set{\tuple{d_1,\ldots,d_{\#(q)}} \mid q(d_1,\ldots,d_{\#(q)}) \text{
    labels a node on the $n$-th level in $\T$}}$, we have $\I_{\T},
\data{w} \models \varphi$, for each subformula $\varphi = \forall y_1
\ldots \forall y_{\#(q)} ~.~ \stamp{q}{n-1}(\vec{y}) \rightarrow
\stamp{\psi}{n}$ of $\pathform{\event{w}}$, by the execution semantics
of $\A$. This is the case because the children of each node labeled
with $q(d_1,\ldots,d_{\#(q)})$ on the $(n-1)$-th level of $\T$ form a
cube from $\cube{\minsem{\psi}_\nu}$, where $\nu$ is a valuation that
assigns each $y_i$ the value $d_i$ and behaves like $\data{w}$,
otherwise. Now supppose, for a contradiction, that $\I_{\T}$ is not
minimal and let $\J \subsetneq \I_{\T}$ be an interpretation such that
$\J,\data{w} \models \pathform{\event{w}}$. First, we show that the
restriction $\J'$ of $\J$ to $\bigcup_{i=0}^{n-1} \stamp{Q}{i}$ must
coincide with $\I_{\G}$. Assuming this is not the case, i.e.\ $\J'
\subsetneq \I_{\G}$, contradicts the minimality of $\I_{\G}$. Then the
only possibility is that $\J(\stamp{q}{n}) \subsetneq
\I_{\T}(\stamp{q}{n})$, for some $q \in Q$. Let
$p_1(y_1,\ldots,y_{\#(p_1)}) \arrow{a_n}{} \psi_1, \ldots,
p_k(y_1,\ldots,y_{\#(p_k)}) \arrow{a_n}{} \psi_k$ be the set of
transition rules in which the predicate symbol $q$ occurs on the
right-hand side. Then it must be the case that, for some node on the
$(n-1)$-th level of $\G$, labeled with a configuration
$p_i(d_1,\ldots,d_{\#(p_i)})$, the set of children does not form a
minimal cube from $\cube{\minsem{\stamp{\psi_i}{n}}}$, which
contradicts the execution semantics of $\A$. \qed

\subsection{Proof of Lemma \ref{lemma:acceptance}}

``$(\ref{it1:lemma:acceptance}) \Rightarrow
(\ref{it2:lemma:acceptance})$'' Let $\I$ be an interpretation such
that $\I,\data{w} \models \accform{\event{w}}$. By Lemma
\ref{lemma:path-formula}, $\A$ has an execution $\T$ over $w$ such
that $\I = \I_{\T}$. To prove that $\T$ is accepting, we show
that \begin{inparaenum}[(i)]
\item\label{it1:acceptance} all paths in $\T$ have length $n$ and
  that
\item\label{it2:acceptance} the frontier of $\T$ is labeled with final
  configurations only. \end{inparaenum} First, assume that
(\ref{it1:acceptance}) there exists a path in $\T$ of length $0 \leq m
< n$. Then there exists a node on the $m$-th level, labeled with some
configuration $q(d_1,\ldots,q_{\#(q)})$, that has no children. By the
definition of the execution semantics of $\A$, we have
$\cube{\minsem{\psi}_{\eta}} = \emptyset$, where $q(\vec{y})
\arrow{a_{m+1}(X)}{} \psi$ is the transition rule of $\A$ that applies
for $q$ and $a_{m+1}$ and $\eta= \data{w}[y_1 \leftarrow
  d_1,\ldots,y_{\#(q)} \leftarrow d_{\#(q)}]$. Hence
$\sem{\psi}_{\eta} = \emptyset$, and because $\I,\data{w} \models
\accform{\alpha}$, we obtain that $\I,\eta \models q(\vec{y})
\rightarrow \stamp{\psi}{m+1}$, thus $\tuple{d_1,\ldots,d_{\#(q)}}
\not\in \I(q)$. However, this contradicts the fact that $\I=\I_{\T}$
and that $q(d_1,\ldots,d_{\#(q)})$ labels a node of $\T$. Second,
assume that (\ref{it2:acceptance}), there exists a frontier node of
$\T$ labeled with a configuration $q(d_1,\ldots,d_{\#(q)})$ such that
$q \in Q \setminus F$. Since $\I,\data{w} \models \forall y_1 \ldots
\forall y_{\#(q)} ~.~ q(\vec{y}) \rightarrow \bot$, by a similar
reasoning as in the above case, we obtain that
$\tuple{d_1,\ldots,d_{\#(q)}} \not\in \I(q)$, contradiction.

``$(\ref{it2:lemma:acceptance}) \Rightarrow
(\ref{it1:lemma:acceptance})$'' Let $\T$ be an accepting execution of
$\A$ over $w$. We prove that $\I_\T,\data{w} \models
\accform{\event{w}}$. By Lemma \ref{lemma:path-formula}, we obtain
$\I_\T,\data{w} \models \pathform{\event{w}}$. Since every path in
$\T$ is of length $n$ and all nodes on the $n$-th level of $\T$ are
labeled by final configurations, we obtain that $\I_\T,\data{w}
\models \bigwedge_{q \in Q \setminus F} \forall y_1 \ldots \forall
y_{\#(q)} ~.~ \stamp{q}{n}(\vec{y}) \rightarrow \bot$, trivially.
\qed

\subsection{Proof of Lemma \ref{lemma:quant}}

(\ref{it1:quant}) Trivial, since every subformula
$q(t_1,\ldots,t_{\#(q)}) \rightarrow
\psi[t_1/y_1,\ldots,t_{\#(q)}/y_{\#(q)}]$ of $\quantform{\alpha}$ is
entailed by a subformula $\forall y_1 \ldots \forall y_{\#(q)} ~.~
q(y_1, \ldots, y_{\#(q)}) \rightarrow \psi$ of $\accform{\alpha}$.

\noindent(\ref{it2:quant}) By repeated applications of the following
fact:

\begin{fact}
  Given formulae $\phi$ and $\psi$, such that no predicate atom with
  predicate symbol $q$ occurs in $\psi(y_1,\ldots,y_{\#(q)})$, for
  each valuation $\nu$, if there exists an interpretation $\I$ such
  that $\I,\nu \models \phi \wedge \bigwedge_{q(t_1,\ldots,t_{\#(q)})
    \text{ occurs in } \phi} q(t_1,\ldots,t_{\#(q)}) \rightarrow
  \psi[t_1/y_1,\ldots,t_{\#(q)}/y_{\#(q)}]$ then there exists a
  valuation $\J$ such that $\J(q) \subseteq \I(q)$ and $\J(q') =
  \I(q')$ for all $q' \in Q \setminus \set{q}$ and $\J,\nu \models
  \phi \wedge \forall y_1 \ldots \forall y_{\#(q)} ~.~
  q(y_1,\ldots,y_{\#(q)}) \rightarrow \psi$.
\end{fact}
\paragraph{\em Proof.}
Assume w.l.o.g. that $\phi$ is quantifier-free. The proof can be
easily generalized to the case $\phi$ has quantifiers. Let $\J(q) =
\set{\langle t_1^\nu, \ldots, t_{\#(q)}^\nu \rangle \in \I(q) \mid
  q(t_1,\ldots,t_{\#(q)}) \text{ occurs in } \phi}$ and $\J(q') =
\I(q')$ for all $q' \in Q \setminus \set{q}$. Since $\I,\nu \models
\phi$, we obtain that also $\J,\nu \models \phi$ because the tuples of
values in $\I(q) \setminus \J(q)$ are not interpretations of terms
that occur within subformulae $q(t_1,\ldots,t_{\#(q)})$ of
$\phi$. Moreover, $\bigwedge_{q(t_1,\ldots,t_{\#(q)}) \text{ occurs in
  } \phi} q(t_1,\ldots,t_{\#(q)}) \rightarrow
\psi[t_1/y_1,\ldots,t_{\#(q)}/y_{\#(q)}]$ and $\forall y_1 \ldots
\forall y_{\#(q)} ~.~ q(y_1,\ldots,y_{\#(q)}) \rightarrow \psi$ are
equivalent under $\J$, thus $\J,\nu \models \forall y_{\#(q)} ~.~
q(y_1,\ldots,y_{\#(q)}) \rightarrow \psi$, as required. \qed

\noindent This concludes the proof of Lemma \ref{lemma:quant}. \qed

\subsection{Proof of Lemma \ref{lemma:subst}}

By induction on $n\geq0$. The base case $n=0$ is trivial, since
$\quantform{A} = \substform{A} = \stamp{\iota}{0}$. For the induction
step, we rely on the following fact:

 \begin{fact}\label{fact:subst}
   Given formulae $\phi$ and $\psi$, such that $\phi$ is positive
   $q(t_1,\ldots,t_{\#(q)})$ is the only one occurrence of the
   predicate symbol $q$ in $\phi$ and no predicate atom with
   predicate symbol $q$ occurs in $\psi(y_1,\ldots,y_{\#(q)})$, for
   each interpretation $\I$ and each valuation $\nu$, we have:
   \[\begin{array}{l} 
   \I,\nu \models \phi \wedge q(t_1,\ldots,t_{\#(q)}) \rightarrow
   \psi[t_1/y_1,\ldots,t_{\#(q)}/y_{\#(q)}] \iff \\ \nu \models
   \phi[\psi[t_1/y_1,\ldots,t_{\#(q)}/y_{\#(q)}]/q(t_1,\ldots,t_{\#(q)})]\enspace.
   \end{array}\]
 \end{fact}

 \paragraph{\em Proof.}
 We assume w.l.o.g. that $\phi$ is quantifier-free. The proof can be
 easily generalized to the case $\phi$ has quantifiers.

 \noindent''$\Rightarrow$'' 
    We distinguish two cases: \begin{compactitem}
    \item if $\tuple{t_1^\nu, \ldots, t_{\#(q)}^\nu} \in \I(q)$ then
      $\I,\nu \models \psi[t_1/y_1,\ldots,t_{\#(q)}/y_{\#(q)}]$. Since
      $\phi$ is positive, replacing $q(t_1,\ldots,t_{\#(q)})$ with
      $\psi[t_1/y_1,\ldots,t_{\#(q)}/y_{\#(q)}]$ does not change the
      truth value of $\phi$ under $\nu$, thus $\nu \models
      \phi[\psi[t_1/y_1,\ldots,t_{\#(q)}/y_{\#(q)}]/q(t_1,\ldots,t_{\#(q)})]$.
      %
    \item else, $\tuple{t_1^\nu, \ldots, t_{\#(q)}^\nu} \not\in
      \I(q)$, thus $\nu \models
      \phi[\bot/q(t_1,\ldots,t_{\#(q)})]$. Since $\phi$ is positive
      and $\bot$ entails $\psi[t_1/y_1,\ldots,t_{\#(q)}/y_{\#(q)}]$,
      we obtain $\nu \models
      \phi[\psi[t_1/y_1,\ldots,t_{\#(q)}/y_{\#(q)}]/q(t_1,\ldots,t_{\#(q)})]$
      by monotonicity. 
    \end{compactitem}

    \noindent``$\Leftarrow$'' Let $\I$ be any interpretation such that
    $\I(q) = \set{\tuple{t_1^\nu, \ldots, t_{\#(q)}^\nu} \mid \nu
      \models \psi[t_1/y_1,\ldots,t_{\#(q)}/y_{\#(q)}]}$. We
    distinguish two cases: \begin{compactitem}
    \item if $\I(q) \not \emptyset$ then $\I,\nu \models
      q(t_1,\ldots,t_{\#(q)})$ and $\nu \models \psi[t_1/y_1, \ldots,
        t_{\#(q)}/y_{\#(q)}]$. Thus replacing $\psi[t_1/y_1, \ldots,
        t_{\#(q)}/y_{\#(q)}]$ by $q(t_1,\ldots,t_{\#(q)})$ does not
      change the truth value of $\phi$ under $\I$ and $\nu$, and we
      obtain $\I,\nu \models \phi$. Moreover, $\I,\nu \models
      \psi[t_1/y_1, \ldots, t_{\#(q)}/y_{\#(q)}]$ implies $\I,\nu
      \models q(t_1,\ldots,t_{\#(q)}) \rightarrow \psi[t_1/y_1,
        \ldots, t_{\#(q)}/y_{\#(q)}]$.
        %
    \item else $\I(q) = \emptyset$, hence $\nu \not\models
      \psi[t_1/y_1, \ldots, t_{\#(q)}/y_{\#(q)}]$, thus $\nu \models
      \phi[\bot/q(t_1,\ldots,t_{\#(q)})]$. Because $\phi$ is positive,
      we obtain $\I,\nu \models \phi$ by monotonicity. But $\I,\nu
      \models q(t_1,\ldots,t_{\#(q)}) \rightarrow \psi[t_1/y_1,
        \ldots, t_{\#(q)}/y_{\#(q)}]$ trivially, because $\I,\nu
      \not\models q(t_1,\ldots,t_{\#(q)})$. \qed
      \end{compactitem}
      
This concludes the proof of Lemma \ref{lemma:subst}. \qed

\subsection{Proof of Lemma \ref{lemma:quant-pred-acceptance}}

By Lemma \ref{lemma:acceptance}, $w \in \lang{\A}$ if and only if
$\I,\data{w} \models \accform{\event{w}}$, for some interpretation
$\I$. By Lemma \ref{lemma:quant}, there exists an interpretation $\I$
such that $\I,\data{w} \models \accform{\event{w}}$ if and only if
there exists an interpretation $\J$ such that $\J,\nu \models
\quantform{\event{w}}$. By Lemma \ref{lemma:subst}, there exists an
interpretation $\J$ such that $\J,\nu \models \quantform{\event{w}}$
if and only if $\nu \models \substform{\event{w}}$. \qed

\subsection{Proof of Theorem \ref{thm:soundness}}

Let $U$ be a safe and complete unfolding of $\A$, such that $\dom(U)
\neq \emptyset$. Suppose, by contradiction, that there exists a word
$w \in \lang{A}$ and let $\alpha \isdef \event{w}$. Since $w \in
\lang{A}$, by Lemma \ref{lemma:acceptance}, there exists an
interpretation $\I$ such that $\I,\data{w} \models
\accform{\alpha}$. Assume first that $\alpha \in \dom(U)$.  In this
case, one can show, by induction on the length $n\geq0$ of $w$, that
$\pathform{\alpha} \models \stamp{U(\alpha)}{n}$, thus $\I,\data{w}
\models \stamp{U(\alpha)}{n}$. Since $\I,\data{w} \models
\accform{\alpha}$, we have $\I,\data{w} \models \bigwedge_{q \in Q
  \setminus F} \forall y_1 \ldots \forall y_{\#(q)} ~.~
\stamp{q}{n}(\vec{y}) \rightarrow \bot$, hence $\stamp{U(\alpha)}{n}
\wedge \bigwedge_{q \in Q \setminus F} \forall y_1 \ldots \forall
y_{\#(q)} ~.~ \stamp{q}{n}(\vec{y}) \rightarrow \bot$. By renaming
$\stamp{q}{n}$ with $q$ in the previous formula, we obtain $U(\alpha)
\wedge \forall y_1 \ldots \forall y_{\#(q)} ~.~ q(\vec{y}) \rightarrow
\bot$ is satisfiable, thus $U$ is not safe, contradiction.

We proceed thus under the assumption that $\alpha \not\in
\dom(U)$. Since $\dom(U)$ is a nonempty prefix-closed set, there
exists a strict prefix $\alpha'$ of $\alpha$ that is a leaf of
$\dom(U)$. Since $U$ is closed, the leaf $\alpha'$ must be covered and
let $\alpha_1 \prefix \alpha' \prefix \alpha$ be a node such that
$U(\alpha_1) \models U(\beta_1)$, for some uncovered node $\beta_1 \in
\dom(U)$. Let $\gamma_1$ be the unique sequence such that
$\alpha_1\gamma_1 = \alpha$. By Definition \ref{def:coverage}, since
$\alpha_1 \cover \beta_1$ and $\event{w} = \alpha_1\gamma_1 \in
\lang{\A}$, there exists a word $w_1$ and a cube $c_1 \in
\cube{\sem{U(\alpha_1)}} \subseteq \cube{\sem{U(\beta_1)}}$, such that
$\event{w_1} = \gamma_1$ and $\A$ accepts $w_1$ starting with
$c_1$. If $\beta_1\gamma_1 \in \dom(U)$, we obtain a contradiction by
a similar argument as above. Hence $\beta_1\gamma_1 \not\in \dom(U)$
and there exists a leaf of $\dom(U)$ which is also a prefix of
$\beta_1\gamma_1$. Since $U$ is closed, this leaf is covered by an
uncovered node $\beta_2 \in \dom(U)$ and let $\alpha_2 \in \dom(U)$ be
the minimal (in the prefix partial order) node such that $\beta_1
\prefix \alpha_2 \prefix \beta_1\gamma_1$ and $\alpha_2 \cover
\beta_2$. Let $\gamma_2$ be the unique sequence such that
$\alpha_2\gamma_2 = \beta_1\gamma_1$. Since $\beta_1$ is uncovered, we
have $\beta_1 \neq \alpha_2$ and thus $\len{\gamma_1} >
\len{\gamma_2}$. By repeating the above reasoning for $\alpha_2$,
$\beta_2$ and $\gamma_2$, we obtain an infinite sequence
$\len{\gamma_1} > \len{\gamma_2} > \ldots$, which is again a
contradiction. \qed

\subsection{Proof of Lemma \ref{lemma:subst-quant}}

(\ref{it1:subst-quant}) If $\accform{\alpha}$ is unsatisfiable, by
Lemmas \ref{lemma:quant} and \ref{lemma:subst}, we obtain that,
successively $\quantform{\alpha}$ and $\substform{\alpha}$ are
unsatisfiable. Let $Q_1x_1 \ldots Q_mx_m ~.~ \widehat{\Phi}$ and
$Q_1x_1 \ldots Q_mx_m ~.~ \overline{\Phi}$ be prenex forms for
$\quantform{\alpha}$ and $\substform{\alpha}$, respectively. Since we
assumed that the first order theory of the data domain has
witness-producing quantifier elimination, using Theorem
\ref{thm:skolem} one can effectively build witness terms $\tau_{i_1},
\ldots, \tau_{i_\ell}$, where $\set{i_1,\ldots,i_\ell}=\set{i \in
  [1,m] \mid Q_i=\forall}$ and: \begin{compactitem}
\item $\voc{\tau_{i_j}} \subseteq \stamp{X}{\leq \xi(i_j)} \cup
  \set{x_k \mid k < i_j, Q_k = \exists}$, for all $j \in [1,\ell]$ and
  %
\item $\overline{\Phi}[\tau_{i_1}/x_{i_1}, \ldots,
  \tau_{i_\ell}/x_{i_\ell}]$ is unsatisfiable.
\end{compactitem}
Let $\widehat{\Phi}_0$, \ldots, $\widehat{\Phi}_n$ be the sequence of
quantifier-free formulae, defined as follows: \begin{compactitem}
\item $\widehat{\Phi}_0$ is the matrix of some prenex form of
  $\stamp{\iota}{0}$,
  %
\item for all $i=1,\ldots,n$, let $\widehat{\Phi}_i$ be the matrix of
  some prenex form of:
  \[\widehat{\Phi}_i \isdef \widehat{\Phi}_{i-1} \wedge
  \underbrace{\bigwedge_{\begin{array}{l}
        \scriptstyle{\stamp{q}{i-1}(t_1,\ldots,t_{\#(q)}) \text{ occurs in } \widehat{\Phi}_{i-1}} \\
        \scriptstyle{q(y_1,\ldots,y_{\#(q)}) \arrow{a_i(X)}{} \psi \in \Delta} 
    \end{array}} \stamp{q}{i-1}(t_1,\ldots,t_{\#(q)}) \rightarrow 
    \stamp{\psi}{i}[t_1/y_1, \ldots, t_{\#(q)}/y_{\#(q)}]}_{\isdef \phi_i}\]  
\end{compactitem}
It is easy to see that $\widehat{\Phi}$ is the matrix of some prenex
form of:
\[\widehat{\Phi}_n \wedge \underbrace{\bigwedge_{\begin{array}{l}
      \scriptstyle{\stamp{q}{n}(t_1,\ldots,t_{\#(q)}) \text{ occurs in } \widehat{\Phi}_n} \\
      \scriptstyle{q \in Q \setminus F}
  \end{array}} \stamp{q}{n}(t_1,\ldots,t_{\#(q)}) \rightarrow \bot}_{\isdef \psi}\]
Applying the equivalence from Fact \ref{fact:subst} in the proof of
Lemma \ref{lemma:subst}, we obtain a sequence of quantifier-free
formulae $\overline{\Phi}_0, \ldots, \overline{\Phi}_n$ such that
$\overline{\Phi}_i \equiv \widehat{\Phi}_i$, for all $i \in [n]$ and
$\overline{\Phi}$ is obtained from $\overline{\Phi}_n$ by replacing
each occurrence of a predicate atom $q(t_1,\ldots,t_{\#(q)})$ in
$\overline{\Phi}_n$ by $\bot$ if $q \in Q\setminus F$ and by $\top$ if
$q \in F$. Clearly $\overline{\Phi} \equiv \widehat{\Phi}$, thus
$\widehat{\Phi}[\tau_{i_1}/x_{i_1}, \ldots, \tau_{i_\ell}/x_{i_\ell}]
\equiv \overline{\Phi}[\tau_{i_1}/x_{i_1}, \ldots,
  \tau_{i_\ell}/x_{i_\ell}] \equiv \bot$.

\noindent(\ref{it2:subst-quant}) With the notation introduced at point
(\ref{it1:subst-quant}), we have $\widehat{\Phi} = \widehat{\Phi}_0
\wedge \bigwedge_{i=1}^n \phi_i \wedge \psi$. Consider the sequence of
witness terms $\tau_{i_1}, \ldots, \tau_{i_\ell}$, whose existence is
proved by point (\ref{it1:subst-quant}). Because $\voc{\tau_{i_j}}
\subseteq \stamp{X}{\leq \xi(i_j)} \cup \set{x_k \mid k < i_j, Q_k =
  \exists}$, for all $j \in [1,\ell]$, and moreover $\xi^{-1}$ is
strictly monotonic, we obtain: \begin{compactitem}
\item $\voc{\widehat{\Phi}_0[\tau_{i_1}/x_{i_1}, \ldots,
    \tau_{i_\ell}/x_{i_\ell}]} \subseteq \stamp{Q}{0} \cup
  \stamp{X}{0} \cup \set{x_j \mid j < \xi_{\max}^{-1}(0), Q_j = \exists}$,
  %
\item $\voc{\phi_i[\tau_{i_1}/x_{i_1}, \ldots,
    \tau_{i_\ell}/x_{i_\ell}]} \subseteq \stamp{Q}{i-1} \cup
  \stamp{Q}{i} \cup \stamp{X}{\leq i} \cup \set{x_j \mid j <
    \xi_{\max}^{-1}(i), Q_j = \exists}$, for all $i \in [1,n]$,
  %
\item $\voc{\psi[\tau_{i_1}/x_{i_1}, \ldots,
    \tau_{i_\ell}/x_{i_\ell}]} \subseteq \stamp{Q}{n} \cup
  \stamp{X}{\leq n} \cup \set{x_j \mid j \in [1,m], Q_j = \exists}$.
\end{compactitem}
By repeatedly applying Lyndon's Interpolation Theorem, we obtain a
sequence of formulae $(I_0, \ldots, I_n)$ such
that: \begin{compactitem}
\item $\widehat{\Phi}_0[\tau_{i_1}/x_{i_1}, \ldots,
  \tau_{i_\ell}/x_{i_\ell}] \models I_0$ and $\voc{I_0} \subseteq
  \stamp{Q}{0} \cup \stamp{X}{0} \cup \set{x_j \mid j <
    \xi_{\max}^{-1}(0), Q_j = \exists}$,
  %
\item $I_{k-1} \wedge \phi_i[\tau_{i_1}/x_{i_1}, \ldots,
  \tau_{i_\ell}/x_{i_\ell}] \models I_k$ and $\voc{I_k} \subseteq
  \stamp{Q}{k} \cup \stamp{X}{\leq k} \cup \set{x_j \mid j <
    \xi_{\max}^{-1}(k), Q_j = \exists}$, for all $k \in [1,n]$,
  %
\item $I_n \wedge \psi[\tau_{i_1}/x_{i_1}, \ldots,
  \tau_{i_\ell}/x_{i_\ell}]$ is unsatisfiable.
\end{compactitem}
To show that $(I_0,\ldots,I_n)$ is a GLI for $a_1\ldots a_n$, it is
sufficient to notice that \[\bigwedge_{q(\vec{y}) \arrow{a_k(X)}{}
  \psi \in \Delta} \forall y_1 \ldots \forall y_{\#(q)} ~.~
\stamp{q}{k}(\vec{y}) \rightarrow \stamp{\psi}{k+1} \models \phi_k\]
for all $k \in [1,n]$. Consequently, we obtain: \begin{compactitem}
\item $\stamp{\iota}{0} \models \widehat{\Phi}_0 \models I_0$, by
  Theorem \ref{thm:skolem},
  %
\item $I_{k-1} \wedge \Big(\bigwedge_{q(\vec{y}) \arrow{a_k(X)}{}
  \psi \in \Delta} \forall y_1 \ldots \forall y_{\#(q)} ~.~
  \stamp{q}{k-1}(\vec{y}) \rightarrow \stamp{\psi}{k}\Big) \models
  I_{k-1} \wedge \phi_k \models I_k$, and 
  %
\item $I_n \wedge \Big(\bigwedge_{q \in Q \setminus F} \forall y_1
  \ldots \forall y_{\#(q)} ~.~ q(\vec{y}) \rightarrow \bot\Big)
  \models I_n \wedge \psi \models \bot$,     
\end{compactitem}
as required by Definition
\ref{def:generalized-lyndon-interpolant}. \qed

\subsection{Proof of Lemma \ref{lemma:refinement}}

The new set of formulae $U'(\alpha_0), \ldots, U'(\alpha_n)$ complies
with Definition \ref{def:unfolding}, because: \begin{compactitem}
\item $U'(\alpha_0) \equiv \iota$, since, by point
  \ref{it2:generalized-lyndon-interpolant} of Definition
  \ref{def:generalized-lyndon-interpolant}, we have $\stamp{\iota}{0}
  \models I_0$, thus $\iota \models J_0$ and $U'(\alpha_0) =
  U(\alpha_0) \wedge J_0 \equiv \iota \wedge J_0 \equiv \iota$, and
    %
\item by Definition \ref{def:generalized-lyndon-interpolant}
  (\ref{it3:generalized-lyndon-interpolant}) we have, for all $k \in
  [n-1]$:
  \[I_k \wedge \bigwedge_{\scriptscriptstyle{q(\vec{y}) \arrow{a_k(X)}{} \psi \in \Delta}} 
  \forall y_1 \ldots \forall y_{\#(q)} ~.~ \stamp{q}{k}(\vec{y})
  \rightarrow \stamp{\psi}{k+1} \models I_{k+1}\] We write
  $\overstamp{I}{j}_k$ for the formula in which each predicate
  symbol $\stamp{q}{k}$ is replaced by $\stamp{q}{j}$. Then the
  following entailment holds:
  \[\overstamp{I}{0}_k \wedge \bigwedge_{\scriptscriptstyle{q(\vec{y}) \arrow{a_k(X)}{} \psi \in \Delta}} 
  \forall y_1 \ldots \forall y_{\#(q)} ~.~ \stamp{q}{0}(\vec{y})
  \rightarrow \stamp{\psi}{1} \models \overstamp{I}{1}_{k+1}\]
  Because $J_k$ is obtained by removing the time stamps from the predicate symbols and  
  existentially quantifying all the free variables of $I_k$, we also
  obtain, applying Fact \ref{fact:ex-entail} below:
  \[\stamp{J}{0}_k \wedge \bigwedge_{\scriptscriptstyle{q(\vec{y}) \arrow{a_k(X)}{} \psi \in \Delta}} 
  \forall y_1 \ldots \forall y_{\#(q)} ~.~ \stamp{q}{0}(\vec{y})
  \rightarrow \stamp{\psi}{1} \models \stamp{J}{1}_{k+1}\] Since $U$
  satisfies the labeling condition of Definition \ref{def:unfolding}
  and $U'(\alpha_k) = U(\alpha_k) \wedge J_k$, we obtain, as required:
  \[\stamp{U'(\alpha_k)}{0} \wedge \bigwedge_{\scriptscriptstyle{q(\vec{y}) \arrow{a_k(X)}{} \psi \in \Delta}} 
  \forall y_1 \ldots \forall y_{\#(q)} ~.~ \stamp{q}{0}(\vec{y})
  \rightarrow \stamp{\psi}{1} \models
  \stamp{U'(\alpha_{k+1})}{1}\enspace.\]
\end{compactitem}  

\begin{fact}\label{fact:ex-entail}
  Given formulae $\phi(\vec{x}, \vec{y})$ and $\psi(\vec{x})$ such
  that $\phi(\vec{x}, \vec{y}) \models \psi(\vec{x})$, we also have
  $\exists \vec{x} ~.~ \phi(\vec{x},\vec{y}) \models \exists \vec{x}
  ~.~ \psi(\vec{x})$. 
\end{fact}
\paragraph{\em Proof.} For each choice of a valuation for the existentially
quantified variables on the left-hand side, we chose the same
valuation for the variables on the right-hand side. \qed 
