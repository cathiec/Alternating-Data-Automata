\documentclass{llncs}
% \documentclass[10pt,conference,letterpaper,twocolumn]{IEEEtran}

\usepackage[usenames,dvipsnames]{color}

\usepackage{latexsym}
\usepackage{amsxtra} 
\usepackage{amssymb}
\usepackage{amsmath}
\usepackage{pslatex}
\usepackage{epsfig}
\usepackage{wrapfig}
\usepackage{paralist}
% \usepackage{stmaryrd}
\usepackage{txfonts}
\usepackage{framed}
\usepackage{makecell}
\usepackage{url}
\usepackage{tikz}
\usetikzlibrary{automata,positioning, calc}
\usepackage[inline,shortlabels]{enumitem}

\usepackage{proof}

\usepackage{algorithm}
\usepackage{algorithmicx}
\usepackage[noend]{algpseudocode}

\usepackage[draft]{commenting}

\pagestyle{plain}

%\newtheorem{definition}{Definition}
%\newtheorem{lemma}{Lemma}
%\newtheorem{theorem}{Theorem}
%\newtheorem{proposition}{Proposition}
%\newtheorem{corollary}{Corollary}
%\newtheorem{conjecture}{Conjecture}
%\newtheorem{claim}{Claim}
%\newtheorem{property}{Property}
\newtheorem{fact}{Fact}

\newcommand*\circled[1]{\tikz[baseline=(char.base)]{
            \node[shape=circle,draw,inner sep=0.5pt] (char) {#1};}}

% Marius commands
\newcommand{\Z}{{\mathbb Z}}
\newcommand{\N}{{\mathbb N}}
\newcommand{\relmat}[1] { \{\!\!\{#1\}\!\!\}}
\newcommand{\matrel}[1] { [\!\![#1]\!\!]}
\newcommand{\key}[1]{\underline{#1}}

% complexity 
\newcommand{\np}{$\mathsf{NP}$}
\newcommand{\pspace}{$\mathsf{PSPACE}$~}
\newcommand{\sigmatwop}{$\mathsf{\Sigma}_2^p$}
\newcommand{\pitwop}{$\mathsf{\Pi}_2^p$}

% semantic stuff
\newcommand{\bi}[1]{{}^\omega\!{#1}^\omega}
\newcommand{\rbr}{{\bf ]\!]}}
\newcommand{\lbr}{{\bf [\![}}
\newcommand{\sem}[1]{\lbr #1 \rbr}
\newcommand{\model}[2]{\sem{#1}_{#2}}
\newcommand{\lfp}[1]{\mbox{fix}\ {#1}}
\newcommand{\wpr}{\widetilde{pre}}
\newcommand{\cceil}[1]{\lfloor #1 \rfloor + 1}
\newcommand{\set}[1]{\{ #1 \}}
\newcommand{\tuple}[1]{\langle #1 \rangle}
\renewcommand{\vec}[1]{\mathbf #1}
\newcommand{\tup}[1]{\overline #1}
\newcommand{\ite}[3]{\mathsf{ite}(#1,#2,#3)}
\newcommand{\tupite}[2]{\mathsf{ite}(#1,#2)}
\newcommand{\minsem}[1]{\sem{#1}^\mu}

% alias
\newcommand{\alias}{\Diamond}
\newcommand{\may}[2]{{\langle #1 \rangle}{#2}}
\newcommand{\must}[2]{{\lbrack #1 \rbrack}{#2}}
\newcommand{\defd}[1]{\delta({#1})}

% definitions
\newcommand{\isdef}{\stackrel{\scriptscriptstyle{\mathsf{def}}}{=}}
\renewcommand{\iff}{\Leftrightarrow}

% algorithms
\newcommand{\sig}{\mathsf{S}}
\newcommand{\stamp}[2]{{#1}^{\scriptscriptstyle{({#2})}}}
\newcommand{\overstamp}[2]{{#1}^{\scriptscriptstyle{\langle{#2}\rangle}}}

% recursive structures
\newcommand{\LS}{\mathtt{LS}}
\newcommand{\DLL}{\mathtt{DLL}}
\newcommand{\TLL}{\mathtt{TLL}}
\newcommand{\TREE}{\mathtt{TREE}}
\newcommand{\ax}{\mathtt{a}}
\newcommand{\bx}{\mathtt{b}}
\newcommand{\cx}{\mathtt{c}}
\newcommand{\dx}{\mathtt{d}}

% cardinal, degree, ...
\newcommand{\sgn}{\mbox{sgn}}
%\newcommand{\gcd}{\mbox{gcd}}
\newcommand{\lcm}{\mbox{lcm}}
\newcommand{\len}[1]{{|{#1}|}}
\newcommand{\card}[1]{{|\!|{#1}|\!|}}
\newcommand{\minimum}[2]{\mbox{min}({#1},{#2})}
\newcommand{\maximum}[2]{\mbox{max}({#1},{#2})}
\newcommand{\minset}[2]{\mbox{min}\{{#1}\ |\ {#2}\}}
\newcommand{\maxset}[2]{\mbox{max}\{{#1}\ |\ {#2}\}}
\newcommand{\divs}[1]{\mbox{div}({#1})}

% automata and languages
\newcommand{\auth}[1]{{\mathcal A}_{#1}}
\newcommand{\lang}[1]{{\mathcal L}({#1})}
\newcommand{\runs}[1]{{\mathcal R}({#1})}
\newcommand{\val}[1]{{\mathcal V}({#1})}
\newcommand{\trace}[1]{{Tr}({#1})}
\newcommand{\state}[1]{{\Sigma}({#1})}
\newcommand{\trans}[1]{{\Theta}({#1})}
\newcommand{\access}[3]{{\mathcal L}_{{#1},{#2}}({#3})}
\newcommand{\eats}{\rightarrow}
\newcommand{\Eats}[1]{\stackrel{#1}{\longrightarrow}}
\newcommand{\arrow}[2]{\xrightarrow{{\scriptscriptstyle #1}}_{{\scriptstyle #2}}}
\newcommand{\larrow}[2]{\xleftarrow[{\scriptscriptstyle #2}]{{\scriptstyle #1}}} 
\newcommand{\darrow}[2]{\stackrel{\scriptscriptstyle #1}{\Longrightarrow}_{\!\!\scriptscriptstyle #2}}
\newcommand{\harrow}[2]{\stackrel{\scriptscriptstyle #1}{\leadsto}_{\!\!\scriptscriptstyle #2}}

% numbers
\newcommand{\nat}{{\bf \mathbb{N}}}
\newcommand{\zed}{{\bf \mathbb{Z}}}
\newcommand{\rat}{{\bf \mathbb{Q}}}
\newcommand{\real}{{\bf \mathbb{R}}}
\newcommand{\linzed}{{\bf \mathcal{L}\mathbb{Z}}}

% powerset
\newcommand{\pow}[1]{{\mathcal P}(#1)}
\newcommand{\pown}[2]{{\mathcal P}^{#1}(#2)}

% by definition
\newcommand{\bydef}{\stackrel{\Delta}{=}}

% projection, lifting
\newcommand{\proj}[2]{{#1}\!\!\downarrow_{{#2}}}
\newcommand{\lift}[2]{{#1}\!\!\uparrow_{{#2}}}

% forward, backward
\newcommand{\forw}[1]{\stackrel{\rightarrow}{#1}}
\newcommand{\back}[1]{\stackrel{\leftarrow}{#1}}

% equation figures
\newcommand{\eqnfigx}[1]{
\begin{minipage}{3in}
\begin{center}
\begin{eqnarray*}#1
\end{eqnarray*}
\end{center}
\end{minipage}}

\newcommand{\twoeqnfigx}[2]{
\eqnfigx{#1}
\eqnfigx{#2}
}

\newcommand{\threeeqnfigx}[3]{
\eqnfigx{#1}\\
\eqnfigx{#2}\eqnfigx{#3}
}

\newcommand{\eqnfig}[1]{
\begin{minipage}{3in}
\begin{center}
\begin{eqnarray}#1
\end{eqnarray}
\end{center}
\end{minipage}}

\newcommand{\twoeqnfig}[2]{
\eqnfig{#1}
\eqnfig{#2}
}

% transitions and SOS rules
\newcommand{\storetrans}[1]{\stackrel{#1}{\leadsto}}
\newcommand{\storetransx}[1]{\stackrel{#1}{\leadsto^*}}
\newcommand{\storetransprime}[3]{{#1} \storetrans{#2} {#3}}
\newcommand{\gammatrans}[1]{\stackrel{#1}{\hookrightarrow}}
\newcommand{\gammatransprime}[3]{{#1} \gammatrans{#2} {#3}}
\newcommand{\gammatransx}[1]{\stackrel{#1}{\hookrightarrow^*}}
%\newcommand{\trans}[3]{{#1} \stackrel{#2}{\longrightarrow} {#3}}
\newcommand{\run}[3]{{#1} \stackrel{#2}{\Longrightarrow} {#3}}
\newcommand{\sosrule}[2]{\frac{\begin{array}{c}#1\end{array}}{#2}}
\newcommand{\hoare}[3]{\{#1\}\ {\bf #2}\ \{#3\}}

\newcommand{\derivone}[3]{\mbox{\infer[#3]{#2}{#1}}}
\newcommand{\derivtwo}[4]{\mbox{\infer[#4]{#3}{{#1} & {#2}}}}
\newcommand{\derivdot}[4]{\mbox{\infer[#4]{#3}
{\begin{array}{l} {#1} \\ \vdots \\ {#2} \end{array}}}}

% comments, paragraphs
\newcommand{\mypar}[1]{\vspace*{\baselineskip}\noindent\emph{#1}}
\renewcommand{\paragraph}[1]{\noindent{\bf #1}}

% theories
\newcommand{\arith}{\langle \zed, +, \cdot, 0, 1 \rangle}
\newcommand{\hilbert}{\langle \zed, +, \cdot, 0, 1 \rangle^\exists}
\newcommand{\presbg}{\langle \zed, \geq, +, 0, 1 \rangle}
% \newcommand{\divides}{\langle \zed, +, |, 0, 1 \rangle}
\newcommand{\lipshitz}{\langle \zed, \geq, +, |, 0, 1 \rangle^\exists}
\newcommand{\dioph}[1]{\mathfrak{D}(#1)}

\let\Asterisk\undefined
\newcommand{\Asterisk}{\mathop{\scalebox{1.7}{\raisebox{-0.2ex}{$\ast$}}}}%

% proofs
\newif\ifLongVersion\LongVersiontrue
%\newcommand{\proof}[1]{\noindent {\em Proof}: {#1}}
%\newcommand{\qed}{\hfill$\Box$}
%\newcommand{\inproof}[1]{\ifLongVersion \proof{#1} \fi}
%\newcommand{\outproof}[2]{\ifLongVersion \else \vspace*{\baselineskip}\noindent Proof of {#1}: \proof{#2} \fi}
%\renewcommand{\subsubsection}[1]{\noindent{\bf #1}}

% multisorted first-order logic
\newcommand{\sorts}[1]{#1^\mathrm{S}}
\newcommand{\funs}[1]{#1^\mathrm{F}}
\newcommand{\ssorts}[1]{#1^\mathrm{s}}
\newcommand{\sfuns}[1]{#1^\mathrm{f}}
\newcommand{\sort}[1]{\sigma({#1})}
\newcommand{\mods}{\mathbf{I}}
\newcommand{\modsof}[2]{\sem{#1}_{#2}}
\newcommand{\iffeq}{\Leftrightarrow}

% separation logic
\newcommand{\symhp}{\Pi}
\newcommand{\symhs}{\Theta}
\newcommand{\locsi}{\mathsf{L}}
\newcommand{\heaps}{\mathsf{Heaps}}
\newcommand{\teq}{\approx}
\newcommand{\Spatial}{\Sigma}
\newcommand{\Emp}{\mathbf{emp}}
\newcommand{\PointsTo}[2]{ {#1 \mapsto #2} }
\newcommand{\PointsToL}[3]{ {#2 \mapsto^{#1} #3} }
\newcommand{\I}{\mathcal{I}}
\newcommand{\J}{\mathcal{J}}
\newcommand{\G}{\mathcal{G}}
\newcommand{\K}{\mathcal{K}}
\newcommand{\herb}{\mathcal{H}}
\newcommand{\tinyherb}{{\scriptscriptstyle\mathcal{H}}}
\newcommand{\locs}{\mathsf{Loc}}
\newcommand{\nil}{\mathsf{nil}}
\newcommand{\emp}{\mathsf{emp}}
\newcommand{\wand}{
 \mathrel{\mbox{$\hspace*{-0.03em}\mathord{-}\hspace*{-0.66em}
 \mathord{-}\hspace*{-0.36em}\mathord{*}$\hspace*{-0.005em}}}} % {\multimap}
\newcommand{\seplog}{\mathsf{SL}}
\newcommand{\tinyseplog}{\mathsf{\scriptscriptstyle{SL}}}
\newcommand{\seplogl}[1]{\mathsf{SL}\parens{#1}}
\newcommand{\tterm}{\mathsf{t}}
\newcommand{\uterm}{\mathsf{u}}
\newcommand{\vterm}{\mathsf{v}}
\newcommand{\wterm}{\mathsf{w}}
\newcommand{\xterm}{\mathsf{x}}
\newcommand{\yterm}{\mathsf{y}}
\newcommand{\zterm}{\mathsf{z}}
\newcommand{\nextp}{\mathsf{next}}
\newcommand{\datap}{\mathsf{data}}
\newcommand{\lterm}{\ell}
\newcommand{\pto}{\mathsf{pt}}
\newcommand{\sets}{\mathsf{Set}}
\newcommand{\lab}{\triangleleft}
\newcommand{\labd}[1]{#1\!\!\Downarrow}
\newcommand{\thsep}{\mathsf{Sep}}
\newcommand{\fv}[2]{\mathrm{FV}_{#1}(#2)}
\newcommand{\pset}[2]{\mathrm{P}^{#1}(#2)}
\newcommand{\voc}[1]{\mathrm{V}({#1})}
\newcommand{\terms}[1]{\mathrm{Pt}(#1)}
\newcommand{\bounds}[2]{\mathrm{Bnd}(#1,#2)}
\newcommand{\dom}{\mathrm{dom}}
\newcommand{\img}{\mathrm{img}}
\newcommand{\size}[1]{\mathrm{Size}(#1)}
\newcommand{\vars}{\mathsf{Var}}
\newcommand{\preds}{\mathsf{Pred}}
\newcommand{\X}{\mathcal{X}}
\newcommand{\Y}{\mathcal{Y}}
\newcommand{\sys}{\mathcal{S}}
\newcommand{\lseq}{\Gamma}
\newcommand{\rseq}{\Delta}
\newcommand{\psys}{\mathcal{P}}
\newcommand{\cf}{\mathcal{F}}
\newcommand{\sw}{\leq_{\mathrm{sw}}}
\newcommand{\subst}[1]{\mathrm{VIS}(#1)}
\newcommand{\DS}{\mathfrak{D}^{\mathsf{\scriptscriptstyle{s}}}}

\newcommand{\infname}[1]{\begin{array}{c}\text{ ($#1$) }\\\\\end{array}}

\newcommand{\subtree}{\sqsubset}
\newcommand{\subtreeq}{\sqsubseteq}
\newcommand{\suptree}{\sqsupset}

\newcommand{\ta}{\mathrm{TA}}
\newcommand{\LU}{\mathrm{LU}}
\newcommand{\RU}{\mathrm{RU}}
\newcommand{\RD}{\mathrm{RD}}
\newcommand{\RI}{\wedge\mathrm{R}}
\newcommand{\SP}{\mathrm{SP}}
\newcommand{\ID}{\mathrm{ID}}
\newcommand{\AX}{\mathrm{AX}}
\newcommand{\Aone}{\mathrm{A1}}
\newcommand{\Atwo}{\mathrm{A2}}
\newcommand{\perms}{\mathsf{Perm}}

\newcommand{\Label}[1]{ \llbracket #1 \rrbracket_{L} }
\DeclareMathOperator{\Disjoint}{disjoint}
\newcommand{\parens}[1]{ \left( #1 \right) }
\newcommand{\Set}[1]{ \left\{ #1 \right\} }
\DeclareMathOperator{\CardOp}{card}
\newcommand{\Card}[1]{ \CardOp\parens{#1} }
\DeclareMathOperator{\Wit}{witness}
\DeclareMathOperator{\Choice}{Choice}
\DeclareMathOperator{\Explain}{explain}

\newcommand{\funcsolve}{\small\mathsf{solve}}
\newcommand{\funcsmtsolve}{\small\mathsf{solve\_rec}}

% \newcommand{\lan}{\mathbf{L}}
\newcommand{\purify}[1]{\lfloor #1 \rfloor}
\newcommand{\purifyrec}[1]{\purify{#1}^\ast}
\newcommand{\unpurify}[1]{\lceil #1 \rceil}
\newcommand{\quants}[1]{\mathrm{Q}(#1)}
\newcommand{\solvesl}[1]{\mathsf{solve_{SL(#1)}}}
\newcommand{\solveslt}[1]{\mathsf{solve_{#1}}}
\newcommand{\lblf}{\mathsf{lbl}}
\newcommand{\ltrue}{\top}
\newcommand{\lfalse}{\bot}
\newcommand{\ord}[2]{\prec_{#1,#2}}
\newcommand{\qvar}{x}
\newcommand{\Int}{\mathbb{Z}}
\newcommand{\Data}{\mathbb{D}}
\newcommand{\Bool}{\mathbb{B}}
\newcommand{\true}{\mathtt{t}}
\newcommand{\false}{\mathtt{f}}
\newcommand{\euf}{\mathsf{E}}
\newcommand{\lia}{\mathsf{LIA}}
\newcommand{\euflia}{\mathsf{ELIA}}
\newcommand{\tinylia}{\mathsf{\scriptscriptstyle{LIA}}}
\newcommand{\ul}{\mathsf{UL}}
\newcommand{\ulk}{\mathsf{UL}^{\!\!\scriptscriptstyle{K}}}
\newcommand{\il}{\mathsf{IL}}
\newcommand{\ulid}{\mathsf{ULID}}
\newcommand{\theory}{\mathbb{T}}
\newcommand{\dual}[1]{{#1}^{\sim}}
\newcommand{\bigO}[1]{\mathcal{O}({#1})}

% automata
\newcommand{\A}{\mathcal{A}}
\newcommand{\Post}[1]{\mathsf{Post}({#1})}
\newcommand{\Accept}[2]{\mathsf{Acc}_{\scriptscriptstyle{#1}}({#2})}
\newcommand{\AbsPost}[1]{\abs{\mathsf{Post}}_{#1}}
\newcommand{\AbsAccept}[1]{\abs{\mathsf{Acc}}_{#1}}
\newcommand{\Inv}{\mathsf{I}}
\newcommand{\abs}[1]{{#1}^\sharp}
\newcommand{\Art}{\mathcal{T}}
\newcommand{\SubArt}{\mathcal{S}}
\newcommand{\worklist}{\mathtt{WorkList}}
\newcommand{\rootNode}{\mathtt{r}}
\newcommand{\pivot}[1]{\Psi(#1)}
\newcommand{\impact}{\textsc{Impact}}
\newcommand{\PA}{{\scriptscriptstyle{\mathsf{PA}}}}
\newcommand{\ADA}{{\scriptscriptstyle{\mathsf{ADA}}}}
\newcommand{\cube}[1]{\mathsf{c}({#1})}
\newcommand{\T}{\mathcal{T}}
\newcommand{\event}[1]{{#1}_\Sigma}
\newcommand{\data}[1]{{#1}_\Data}
\newcommand{\pathform}[1]{\Theta({#1})}
\newcommand{\accform}[1]{\Upsilon({#1})}
\newcommand{\accformA}[2]{\Upsilon_{#1}({#2})}
\newcommand{\substform}[1]{\overline{\Upsilon}({#1})}
\newcommand{\substformn}[1]{\overline{\Theta}({#1})}
\newcommand{\substformA}[2]{\overline{\Upsilon}_{#1}({#2})}        
\newcommand{\quantform}[1]{\widehat{\Upsilon}({#1})}
\newcommand{\quantformn}[1]{\widehat{\Theta}({#1})}
\newcommand{\nextf}[1]{{#1}^\prime}
\newcommand{\supsetforex}{\supseteq^{\scriptscriptstyle\forall\exists}}
\newcommand{\prefix}{\preceq}
\newcommand{\posforms}{\mathsf{Form}^+}
\newcommand{\cover}{\sqsubseteq}


%%%%%%%%%%%%%%%%%%%%%%%%%%%%%%%%%%%%%%%%%%%%%%%%%%%%%%%%%%%%%%%%%%%%%%%%%%%%%%%
\begin{document}
%%%%%%%%%%%%%%%%%%%%%%%%%%%%%%%%%%%%%%%%%%%%%%%%%%%%%%%%%%%%%%%%%%%%%%%%%%%%%%%

\title{First Order Alternation}

\author{Radu Iosif and Xiao Xu \\
CNRS, Verimag, Universit\'e de Grenoble Alpes \\
Email: \{Radu.Iosif,Xiao.Xu\}@univ-grenoble-alpes.fr}

\maketitle

\begin{abstract}
\end{abstract}

\section{Preliminaries}

For two integers $0 \leq i \leq j$, we denote by $[i,j]$ the set
$\set{i,i+1,\ldots,j}$ and by $[i]$ the set $[1,i]$, where $[0]$ is
the empty set.

A \emph{signature} $\sig = (\ssorts{\sig},\sfuns{\sig})$ consists of a
set $\ssorts{\sig}$ of \emph{sort symbols} and a set $\sfuns{\sig}$ of
sorted \emph{function symbols}. To simplify the presentation, we
assume w.l.o.g. that $\ssorts{\sig} = \set{\Data,\Bool}$\footnote{The
  generalization to more than two sorts is without difficulty, but
  would unnecessarily clutter the technical presentation.} and each
function symbol $f \in \sfuns{\sig}$ has $\#(f) \geq 0$ arguments of
sort $\Data$ and return value $f_\sigma \in \ssorts{\sig}$. If
$\#(f)=0$ then $f$ is a \emph{constant}. Let $\top$ and $\bot$ be
constants of sort $\Bool$. 

%% and $\mathbf{0}$ be a constant of sort $\Data$.

We consider an infinite countable set of \emph{variables} $\vars$,
where each $x \in \vars$ has an associated sort $x_\sigma$. A
\emph{term} $t$ of sort $t_\sigma = S$ is a variable $x \in \vars$
where $x_\sigma=S$, or $f(t_1,\ldots,t_{\#(f)})$ where
$t_1,\ldots,t_{\#(f)}$ are terms of sort $\Data$ and $f_\sigma=S$. An
\emph{atom} is a term $p(t_1,\ldots,t_{\#(p)})$ of sort $\Bool$, or an
equality $t \teq s$ between two terms of sort $\Data$. A
\emph{formula} is a possibly quantified boolean combination of atoms
and a \emph{sentence} is a formula in which each variable occurs under
the scope of a quantifier.

We denote by $\fv{\sigma}{\phi}$ the set of free variables of sort
$\sigma$ in $\phi$ and write $\fv{}{\phi}$ for $\bigcup_{\sigma \in
  \ssorts{\sig}} \fv{\sigma}{\phi}$. For a variable $x \in
\fv{}{\phi}$ and a term $t$ such that $t_\sigma = x_\sigma$, let
$\phi[t/x]$ be the result of replacing each occurrence of $x$ by
$t$. For indexed sets $T=\set{t_1,\ldots,t_n}$ and
$X=\set{x_1,\ldots,x_n}$, we write $\phi[T/X]$ for the formula
obtained by simultaneously replacing $x_i$ with $t_i$ in $\phi$, for
all $i\in[n]$. The size $\len{\phi}$ is the number of symbols occuring
in $\phi$.

An \emph{interpretation} $\I$ maps\begin{inparaenum}[(1)]
\item the sort $\Data$ into a non-empty set $\Data^\I$, 
%
\item the sort $\Bool$ into the set $\booli = \set{\true,\false}$, where
  $\top^\I = \true$, $\bot^\I = \false$, and
%
\item each function symbol $f$ into a total function $f^\I :
  (\Data^\I)^{\#(f)} \rightarrow f_\sigma^I$, or an element of
  $f_\sigma^I$ when $\#(f)=0$.
\end{inparaenum}
Given an interpretation $\I$, a \emph{valuation} $\nu$ maps each
variable $x \in \vars$ into an element of $x_\sigma^\I$. For a term
$t$, we denote by $t^\I_\nu$ the value obtained by replacing each
function symbol $f$ by its interpretation $f^\I$ and each variable $x$
by its valuation $\nu(x)$. For a formula $\phi$, we write $\I,\nu
\models \phi$ if the formula obtained by replacing each term $t$ in
$\phi$ by the value $t^\I_\nu$ is logically equivalent to true.  The
semantics of quantifiers is given as usual: $\I,\nu \models \exists x
~.~ \phi$ if and only if $\I,\nu[x \leftarrow u] \models \phi$, for
some value $u \in x_\sigma^\I$, where $\nu[x \leftarrow u]$ is the
valuation that assigns $u$ to $x$ and $\nu(y)$ to all $y \neq x$.

A formula $\phi$ is \emph{satisfiable} in the interpretation $\I$ if
there exists a valuation $\nu$ such that $\I,\nu \models \phi$, and
\emph{valid} if $\I,\nu \models \phi$ for all valuations $\nu$.  The
\emph{first-order theory} $\theory(\sig,\I)$ is the set of valid
formulae written in the signature $\sig$, with the interpretation
$\I$. A \emph{decision procedure} for $\theory(\sig,\I)$ is an
algorithm that takes a formula $\phi$ in the signature $\sig$ and
returns yes if and only if $\phi \in \theory(\sig,\I)$.

Given formulae $\varphi$ and $\psi$, we say that \emph{$\phi$ entails
  $\psi$ modulo $\theory(\sig,\I)$}, denoted $\phi
\models_{\theory(\sig,\I)} \psi$ iff $\varphi \rightarrow \psi \in
\theory(\sig,\I)$. We say that $\varphi$ and $\psi$ are
\emph{$\theory(\sig,\I)$-equivalent}, denoted $\phi \iffeq_{\theory(\sig,\I)} \psi$ iff
$\phi \models_{\theory(\sig,\I)} \psi$ and $\psi
\models_{\theory(\sig,\I)} \phi$. The theory $\theory(\sig,\I)$ has
\emph{quantifier elimination} iff for any formula $\exists x ~.~
\phi(x)$, where $\phi$ is quantifier-free, there exists a quantifier
free formula $\psi$ such that $\phi$ and $\psi$ are
$\theory(\sig,\I)$-equivalent.


\section{First Order Alternating Data Automata}

We fix a theory $\T = \theory(\sig,\I)$ and a finite alphabet $\Sigma$
of \emph{input events}. Given a finite set $X \subset \vars$ of
variables of sort $\Data$, let $X \mapsto \Data^\I$ be the set of
valuations of the variables $X$ and $\Sigma[X] = \Sigma \times (X
\mapsto \Data^\I)$ be the possibly infinite set of \emph{data
  symbols}. A \emph{data word} (word in the sequel) is a finite
sequence $(a_1,\nu_1)(a_2,\nu_2) \ldots (a_n,\nu_n)$ of data symbols,
where $a_1,\ldots,a_n \in \Sigma$ and $\nu_1,\ldots,\nu_n : X
\rightarrow \Data^\I$ are valuations. We denote by $\varepsilon$ the
empty sequence, by $\Sigma^*$ the set of finite sequences of input
events and by $\Sigma[X]^*$ the set of data words over the variables
$X$.

Given sets of variables $B,X,Y \subset \vars$ of sorts $\Bool$,
$\Data$ and $\Data$, respectively, we denote by $\Form(B;X;Y)$ the set
of formulae $\phi$ such that $\fv{\Bool}{\phi} = B$, $\fv{\Data}{\phi}
= X$ and each variable from $Y$ occurs under the scope of a
quantifier. By $\Form^+(B;X;Y)$ we denote the subset of $\Form(B;X;Y)$
in which each boolean variable from $B$ occurs under an even number of
negations. We denote by $\phi(B;X;Y)$ any formula in $\Form(B;X;Y)$.

A \emph{first-order alternating data automaton} (automaton in the
sequel) is a tuple $\A = \tuple{G,L,Q,\iota,F,\Delta}$,
where: \begin{compactitem}
%
\item $G \subseteq \vars$ is a finite set of \emph{global
  variables} of sort $\Data$, 
%
\item $L \subseteq \vars$ is a finite set of \emph{local
  variables} of sort $\Data$,
%
\item $Q \subset \vars$ is a finite set of variables of sort $\Bool$
  (\emph{states}),
%
\item $\iota \in \Form^+(Q;\emptyset;\emptyset)$ is the \emph{initial
  configuration},
%
\item $\Delta : Q \times \Sigma \rightarrow
  \Form^+(Q;G\cup\overline{G}\cup\overline{L};L)$ is a
  \emph{transition function},
\end{compactitem}
where, for a set $X \subseteq \vars$, we denote $\overline{X} =
\set{\overline{x} \mid x \in X}$. Intuitively, the variables
$\overline{x}$ from a transition formula $\Delta(q,a)$ refer to past
values, whereas the other variables refer to the current
values. Observe that the initial values of the global variables are
unconstrained (since the initial configuration is a boolean
combination of states without variables) and quantification occurs
only over the current copy of the local variables $L$. As a remark,
the transition relation does not quantify over global variables $G
\cup \overline{G}$, which must be visible throughout the
execution. Furthermore, we assume that the past values of the local
variables are coined and beyond our control, thus quantification over
$\overline{L}$ is not allowed either. Last, observe that $\Delta$ is a
total function, defined\footnote{A partial transition relation $R$ can
  be encoded by setting $\Delta(q,a) = \bot$ when $R(q,a)$ is
  undefined.} for all pairs $(q,a) \in Q \times \Sigma$. The
\emph{size} of $\A$ is defined as $\len{\A} = \len{\iota} +
\sum_{(q,a) \in Q\times\Sigma}\len{\Delta(q,a)}$.

In order to define the semantics of automata, we consider the
\emph{time-stamped} sets of variables $\stamp{X}{n} = \{\stamp{x}{n}
\mid x \in X\}$ for any $n\in\zed$. For an input event $a \in \Sigma$,
a positive integer $n\geq0$ and a formula $\varphi\in\Form^+(Q;G;L)$,
we write $\stamp{\Delta}{n}(\varphi,a)$ for the formula obtained from
$\varphi$ by simultaneously replacing each state $q \in
\fv{\Bool}{\phi}$ with the formula
$\Delta(q,a)[\stamp{G}{n}/\overline{G}, \stamp{G}{n+1}/G,
  \stamp{L}{n}/\overline{L}, \stamp{L}{n+1}/L]$. For a sequence $a_1
\ldots a_n \in \Sigma^*$, we consider the formulae \(\iota \equiv
\phi_0, \phi_1, \ldots, \phi_n \), where $\phi_{i} \equiv
\stamp{\Delta}{i-1}(\phi_{i-1},a_{i})$, for all $i \in [n]$. $\A$
\emph{accepts} a word $w = (a_1,\nu_1) \ldots (a_n,\nu_n) \in
\Sigma[G]^*$ iff there exists a valuation $\nu$ such that
$\I,\nu \models \phi_n$ and: \begin{compactitem}
%% \item $\nu(\stamp{x}{0}) = \mathbf{0}^\I$ for each variable $x \in G
%%   \cup L$, i.e.\ all data variables are initially $\mathbf{0}$,
%
\item $\nu(\stamp{x}{i}) = \nu_i(x)$ for each global variable $x\in G$
  and each time-stamp $i \in [n]$,
%
\item $\nu(q) = \true$ if and only if $q \in F$. 
\end{compactitem}
The \emph{language} of $\A$ is the set $L(\A)$ of words from
$\Sigma[G]^*$ accepted by $\A$. In this paper, we ask the following
questions: \begin{compactenum}
\item \emph{boolean closure}: given automata $\A_i =
  \tuple{G,L_i,Q_i,\iota_i,F_i,\Delta_i}$, for $i=1,2$, with $L_1 \cap
  L_2 = \emptyset$ and $Q_1 \cap Q_2 = \emptyset$, do there exist
  automata $\A_\cap$, $\A_\cup$ and $\overline{\A}_1$ such that
  $L(\A_\cap) = L(\A_1) \cap L(\A_2)$, $L(\A_\cup) = L(\A_1) \cup
  L(\A_2)$ and $L(\overline{\A}_1) = \Sigma[G]^* \setminus L(\A_1)$ ?
%
\item \emph{emptiness}: given an automaton $\A$, is $L(\A) =
  \emptyset$ ?
\end{compactenum}

\subsection{Boolean Closure} 

Given a set $B$ of boolean variables, two disjoint sets $X$ and $Y$ of
data variables, for any formula $\phi \in \Form^+(B;X;Y)$,
define $\overline{\phi} \in \Form^+(B;X;Y)$ recursively:
\[\begin{array}{rclcrclcrcl}
\overline{\phi_1 \vee \phi_2} & \equiv & \overline{\phi_1} \wedge \overline{\phi_2} & \hspace*{5mm} & 
\overline{\phi_1 \wedge \phi_2} & \equiv & \overline{\phi_1} \vee \overline{\phi_2} & \hspace*{5mm} & 
\overline{\phi} & \equiv & \neg\phi \text{ if $\phi \not\in B$ is an atom} \\ 
\overline{\exists x ~.~ \phi_1} & \equiv & \forall x ~.~ \overline{\phi_1} && 
\overline{\forall x ~.~ \phi_1} & \equiv & \exists x ~.~ \overline{\phi_1} && 
\overline{\neg\phi} & \equiv & \neg \overline{\phi} \text{ if $\phi$ is not an atom} \\
&& && && && \overline{\phi} & \equiv & \phi \text{ if $\phi \in B$}
\end{array}\]
Clearly $\phi$ and $\overline{\phi}$ have the same size. The relation
between $\phi$ and $\overline{\phi}$ is formalized below:

\begin{proposition}\label{prop:complement}
  Given sets of variables $B,X,Y \subset \vars$ of sorts boolean, data
  and data respectively, and a formula $\phi \in
  \Form^+(B;X;Y)$, for any valuation $\nu$ we have $\nu \models
  \phi$ if and only if $\overline{\nu} \models \neg\overline{\phi}$,
  where $\overline{\nu}(x) = \nu(x)$ for all $x \in X$ and
  $\overline{\nu}(b) = \neg\nu(b)$ for all $b \in B$.
\end{proposition}
\proof{By induction on the structure of $\phi$. \qed}

Consider the automata $\A_i = \tuple{G,L_i,Q_i,\iota_i,F_i,\Delta_i}$,
for $i=1,2$, where $L_1 \cap L_2 = \emptyset$ and $Q_1 \cap Q_2 =
\emptyset$. We define the following automata: 
\[\begin{array}{rcl}
\A_\cup & = & \tuple{G,~ L_1 \cup L_2,~ Q_1 \cup Q_2,~ \iota_1 \vee \iota_2,~ F_1 \cup F_2,~ \Delta_1 \cup \Delta_2} \\
\A_\cap & = & \tuple{G,~ L_1 \cup L_2,~ Q_1 \cup Q_2,~ \iota_1 \wedge \iota_2,~ F_1 \cup F_2,~ \Delta_1 \cup \Delta_2} \\
\overline{A_1} & = & \tuple{G,~ L_1,~ Q_1,~ \overline{\iota_1},~ Q_1 \setminus F_1,~ \lambda q \in Q_1~ \lambda a \in \Sigma ~.~ \overline{\Delta_1(q,a)}}
\end{array}\]

\begin{lemma}\label{lemma:boolean-closure}
  Given automata $\A_i = \tuple{G,L_i,Q_i,\iota_i,F_i,\Delta_i}$, for
  $i=1,2$, where $L_1 \cap L_2 = \emptyset$ and $Q_1 \cap Q_2 =
  \emptyset$, we have $L(\A_\cup) = L(\A_1) \cup L(\A_2)$, $L(A_\cap)
  = L(\A_1) \cap L(\A_2)$ and $L(\overline{\A_1}) = \Sigma[G]*
  \setminus L(\A_1)$. 
\end{lemma}
\proof{
  Same as \cite[Lemma 1]{IosifX17}. 
\qed}

\section{Checking Emptiness with Interpolants and Skolem Witnesses}

Since the emptiness problem is undecidable even for automata with only
two global variables and no local variables, by reduction from the
emptiness problem for 2-counter machines \cite{Minsky67}, we give a
semi-algorithm with the following properties. Given an automaton $\A =
\tuple{G,L,Q,\iota,F,\Delta}$ as input, the procedure
will\begin{inparaenum}[(1)]
\item return a word $w \in L(\A)$ if $L(\A) \neq \emptyset$, and
%
\item if the procedure terminates without returning such a word, then
  we can safely conclude that $L(\A) = \emptyset$. \end{inparaenum}
Due to the undecidability of the emptiness problem, there exist
automata with empty language on which the procedure runs forever.

The semi-algorithm described in this section reduces the language
emptiness problem for $\A$ to a (possibly infinite) sequence of
satisfiability problems. We generate these formulae systematically
using a tree whose backbone is a finite set of input events and whose
labels are quantifier-free formulae. Given two input sequences
$\alpha,\beta \in \Sigma^*$, we say that $\beta$ is a prefix of
$\alpha$, written $\beta \preceq \alpha$ if $\alpha = \beta \gamma$,
for some sequence $\gamma \in \Sigma^*$. A set $S \subseteq \Sigma^*$
is \emph{prefix-closed} if for each $\alpha \in S$, $\beta \preceq
\alpha$ implies $\beta \in S$ and \emph{complete} if for each $\alpha
\in S$ there exists $a \in \Sigma$ such that $\alpha a \in S$ if and
only if $\alpha b \in S$ for all $b \in \Sigma$. 

For a set of variables $X \subseteq \vars$, let $\stamp{X}{-\infty} =
\bigcup_{n\leq0} \stamp{X}{n}$ be a set of \emph{history} variables,
where $\stamp{X}{0}$ track the current values, $\stamp{X}{-1}$ the
immediate predecessors, etc. For a formula $\varphi$ and an integer $n
\in \zed$, we denote by $\stamp{\varphi}{n}$ the formula in which each
time-stamped free variable $\stamp{x}{i}$ is replaced with
$\stamp{x}{i+n}$. Given formulae $\varphi,\psi \in
\Form^+(Q;\stamp{X}{-\infty}; \emptyset)$ and a sequence $a_1 \ldots
a_n \in \Sigma^*$, we write $\varphi \darrow{a_1 \ldots a_n}{\A} \psi$
if there exist formulae $\phi_0, \ldots, \phi_n$ such
that\begin{inparaenum}[(i)]
\item $\varphi \equiv \phi_0$, 
\item $\phi_i \equiv \stamp{\Delta}{i-1}(\phi_{i-1},a_i)$, for all $i
  \in [n]$, and
\item $\psi \equiv \stamp{\phi_n}{-n}$. \end{inparaenum} Observe that
$\psi$ is uniquely determined by $\varphi$, because the result of the
substitutions $\stamp{\Delta}{i-1}(\phi_{i-1},a_i)$ and
$\stamp{\phi_n}{-n}$ is unique.

\begin{proposition}\label{prop:monotonicity}
  For all $\varphi,\psi \in \Form^+(Q;\stamp{X}{-\infty}; \emptyset)$
  and $w \in \Sigma^*$, if $\varphi \models_\T \psi$ and $\varphi',
  \psi'$ are the unique formulae such that $\varphi \darrow{w}{\A}
  \varphi'$ and $\psi \darrow{w}{\A} \psi'$, then $\varphi' \models_\T
  \psi'$.
\end{proposition}
\proof{By induction on the length of $w = a_1 \ldots a_n$, using the
  fact that any state $q \in Q$ occurs positively in each formula from
  the sequences $\varphi \equiv \varphi_0, \ldots, \varphi_n$ and
  $\psi \equiv \psi_0, \ldots, \psi_n$, defining $\varphi
  \darrow{w}{\A} \varphi' \equiv \stamp{\psi_n}{-n}$ and $\psi
  \darrow{w}{\A} \stamp{\psi_n}{-n} \equiv \psi'$, respectively. \qed}

\begin{definition}\label{def:unfolding}
An \emph{unfolding} of $\A$ is a finite mapping $U : \Sigma^*
\rightharpoonup_{\mathit{fin}} \Form^+(Q;\stamp{(G\cup L)}{-\infty};
\emptyset)$ such that $\dom(U)$ is a prefix-closed complete set,
$U(\varepsilon) = \iota$ and, for all $\alpha a \in \dom(U)$, we have
$\stamp{\Delta}{0}(U(\alpha),a) \models_\T \stamp{U(\alpha a)}{1}$.
Moreover, $U$ is \emph{safe} if for all $\alpha \in \dom(U)$, the
formula $U(\alpha) \wedge \mathit{Final}$ is unsatisfiable, where
$\mathit{Final} \equiv \bigwedge_{q \not\in F} q \rightarrow \bot$.
%% $\mathit{Init} \equiv \bigwedge_{x \in \stamp{G}{0} \cup
%% \stamp{L}{0}} x \teq \mathbf{0}$ and
\end{definition}

The elements of $\dom(U)$ are called \emph{nodes} and the
prefix-maximal nodes \emph{leaves}. A node $\alpha \in \dom(U)$ is
\emph{covered} by another node $\beta \in \dom(U)$ if there exists a
prefix $\alpha' \preceq \alpha$ such that $U(\alpha') \models_\T
U(\beta)$. A node is \emph{uncovered} if it is not covered by any
other node. We say that $U$ is \emph{closed} when each leaf in
$\dom(U)$ is covered by an uncovered node.

\begin{lemma}\label{lemma:art}
  If $\A$ has a nonempty safe closed unfolding, then $L(\A) =
  \emptyset$.
\end{lemma}
\proof{ Let $U$ be a safe closed unfolding of $\A$ and suppose, for a
  contradiction, that $w = (a_1,\nu_1) \ldots (a_n,\nu_n) \in
  L(\A)$. Let $\alpha = a_1 \ldots a_n \in \Sigma^*$ and
  $\phi_{\alpha}$ be the unique formula such that $\iota
  \darrow{\alpha}{\A} \phi_{\alpha}$. Since $w \in L(\A)$, we have
  that $\phi_{\alpha} \wedge \mathit{Final}$ is satisfiable. It is
  sufficient to show that there exists a node $\beta \in \dom(U)$ such
  that $\phi_{\alpha} \models_\T U(\beta)$, because then $U(\beta)
  \wedge \mathit{Final}$ is satisfiable as well and $U$ is not safe,
  contradiction.

  If $\alpha \in \dom(U)$ then we obtain $\phi_{\alpha} \models_\T
  U(\alpha)$ from the fact that $U$ is an unfolding, applying
  Proposition \ref{prop:monotonicity} inductively on the length of
  $\alpha$. Otherwise, let $\alpha' \in \dom(U)$ be the unique leaf
  which is a prefix of $\alpha = \alpha'\gamma$, for a unique nonempty
  sequence $\gamma$. Such a leaf must exist because $\dom(U) \neq
  \emptyset$. Moreover, since $U$ is closed, $\alpha'$ must be covered
  and let $\alpha_1$ be the prefix of $\alpha'$ such that $U(\alpha_1)
  \models_\T U(\beta_1)$, for some uncovered node $\beta_1$. Let
  $\phi_1$ be the unique formula such that $\iota
  \darrow{\alpha_1}{\A} \phi_1$. Applying Proposition
  \ref{prop:monotonicity} inductively, we obtain that $\phi_1
  \models_\T U(\alpha_1)$, hence $\phi_1 \models_\T U(\beta_1)$.
  Because $\beta_1$ is uncovered, it cannot be a leaf and let
  $\gamma_1$ be the unique prefix of $\gamma$ such that
  $\beta_1\gamma_1$ is a leaf. Since $U$ is closed, $\beta_1\gamma_1$
  is covered and, because $\beta_1$ is uncovered, there exists a
  nonempty sequence $\gamma'_1 \preceq \gamma_1$ such that
  $U(\beta_1\gamma'_1) \models_\T U(\beta_2)$ for some uncovered node
  $\beta_2 \in \dom(U)$. Let $\alpha_2 = \alpha_1\gamma'_1$ and
  $\phi_2$ be the unique formula such that $\iota
  \darrow{\alpha_2}{\A} \phi_2$. Since $\phi_1 \models_\T U(\beta_1)$
  and $\phi_1 \darrow{\gamma'_1}{\A} \phi_2$, we obtain that $\phi_2
  \models_\T U(\beta_2)$ by an inductive application of Proposition
  \ref{prop:monotonicity}. Continuing this argument inductively, we
  discover a strictly increasing sequence $\alpha_1 \prec \alpha_2
  \prec \ldots$ of prefixes of $\alpha$, a sequence of formulae $\iota
  \darrow{\alpha_1}{\A} \phi_1, \iota \darrow{\alpha_2}{\A} \phi_2,
  \ldots$ and a sequence of nodes $\beta_1, \beta_2, \ldots \in
  \dom(U)$ such that $\phi_i \models_\T U(\beta_i)$, for all $i =
  1,2,\ldots$. Since $\alpha$ is finite, it must be the case that
  $\phi_\alpha \models_\T U(\beta_n)$, for some node $\beta_n$ in the
  sequence. \qed}

Lazy annotation semi-algorithms \cite{McMillan06,McMillan14}
essentially try to prove language emptiness by building safe closed
unfoldings of automata. The labels of these unfoldings are
interpolants obtained from the proof of unsatisfiability of the
formulae that encode the acceptance conditions for the nodes
(sequences of input events) in the unfolding. Given an unsatisfiable
conjunction of formulae $\psi_1 \wedge \ldots \wedge \psi_n$, an
\emph{interpolant} is a tuple of formulae
$\tuple{I_1,\ldots,I_{n-1},I_n}$ such that $I_n \equiv \bot$, $I_i
\wedge \psi_i \models_\T I_{i+1}$ and $I_i$ contains only variables
and function symbols that are common to $\psi_i$ and $\psi_{i+1}$, for
all $i \in [n-1]$. Moreover, by Lyndon's Interpolation Theorem
\cite{Lyndon59}, we can assume without loss of generality that every
boolean variable with at least one positive (negative) occurrence in
$I_i$ has at least one positive (negative) occurrence in both $\psi_i$
and $\psi_{i+1}$.

The execution semantics of automata is defined in terms of successive
rewritings of the boolean variables (states) by time-stamped
transition formulae. Formally, given a sequence $a_1 \ldots a_n \in
\Sigma^*$ of input events, we have a sequence of formulae $\Phi_1,
\ldots, \Phi_n$, where $\iota \darrow{a_1 \ldots a_i}{\A} \Phi_i$, for
all $i \in [n]$. Altough natural for defining acceptance of
alternating automata, this semantics does not cope well with
interpolation, because\begin{inparaenum}[(1)]
\item rewriting states loses track of the history of the computation,
  and
%
\item quantifiers may occur nested in each of the intermediate
  formulae $\Phi_1,\ldots,\Phi_n$.
\end{inparaenum}

More precisely, each formula in the sequence defining acceptance is of
the form: \[\Phi_i \equiv \mathbf{Q}_1 \stamp{\mathcal{L}}{1}_1 \ldots
\mathbf{Q}_i \stamp{\mathcal{L}}{i}_i ~.~ \psi_i(Q,\bigcup_{j=0}^i
\stamp{G}{j} \cup \stamp{\mathcal{L}}{j}_j)\] where $\mathcal{L}_1,
\ldots, \mathcal{L}_i$ are sequences of local variables from $L$ and
$\mathbf{Q} \mathcal{X}$ denotes the quantifier prefix $\mathcal{Q}_1
x_1 \ldots \mathcal{Q}_k x_k$, with $\mathbf{Q} = \mathcal{Q}_1 \ldots
\mathcal{Q}_k$ and $\mathcal{X} = x_1,\ldots,x_k$. We deal with the
first issue above by explicitly encoding the computation history as a
possibly quantified conjunction of implications of the form
$\stamp{q}{i} \rightarrow \stamp{\Delta}{i}(q,a_i)[\stamp{q}{i+1}/q]$,
for $i \in [0,n-1]$. To this end, we use the following fact:

\begin{proposition}\label{prop:substitution}
  Given a formula $\phi$ in which $q \in \fv{\Bool}{\phi}$ occurs
  positively, for any formula $\psi$ and any valuation $\nu$ of
  $\fv{}{\phi} \cup \fv{}{\psi}$, \(\nu \models \phi[\psi/q]\) iff
  \(\nu \models \exists q ~.~ \phi \wedge (q \rightarrow \psi)\).
\end{proposition}
\proof{ ``$\Leftarrow$'' Suppose that $\nu \models \phi[\psi/q]$ and
  distinguish the following cases: \begin{compactenum}
  \item if $\nu \models \psi$, since $q$ occurs positively in $\phi$, we
    obtain that $\nu \models \phi[\top/q]$, thus $\nu \models
    \phi[\top/q] \wedge (\top \rightarrow \psi)$ and $\nu \models
    \exists q ~.~ \phi \wedge (q \rightarrow \psi)$ follows.
    %
  \item if $\nu \not\models \psi$, since $q$ occurs positively in $\phi$, we
    obtain that $\nu \models \phi[\bot/q]$, thus $\nu \models
    \phi[\bot/q] \wedge (\bot \rightarrow \psi)$ and $\nu \models
    \exists q ~.~ \phi \wedge (q \rightarrow \psi)$ follows.
  \end{compactenum}

  ``$\Rightarrow$'' Suppose that $\nu \models \exists q ~.~ \phi
  \wedge (q \rightarrow \psi)$, or equivalently, $\nu \models
  (\phi[\top/q] \wedge \psi) \vee \phi[\bot/q]$. If $\nu \models
  \phi[\top/q] \wedge \psi$ then $\nu \models
  \phi[\psi/q]$. Otherwise, if $\nu \models \phi[\bot/q]$, we
  distinguish two cases: \begin{compactitem} 
  \item if $\nu \not\models \psi$ then clearly $\nu \models
    \phi[\psi/q]$.
    %
  \item else, if $\nu \models \psi$, since $q$ occurs positively in
    $\phi$ and $\nu \models \phi[\bot/q]$ then also $\nu \models
    \phi[\top/q]$, because $\phi$ is monotone in $q$ in the entailment
    partial order. But then $\nu \models \phi[\psi/q]$. \qed
  \end{compactitem}}
Next, we consider the following formulae, logically equivalent to
$\Phi_i$, for all $i \in [n]$:
\[\begin{array}{rcl} 
\Theta_i & \equiv & \mathbf{\exists} \stamp{Q}{0} \mathbf{Q}_1
\stamp{\mathcal{L}}{1}_1 \mathbf{\exists} \stamp{Q}{1} \mathbf{Q}_2
\stamp{\mathcal{L}}{2}_2 \mathbf{\exists} \stamp{Q}{2} \ldots
\mathbf{Q}_i \stamp{\mathcal{L}}{i}_i \mathbf{\exists} \stamp{Q}{i} 
~.~ (\top \rightarrow \stamp{\iota}{0}) ~\wedge~ \bigwedge_{j=0}^i \Psi_j \\
&& \text{where } \Psi_j \equiv \bigwedge_{q \in Q} \left(\stamp{q}{j} \rightarrow
\stamp{\Delta}{j}(q,a_j)[\stamp{q}{j+1}/q]\right)\enspace.
\end{array}\] These formulae are
still not proper interpolation problems, due to the nested quantifier
prefix. Thus we use Skolem witnesses to reduce the (un)satisfiability
of the above formulae to the problem of computing an interpolant for a
quantifier-free formula.

\begin{theorem}\label{thm:skolem}
  Given $Q_1 x_1 \ldots Q_n x_n ~.~ \varphi$ a first order sentence, 
  where $Q_1, \ldots, Q_n \in \set{\exists,\forall}$ and $\varphi$ is
  quantifier-free, let: 
  \[\eta_i \equiv \left\{\begin{array}{ll}
  f_{x_i}(y_1,\ldots,y_{k_i}) & \text{if } Q_i \equiv \forall \\
  x_i & \text{if } Q_i \equiv \exists
  \end{array}\right.\]
  where $f_{x_i}$ is a fresh function symbol and $\set{y_1, \ldots,
    y_{k_i}} = \set{x_j \mid j < i,~ Q_j \equiv \exists}$. Then we
  have \(Q_1 x_1 \ldots Q_n x_n ~.~ \varphi \models_\T
  \varphi[\eta_1/x_1,\ldots,\eta_n/x_n]\).
\end{theorem}
\proof{See \cite[Theorem 2.1.8]{BorgerGraedelGurevich97} and
  \cite[Lemma 2.1.9]{BorgerGraedelGurevich97}. \qed} We further
transform $\Theta_i$ into a quantifier-free formula
$\widehat{\Theta}_i \equiv (\top \rightarrow \stamp{\iota}{0}) \wedge
\bigwedge_{j=0}^i \widehat{\Psi}_j$, where $\widehat{\Psi}_j$ are
obtained from $\Psi_j$ by replacing each universally quantified
variable $x_i$ by a term $f_{x_i}(y_1, \ldots, y_{k_i})$, whose free
variables are among the existentially quantified variables $x_j$ that
precede $x_i$ in the quantifier prefix. By Theorem \ref{thm:skolem},
replacing a universally quantified variable by an uninterpreted Skolem
function symbol yields an overapproximation of the original formula,
thus proving unsatisfiability of $\widehat{\Theta}_i$ suffices to
prove unsatisfiability of $\Theta_i$.

Furthermore, a proof of unsatisfiability for $\widehat{\Theta}_i$
yields an interpolant $\tuple{I_0, \ldots, I_i}$, where $\fv{}{I_j}
\subseteq \bigcup_{\ell=0}^j \stamp{G}{\ell} \cup \stamp{L}{\ell}$,
for all $j \in [0,i]$. That is, the interpolants obtained from the
proof of unsatisfiability of $\widehat{\Theta}_i$ may refer to all
values taken by the global and local variables from the beginning of
the path up to the $i$-th step.



%%%%%%%%%%%%%%%%%%%%%%%%%%%%%%%%%%%%%%%%%%%%%%%%%%%%%%%%%%%%%%%%%%%%%%%%%%%%%%%
\bibliographystyle{abbrv} \bibliography{refs}
%%%%%%%%%%%%%%%%%%%%%%%%%%%%%%%%%%%%%%%%%%%%%%%%%%%%%%%%%%%%%%%%%%%%%%%%%%%%%%%

%%%%%%%%%%%%%%%%%%%%%%%%%%%%%%%%%%%%%%%%%%%%%%%%%%%%%%%%%%%%%%%%%%%%%%%%%%%%%%%
\end{document}
%%%%%%%%%%%%%%%%%%%%%%%%%%%%%%%%%%%%%%%%%%%%%%%%%%%%%%%%%%%%%%%%%%%%%%%%%%%%%%%
