\documentclass{llncs}
% \documentclass[10pt,conference,letterpaper,twocolumn]{IEEEtran}

\usepackage[usenames,dvipsnames]{color}

\usepackage{latexsym}
\usepackage{amsxtra} 
\usepackage{amssymb}
\usepackage{amsmath}
\usepackage{pslatex}
\usepackage{epsfig}
\usepackage{wrapfig}
\usepackage{paralist}
\usepackage{stmaryrd}
\usepackage{txfonts}
\usepackage{framed}
\usepackage{makecell}
\usepackage{url}
\usepackage{tikz}
\usetikzlibrary{automata,positioning, calc}
\usepackage[inline,shortlabels]{enumitem}

\usepackage{proof}

\usepackage{algorithm}
\usepackage{algorithmicx}
\usepackage[noend]{algpseudocode}

\usepackage[draft]{commenting}

\pagestyle{plain}

%\newtheorem{definition}{Definition}
%\newtheorem{lemma}{Lemma}
%\newtheorem{theorem}{Theorem}
%\newtheorem{proposition}{Proposition}
%\newtheorem{corollary}{Corollary}
%\newtheorem{conjecture}{Conjecture}
%\newtheorem{claim}{Claim}
%\newtheorem{property}{Property}
\newtheorem{fact}{Fact}

\newcommand*\circled[1]{\tikz[baseline=(char.base)]{
            \node[shape=circle,draw,inner sep=0.5pt] (char) {#1};}}

% Marius commands
\newcommand{\Z}{{\mathbb Z}}
\newcommand{\N}{{\mathbb N}}
\newcommand{\relmat}[1] { \{\!\!\{#1\}\!\!\}}
\newcommand{\matrel}[1] { [\!\![#1]\!\!]}
\newcommand{\key}[1]{\underline{#1}}

% complexity 
\newcommand{\np}{$\mathsf{NP}$}
\newcommand{\pspace}{$\mathsf{PSPACE}$~}
\newcommand{\sigmatwop}{$\mathsf{\Sigma}_2^p$}
\newcommand{\pitwop}{$\mathsf{\Pi}_2^p$}

% semantic stuff
\newcommand{\bi}[1]{{}^\omega\!{#1}^\omega}
\newcommand{\rbr}{{\bf ]\!]}}
\newcommand{\lbr}{{\bf [\![}}
\newcommand{\sem}[1]{\lbr #1 \rbr}
\newcommand{\model}[2]{\sem{#1}_{#2}}
\newcommand{\lfp}[1]{\mbox{fix}\ {#1}}
\newcommand{\wpr}{\widetilde{pre}}
\newcommand{\cceil}[1]{\lfloor #1 \rfloor + 1}
\newcommand{\set}[1]{\{ #1 \}}
\newcommand{\tuple}[1]{\langle #1 \rangle}
\renewcommand{\vec}[1]{\mathbf #1}
\newcommand{\tup}[1]{\overline #1}
\newcommand{\ite}[3]{\mathsf{ite}(#1,#2,#3)}
\newcommand{\tupite}[2]{\mathsf{ite}(#1,#2)}
\newcommand{\minsem}[1]{\sem{#1}^\mu}

% alias
\newcommand{\alias}{\Diamond}
\newcommand{\may}[2]{{\langle #1 \rangle}{#2}}
\newcommand{\must}[2]{{\lbrack #1 \rbrack}{#2}}
\newcommand{\defd}[1]{\delta({#1})}

% definitions
\newcommand{\isdef}{\stackrel{\scriptscriptstyle{\mathsf{def}}}{=}}
\renewcommand{\iff}{\Leftrightarrow}

% algorithms
\newcommand{\sig}{\mathsf{S}}
\newcommand{\stamp}[2]{{#1}^{\scriptscriptstyle{({#2})}}}
\newcommand{\overstamp}[2]{{#1}^{\scriptscriptstyle{\langle{#2}\rangle}}}

% recursive structures
\newcommand{\LS}{\mathtt{LS}}
\newcommand{\DLL}{\mathtt{DLL}}
\newcommand{\TLL}{\mathtt{TLL}}
\newcommand{\TREE}{\mathtt{TREE}}
\newcommand{\ax}{\mathtt{a}}
\newcommand{\bx}{\mathtt{b}}
\newcommand{\cx}{\mathtt{c}}
\newcommand{\dx}{\mathtt{d}}

% cardinal, degree, ...
\newcommand{\sgn}{\mbox{sgn}}
%\newcommand{\gcd}{\mbox{gcd}}
\newcommand{\lcm}{\mbox{lcm}}
\newcommand{\len}[1]{{|{#1}|}}
\newcommand{\card}[1]{{|\!|{#1}|\!|}}
\newcommand{\minimum}[2]{\mbox{min}({#1},{#2})}
\newcommand{\maximum}[2]{\mbox{max}({#1},{#2})}
\newcommand{\minset}[2]{\mbox{min}\{{#1}\ |\ {#2}\}}
\newcommand{\maxset}[2]{\mbox{max}\{{#1}\ |\ {#2}\}}
\newcommand{\divs}[1]{\mbox{div}({#1})}

% automata and languages
\newcommand{\auth}[1]{{\mathcal A}_{#1}}
\newcommand{\lang}[1]{{\mathcal L}({#1})}
\newcommand{\runs}[1]{{\mathcal R}({#1})}
\newcommand{\val}[1]{{\mathcal V}({#1})}
\newcommand{\trace}[1]{{Tr}({#1})}
\newcommand{\state}[1]{{\Sigma}({#1})}
\newcommand{\trans}[1]{{\Theta}({#1})}
\newcommand{\access}[3]{{\mathcal L}_{{#1},{#2}}({#3})}
\newcommand{\eats}{\rightarrow}
\newcommand{\Eats}[1]{\stackrel{#1}{\longrightarrow}}
\newcommand{\arrow}[2]{\xrightarrow{{\scriptscriptstyle #1}}_{{\scriptstyle #2}}}
\newcommand{\larrow}[2]{\xleftarrow[{\scriptscriptstyle #2}]{{\scriptstyle #1}}} 
\newcommand{\darrow}[2]{\stackrel{\scriptscriptstyle #1}{\Longrightarrow}_{\!\!\scriptscriptstyle #2}}
\newcommand{\harrow}[2]{\stackrel{\scriptscriptstyle #1}{\leadsto}_{\!\!\scriptscriptstyle #2}}

% numbers
\newcommand{\nat}{{\bf \mathbb{N}}}
\newcommand{\zed}{{\bf \mathbb{Z}}}
\newcommand{\rat}{{\bf \mathbb{Q}}}
\newcommand{\real}{{\bf \mathbb{R}}}
\newcommand{\linzed}{{\bf \mathcal{L}\mathbb{Z}}}

% powerset
\newcommand{\pow}[1]{{\mathcal P}(#1)}
\newcommand{\pown}[2]{{\mathcal P}^{#1}(#2)}

% by definition
\newcommand{\bydef}{\stackrel{\Delta}{=}}

% projection, lifting
\newcommand{\proj}[2]{{#1}\!\!\downarrow_{{#2}}}
\newcommand{\lift}[2]{{#1}\!\!\uparrow_{{#2}}}

% forward, backward
\newcommand{\forw}[1]{\stackrel{\rightarrow}{#1}}
\newcommand{\back}[1]{\stackrel{\leftarrow}{#1}}

% equation figures
\newcommand{\eqnfigx}[1]{
\begin{minipage}{3in}
\begin{center}
\begin{eqnarray*}#1
\end{eqnarray*}
\end{center}
\end{minipage}}

\newcommand{\twoeqnfigx}[2]{
\eqnfigx{#1}
\eqnfigx{#2}
}

\newcommand{\threeeqnfigx}[3]{
\eqnfigx{#1}\\
\eqnfigx{#2}\eqnfigx{#3}
}

\newcommand{\eqnfig}[1]{
\begin{minipage}{3in}
\begin{center}
\begin{eqnarray}#1
\end{eqnarray}
\end{center}
\end{minipage}}

\newcommand{\twoeqnfig}[2]{
\eqnfig{#1}
\eqnfig{#2}
}

% transitions and SOS rules
\newcommand{\storetrans}[1]{\stackrel{#1}{\leadsto}}
\newcommand{\storetransx}[1]{\stackrel{#1}{\leadsto^*}}
\newcommand{\storetransprime}[3]{{#1} \storetrans{#2} {#3}}
\newcommand{\gammatrans}[1]{\stackrel{#1}{\hookrightarrow}}
\newcommand{\gammatransprime}[3]{{#1} \gammatrans{#2} {#3}}
\newcommand{\gammatransx}[1]{\stackrel{#1}{\hookrightarrow^*}}
%\newcommand{\trans}[3]{{#1} \stackrel{#2}{\longrightarrow} {#3}}
\newcommand{\run}[3]{{#1} \stackrel{#2}{\Longrightarrow} {#3}}
\newcommand{\sosrule}[2]{\frac{\begin{array}{c}#1\end{array}}{#2}}
\newcommand{\hoare}[3]{\{#1\}\ {\bf #2}\ \{#3\}}

\newcommand{\derivone}[3]{\mbox{\infer[#3]{#2}{#1}}}
\newcommand{\derivtwo}[4]{\mbox{\infer[#4]{#3}{{#1} & {#2}}}}
\newcommand{\derivdot}[4]{\mbox{\infer[#4]{#3}
{\begin{array}{l} {#1} \\ \vdots \\ {#2} \end{array}}}}

% comments, paragraphs
\newcommand{\mypar}[1]{\vspace*{\baselineskip}\noindent\emph{#1}}
\renewcommand{\paragraph}[1]{\noindent{\bf #1}}

% theories
\newcommand{\arith}{\langle \zed, +, \cdot, 0, 1 \rangle}
\newcommand{\hilbert}{\langle \zed, +, \cdot, 0, 1 \rangle^\exists}
\newcommand{\presbg}{\langle \zed, \geq, +, 0, 1 \rangle}
% \newcommand{\divides}{\langle \zed, +, |, 0, 1 \rangle}
\newcommand{\lipshitz}{\langle \zed, \geq, +, |, 0, 1 \rangle^\exists}
\newcommand{\dioph}[1]{\mathfrak{D}(#1)}

\let\Asterisk\undefined
\newcommand{\Asterisk}{\mathop{\scalebox{1.7}{\raisebox{-0.2ex}{$\ast$}}}}%

% proofs
\newif\ifLongVersion\LongVersiontrue
%\newcommand{\proof}[1]{\noindent {\em Proof}: {#1}}
%\newcommand{\qed}{\hfill$\Box$}
%\newcommand{\inproof}[1]{\ifLongVersion \proof{#1} \fi}
%\newcommand{\outproof}[2]{\ifLongVersion \else \vspace*{\baselineskip}\noindent Proof of {#1}: \proof{#2} \fi}
%\renewcommand{\subsubsection}[1]{\noindent{\bf #1}}

% multisorted first-order logic
\newcommand{\sorts}[1]{#1^\mathrm{S}}
\newcommand{\funs}[1]{#1^\mathrm{F}}
\newcommand{\ssorts}[1]{#1^\mathrm{s}}
\newcommand{\sfuns}[1]{#1^\mathrm{f}}
\newcommand{\sort}[1]{\sigma({#1})}
\newcommand{\mods}{\mathbf{I}}
\newcommand{\modsof}[2]{\sem{#1}_{#2}}
\newcommand{\iffeq}{\Leftrightarrow}

% separation logic
\newcommand{\symhp}{\Pi}
\newcommand{\symhs}{\Theta}
\newcommand{\locsi}{\mathsf{L}}
\newcommand{\heaps}{\mathsf{Heaps}}
\newcommand{\teq}{\approx}
\newcommand{\Spatial}{\Sigma}
\newcommand{\Emp}{\mathbf{emp}}
\newcommand{\PointsTo}[2]{ {#1 \mapsto #2} }
\newcommand{\PointsToL}[3]{ {#2 \mapsto^{#1} #3} }
\newcommand{\I}{\mathcal{I}}
\newcommand{\J}{\mathcal{J}}
\newcommand{\G}{\mathcal{G}}
\newcommand{\K}{\mathcal{K}}
\newcommand{\herb}{\mathcal{H}}
\newcommand{\tinyherb}{{\scriptscriptstyle\mathcal{H}}}
\newcommand{\locs}{\mathsf{Loc}}
\newcommand{\nil}{\mathsf{nil}}
\newcommand{\emp}{\mathsf{emp}}
\newcommand{\wand}{
 \mathrel{\mbox{$\hspace*{-0.03em}\mathord{-}\hspace*{-0.66em}
 \mathord{-}\hspace*{-0.36em}\mathord{*}$\hspace*{-0.005em}}}} % {\multimap}
\newcommand{\seplog}{\mathsf{SL}}
\newcommand{\tinyseplog}{\mathsf{\scriptscriptstyle{SL}}}
\newcommand{\seplogl}[1]{\mathsf{SL}\parens{#1}}
\newcommand{\tterm}{\mathsf{t}}
\newcommand{\uterm}{\mathsf{u}}
\newcommand{\vterm}{\mathsf{v}}
\newcommand{\wterm}{\mathsf{w}}
\newcommand{\xterm}{\mathsf{x}}
\newcommand{\yterm}{\mathsf{y}}
\newcommand{\zterm}{\mathsf{z}}
\newcommand{\nextp}{\mathsf{next}}
\newcommand{\datap}{\mathsf{data}}
\newcommand{\lterm}{\ell}
\newcommand{\pto}{\mathsf{pt}}
\newcommand{\sets}{\mathsf{Set}}
\newcommand{\lab}{\triangleleft}
\newcommand{\labd}[1]{#1\!\!\Downarrow}
\newcommand{\thsep}{\mathsf{Sep}}
\newcommand{\fv}[2]{\mathrm{FV}_{#1}(#2)}
\newcommand{\pset}[2]{\mathrm{P}^{#1}(#2)}
\newcommand{\voc}[1]{\mathrm{V}({#1})}
\newcommand{\terms}[1]{\mathrm{Pt}(#1)}
\newcommand{\bounds}[2]{\mathrm{Bnd}(#1,#2)}
\newcommand{\dom}{\mathrm{dom}}
\newcommand{\img}{\mathrm{img}}
\newcommand{\size}[1]{\mathrm{Size}(#1)}
\newcommand{\vars}{\mathsf{Var}}
\newcommand{\preds}{\mathsf{Pred}}
\newcommand{\X}{\mathcal{X}}
\newcommand{\Y}{\mathcal{Y}}
\newcommand{\sys}{\mathcal{S}}
\newcommand{\lseq}{\Gamma}
\newcommand{\rseq}{\Delta}
\newcommand{\psys}{\mathcal{P}}
\newcommand{\cf}{\mathcal{F}}
\newcommand{\sw}{\leq_{\mathrm{sw}}}
\newcommand{\subst}[1]{\mathrm{VIS}(#1)}
\newcommand{\DS}{\mathfrak{D}^{\mathsf{\scriptscriptstyle{s}}}}

\newcommand{\infname}[1]{\begin{array}{c}\text{ ($#1$) }\\\\\end{array}}

\newcommand{\subtree}{\sqsubset}
\newcommand{\subtreeq}{\sqsubseteq}
\newcommand{\suptree}{\sqsupset}

\newcommand{\ta}{\mathrm{TA}}
\newcommand{\LU}{\mathrm{LU}}
\newcommand{\RU}{\mathrm{RU}}
\newcommand{\RD}{\mathrm{RD}}
\newcommand{\RI}{\wedge\mathrm{R}}
\newcommand{\SP}{\mathrm{SP}}
\newcommand{\ID}{\mathrm{ID}}
\newcommand{\AX}{\mathrm{AX}}
\newcommand{\Aone}{\mathrm{A1}}
\newcommand{\Atwo}{\mathrm{A2}}
\newcommand{\perms}{\mathsf{Perm}}

\newcommand{\Label}[1]{ \llbracket #1 \rrbracket_{L} }
\DeclareMathOperator{\Disjoint}{disjoint}
\newcommand{\parens}[1]{ \left( #1 \right) }
\newcommand{\Set}[1]{ \left\{ #1 \right\} }
\DeclareMathOperator{\CardOp}{card}
\newcommand{\Card}[1]{ \CardOp\parens{#1} }
\DeclareMathOperator{\Wit}{witness}
\DeclareMathOperator{\Choice}{Choice}
\DeclareMathOperator{\Explain}{explain}

\newcommand{\funcsolve}{\small\mathsf{solve}}
\newcommand{\funcsmtsolve}{\small\mathsf{solve\_rec}}

% \newcommand{\lan}{\mathbf{L}}
\newcommand{\purify}[1]{\lfloor #1 \rfloor}
\newcommand{\purifyrec}[1]{\purify{#1}^\ast}
\newcommand{\unpurify}[1]{\lceil #1 \rceil}
\newcommand{\quants}[1]{\mathrm{Q}(#1)}
\newcommand{\solvesl}[1]{\mathsf{solve_{SL(#1)}}}
\newcommand{\solveslt}[1]{\mathsf{solve_{#1}}}
\newcommand{\lblf}{\mathsf{lbl}}
\newcommand{\ltrue}{\top}
\newcommand{\lfalse}{\bot}
\newcommand{\ord}[2]{\prec_{#1,#2}}
\newcommand{\qvar}{x}
\newcommand{\Int}{\mathbb{Z}}
\newcommand{\Data}{\mathbb{D}}
\newcommand{\Bool}{\mathbb{B}}
\newcommand{\true}{\mathtt{t}}
\newcommand{\false}{\mathtt{f}}
\newcommand{\euf}{\mathsf{E}}
\newcommand{\lia}{\mathsf{LIA}}
\newcommand{\euflia}{\mathsf{ELIA}}
\newcommand{\tinylia}{\mathsf{\scriptscriptstyle{LIA}}}
\newcommand{\ul}{\mathsf{UL}}
\newcommand{\ulk}{\mathsf{UL}^{\!\!\scriptscriptstyle{K}}}
\newcommand{\il}{\mathsf{IL}}
\newcommand{\ulid}{\mathsf{ULID}}
\newcommand{\theory}{\mathbb{T}}
\newcommand{\dual}[1]{{#1}^{\sim}}
\newcommand{\bigO}[1]{\mathcal{O}({#1})}

% automata
\newcommand{\A}{\mathcal{A}}
\newcommand{\Post}[1]{\mathsf{Post}({#1})}
\newcommand{\Accept}[2]{\mathsf{Acc}_{\scriptscriptstyle{#1}}({#2})}
\newcommand{\AbsPost}[1]{\abs{\mathsf{Post}}_{#1}}
\newcommand{\AbsAccept}[1]{\abs{\mathsf{Acc}}_{#1}}
\newcommand{\Inv}{\mathsf{I}}
\newcommand{\abs}[1]{{#1}^\sharp}
\newcommand{\Art}{\mathcal{T}}
\newcommand{\SubArt}{\mathcal{S}}
\newcommand{\worklist}{\mathtt{WorkList}}
\newcommand{\rootNode}{\mathtt{r}}
\newcommand{\pivot}[1]{\Psi(#1)}
\newcommand{\impact}{\textsc{Impact}}
\newcommand{\PA}{{\scriptscriptstyle{\mathsf{PA}}}}
\newcommand{\ADA}{{\scriptscriptstyle{\mathsf{ADA}}}}
\newcommand{\cube}[1]{\mathsf{c}({#1})}
\newcommand{\T}{\mathcal{T}}
\newcommand{\event}[1]{{#1}_\Sigma}
\newcommand{\data}[1]{{#1}_\Data}
\newcommand{\pathform}[1]{\Theta({#1})}
\newcommand{\accform}[1]{\Upsilon({#1})}
\newcommand{\accformA}[2]{\Upsilon_{#1}({#2})}
\newcommand{\substform}[1]{\overline{\Upsilon}({#1})}
\newcommand{\substformn}[1]{\overline{\Theta}({#1})}
\newcommand{\substformA}[2]{\overline{\Upsilon}_{#1}({#2})}        
\newcommand{\quantform}[1]{\widehat{\Upsilon}({#1})}
\newcommand{\quantformn}[1]{\widehat{\Theta}({#1})}
\newcommand{\nextf}[1]{{#1}^\prime}
\newcommand{\supsetforex}{\supseteq^{\scriptscriptstyle\forall\exists}}
\newcommand{\prefix}{\preceq}
\newcommand{\posforms}{\mathsf{Form}^+}
\newcommand{\cover}{\sqsubseteq}


%%%%%%%%%%%%%%%%%%%%%%%%%%%%%%%%%%%%%%%%%%%%%%%%%%%%%%%%%%%%%%%%%%%%%%%%%%%%%%%
\begin{document}
%%%%%%%%%%%%%%%%%%%%%%%%%%%%%%%%%%%%%%%%%%%%%%%%%%%%%%%%%%%%%%%%%%%%%%%%%%%%%%%

\title{First Order Alternation}

\author{Radu Iosif and Xiao Xu \\
CNRS, Verimag, Universit\'e de Grenoble Alpes \\
Email: \{Radu.Iosif,Xiao.Xu\}@univ-grenoble-alpes.fr}

\maketitle

\begin{abstract}
\end{abstract}

\section{Preliminaries}

A \emph{signature} $\sig = (\ssorts{\sig},\sfuns{\sig})$ consists of a
set $\ssorts{\sig}$ of \emph{sort symbols} and a set $\sfuns{\sig}$ of
sorted \emph{function symbols}. To simplify the presentation, we
assume w.l.o.g. that $\ssorts{\sig} = \set{\Data,\Bool}$\footnote{The
  generalization to more than two sorts is without difficulty, but
  would unnecessarily clutter the technical presentation.} and each
function symbol $f \in \sfuns{\sig}$ has $\#(f) \geq 0$ arguments of
sort $\Data$ and return value $\sigma(f) \in \ssorts{\sig}$. If
$\#(f)=0$ then $f$ is a \emph{constant}. Let
$\top$ and $\bot$ be constants of sort $\Bool$.

We consider an infinite countable set of \emph{variables} $\vars$,
where each $x \in \vars$ has an associated sort $\sigma(x)$. A
\emph{term} $t$ of sort $\sigma(t)=S$ is a variable $x \in \vars$
where $\sigma(x)=S$, or $f(t_1,\ldots,t_{\#(f)})$ where
$t_1,\ldots,t_{\#(f)}$ are terms of sort $\Data$ and $\sigma(f)=S$. An
\emph{atom} is a term of sort $\Bool$ or an equality $t \teq s$
between two terms of sort $\Data$. A \emph{formula} is a quantified
boolean combination of atoms, where the quantifiers are as $\exists x
~.~ \phi$ and $\forall x ~.~ \phi$.

We denote by $\fv{\sigma}{\phi}$ the set of free variables of sort
$\sigma$ in $\phi$ and write $\fv{}{\phi}$ for $\bigcup_{\sigma \in
  \ssorts{\sig}} \fv{\sigma}{\phi}$. For a variable $x \in
\fv{}{\phi}$ and a term $t$ such that $\sigma(t) = \sigma(x)$, let
$\phi[t/x]$ be the result of replacing each occurrence of $x$ by
$t$. For indexed sets $T=\set{t_1,\ldots,t_n}$ and
$X=\set{x_1,\ldots,x_n}$, we write $\phi[T/X]$ for
the formula obtained by simultaneously replacing $x_i$ with $t_i$ in
$\phi$, for all $i\in[1,n]$. The size $\len{\phi}$ is the number of
symbols occuring in $\phi$.

An \emph{interpretation} $\I$ maps\begin{inparaenum}[(1)]
\item the sort $\Data$ into a non-empty set $\Data^\I$, 
%
\item the sort $\Bool$ into the set $\booli = \set{\true,\false}$, where
  $\top^\I = \true$, $\bot^\I = \false$, and
%
\item each function symbol $f$ into a total function $f^\I :
  (\Data^\I)^{\#(f)} \rightarrow \sigma(f)^I$, or an element of
  $\sigma(f)^I$ when $\#(f)=0$.
\end{inparaenum}
Given an interpretation $\I$, a \emph{valuation} $\nu$ maps each
variable $x \in \vars$ into an element $\nu(x) \in \sigma(x)^\I$. For
a term $t$, we denote by $t^\I_\nu$ the value obtained by replacing
each function symbol $f$ by its interpretation $f^\I$ and each
variable $x$ by its valuation $\nu(x)$. For a formula $\phi$, we write
$\I,\nu \models \phi$ if the formula obtained by replacing each term
$t$ in $\phi$ by the value $t^\I_\nu$ is logically equivalent to true.
The semantics of quantifiers is given as usual: $\I,\nu \models
\exists x ~.~ \phi$ ($\forall x ~.~ \phi$) if and only if $\I,\nu[x
  \leftarrow u] \models \phi$, for some (for all) value(s) $u \in
\sigma(x)^\I$, where $\nu[x \leftarrow u]$ is the valuation that
assigns $u$ to $x$ and $\nu(y)$ to all $y \neq x$.

A formula $\phi$ is \emph{satisfiable} in the interpretation $\I$ if
there exists a valuation $\nu$ such that $\I,\nu \models \phi$, and
\emph{valid} if $\I,\nu \models \phi$ for all valuations $\nu$.  The
\emph{first-order theory} $\theory(\sig,\I)$ is the set of valid
formulae written in the signature $\sig$, with the interpretation
$\I$. A \emph{decision procedure} for $\theory(\sig,\I)$ is an
algorithm that takes a formula $\phi$ in the signature $\sig$ and
returns yes if and only if $\phi \in \theory(\sig,\I)$.

Given formulae $\varphi$ and $\psi$, we say that \emph{$\phi$ entails
  $\psi$}, denoted $\phi \models^\I \psi$ iff $\I,\nu \models \varphi$
implies $\I,\nu \models \psi$, for each valuation $\nu$, and $\phi
\iffeq^\I \psi$ iff $\phi \models^\I \psi$ and $\psi \models^\I \phi$.
We omit mentioning the interpretation $\I$ when it is clear from the
context.

\section{First Order Alternating Data Automata}

We fix an interpretation $\I$ and a finite alphabet $\Sigma$ of
\emph{input events}. Given a finite set $X \subset \vars$ of
variables of sort $\Data$, let $X \mapsto \Data^\I$ be the set
of valuations of the variables $X$ and $\Sigma[X] = \Sigma
\times (X \mapsto \Data^\I)$ be the possibly infinite set of
\emph{data symbols}. A \emph{data word} (word in the sequel) is a
finite sequence $(a_1,\nu_1)(a_2,\nu_2) \ldots (a_n,\nu_n)$ of data
symbols, where $a_1,\ldots,a_n \in \Sigma$ and $\nu_1,\ldots,\nu_n :
X \rightarrow \Data^\I$ are valuations. We denote by
$\varepsilon$ the empty sequence, by $\Sigma^*$ the set of finite
sequences of input events and by $\Sigma[X]^*$ the set of data
words over the variables $X$.

Given sets of variables $B,X \subset \vars$ of sort $\Bool$ and
$\Data$, respectively, we denote by $\Form(B,X,Y)$ the set of formulae
$\phi$ such that $\fv{\Bool}{\phi} = B$, $\fv{\Data}{\phi} = X$ and
each variable from $Y$ is of sort $\Data$ and occurs under the scope
of a quantifier. By $\Form^+(B,X,Y)$ we denote the subset of
$\Form(B,X,Y)$ in which each boolean variable occurs under an even
number of negations. We denote by $\phi(B,X,Y)$ any formula from
$\Form(B,X,Y)$. 

A \emph{first-order alternating data automaton} (fo-ADA or automaton
in the sequel) is a tuple $\A =
\tuple{G,L,Q,\iota,F,\Delta}$, where: \begin{compactitem}
%
\item $G \subseteq \vars$ is a finite set of \emph{global
  variables} of sort $\Data$, 
%
\item $L \subseteq \vars$ is a finite set of \emph{local
  variables} of sort $\Data$,
%
\item $Q \subset \vars$ is a finite set of variables of sort $\Bool$
  (\emph{states}),
%
\item $\iota \in \Form^+(Q,\emptyset,\emptyset)$ is the \emph{initial
  configuration},
%
\item $\Delta : Q \times \Sigma \rightarrow
  \Form^+(Q,\overline{G}\cup G\cup\overline{L},L)$
  is a \emph{transition function},
\end{compactitem}
where, for a set $X \subseteq \vars$, we denote $\overline{X} =
\set{\overline{x} \mid x \in X}$. Intuitively, the variables
$\overline{x}$ from a transition formula $\Delta(q,a)$ refer to past
values, whereas the other variables refer to the current
values. Observe that the initial values are unconstrained (since the
initial configuration is a boolean combination of states without
variables) and quantification is allowed only over the current local
variables. In other words, we do not allow quantification over past
values, which are beyond our control. The size of $\A$ is defined as
$\len{\A} = \len{\iota} + \sum_{(q,a) \in
  Q\times\Sigma}\len{\Delta(q,a)}$.

In order to define the semantics of automata, we consider time-stamped
sets of variables $\stamp{X}{n} = \{\stamp{x}{n} \mid x \in X\}$ for
any $n\geq0$. For an input event $a \in \Sigma$, some constant
$n\geq0$ and a formula $\phi$, we write $\stamp{\Delta}{n}(\phi,a)$
for the formula obtained from $\phi$ by simultaneously replacing each
state $q \in \fv{\Bool}{\phi}$ with
$\Delta(\phi,a)[\stamp{G}{n}/\overline{G}, \stamp{G}{n+1}/G,
  \stamp{L}{n}/\overline{L}, \stamp{L}{n+1}/L]$.

Given a word $w = (a_1,\nu_1)(a_2,\nu_2) \ldots (a_n,\nu_n) \in
\Sigma[G]^*$ the \emph{run} of $\A$ over $w$ is the sequence of
formulae: \[\phi_0 \Rightarrow \phi_1(Q,\stamp{G}{0} \cup \stamp{G}{1},
\stamp{L}{0} \cup \stamp{L}{1}) \Rightarrow \ldots \Rightarrow
\phi_n(Q,\stamp{G}{0} \cup \ldots \cup \stamp{G}{n}, \stamp{L}{0} \cup
\ldots \cup \stamp{L}{n})\]
where: \begin{compactitem}
\item $\phi_0 \equiv \iota$, 
%
\item $\phi_1 \equiv \exists \stamp{L}{0} ~.~ \stamp{\Delta}{0}(\iota,a_1)$, and
%
\item $\phi_{i+1} \equiv \stamp{\Delta}{i}(\phi_i,a_{i+1})$, for all $i \in [1,n-1]$.
\end{compactitem}
We say that $\A$ \emph{accepts} $w$ if $\I,\nu \models
\phi_n(Q,\stamp{G}{0} \cup \ldots \cup \stamp{G}{n}, \stamp{L}{0} \cup
\ldots \cup \stamp{L}{n})$, where $\nu$ is a valuation such
that: \begin{compactitem}
\item $\nu(\stamp{g}{i}) = \nu_i(g)$ for each $g \in G$, 
%
\item $\nu(q) = \true$ for each $q \in F$, and
%
\item $\nu(q) = \false$ for each $q \in Q \setminus F$.
\end{compactitem}
The language of $\A$ is the set $L(\A)$ of words from $\Sigma[G]^*$
accepted by $\A$. In the following, we ask two
questions: \begin{compactenum}
\item given automata $\A_i = \tuple{G,L_i,Q_i,\iota_i,F_i,\Delta_i}$,
  with $L_1 \cap L_2 = \emptyset$ and $Q_1 \cap Q_2 = \emptyset$, do
  there exist automata $\A_\cap$, $\A_\cup$ and $\overline{\A}_1$ such
  that $L(\A_\cap) = L(\A_1) \cap L(\A_2)$, $L(\A_\cup) = L(\A_1) \cup
  L(\A_2)$ and $L(\overline{\A}_1) = \Sigma[G]^* \setminus L(\A_1)$ ?
%
\item given an automaton $\A$, is $L(\A) = \emptyset$ ? 
\end{compactenum}

%%%%%%%%%%%%%%%%%%%%%%%%%%%%%%%%%%%%%%%%%%%%%%%%%%%%%%%%%%%%%%%%%%%%%%%%%%%%%%%
\end{document}
%%%%%%%%%%%%%%%%%%%%%%%%%%%%%%%%%%%%%%%%%%%%%%%%%%%%%%%%%%%%%%%%%%%%%%%%%%%%%%%
