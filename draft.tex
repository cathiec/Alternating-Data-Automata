\documentclass{llncs}

\usepackage[usenames,dvipsnames]{color}

\usepackage{latexsym}
\usepackage{amsxtra} 
\usepackage{amssymb}
\usepackage{amsmath}
\usepackage{pslatex}
\usepackage{epsfig}
\usepackage{wrapfig}
\usepackage{paralist}
\usepackage{graphics}
\usepackage{stmaryrd}
\usepackage{txfonts}
\usepackage{framed}
\usepackage{makecell}
\usepackage{url}
\usepackage{tikz}
\usetikzlibrary{automata,positioning, calc}
\usepackage[inline,shortlabels]{enumitem}

\usepackage{proof}

\usepackage{algorithm}
\usepackage{algorithmicx}
\usepackage[noend]{algpseudocode}

\usepackage[draft]{commenting}

\pagestyle{plain}

%\newtheorem{definition}{Definition}
%\newtheorem{lemma}{Lemma}
%\newtheorem{theorem}{Theorem}
%\newtheorem{proposition}{Proposition}
%\newtheorem{corollary}{Corollary}
%\newtheorem{conjecture}{Conjecture}
%\newtheorem{claim}{Claim}
%\newtheorem{property}{Property}
\newtheorem{fact}{Fact}

\newcommand*\circled[1]{\tikz[baseline=(char.base)]{
            \node[shape=circle,draw,inner sep=0.5pt] (char) {#1};}}

% Marius commands
\newcommand{\Z}{{\mathbb Z}}
\newcommand{\N}{{\mathbb N}}
\newcommand{\relmat}[1] { \{\!\!\{#1\}\!\!\}}
\newcommand{\matrel}[1] { [\!\![#1]\!\!]}
\newcommand{\key}[1]{\underline{#1}}

% complexity 
\newcommand{\np}{$\mathsf{NP}$}
\newcommand{\pspace}{$\mathsf{PSPACE}$~}
\newcommand{\sigmatwop}{$\mathsf{\Sigma}_2^p$}
\newcommand{\pitwop}{$\mathsf{\Pi}_2^p$}

% semantic stuff
\newcommand{\bi}[1]{{}^\omega\!{#1}^\omega}
\newcommand{\rbr}{{\bf ]\!]}}
\newcommand{\lbr}{{\bf [\![}}
\newcommand{\sem}[1]{\lbr #1 \rbr}
\newcommand{\model}[2]{\sem{#1}_{#2}}
\newcommand{\lfp}[1]{\mbox{fix}\ {#1}}
\newcommand{\wpr}{\widetilde{pre}}
\newcommand{\cceil}[1]{\lfloor #1 \rfloor + 1}
\newcommand{\set}[1]{\left\{ #1 \right\}}
\newcommand{\tuple}[1]{\left\langle #1 \right\rangle}
\renewcommand{\vec}[1]{\mathbf #1}
\newcommand{\tup}[1]{\overline #1}
\newcommand{\ite}[3]{\mathsf{ite}(#1,#2,#3)}
\newcommand{\tupite}[2]{\mathsf{ite}(#1,#2)}

% alias
\newcommand{\alias}{\Diamond}
\newcommand{\may}[2]{{\langle #1 \rangle}{#2}}
\newcommand{\must}[2]{{\lbrack #1 \rbrack}{#2}}
\newcommand{\defd}[1]{\delta({#1})}

% definitions
\newcommand{\defequal}{\stackrel{\scriptscriptstyle{def}}{=}}
\newcommand{\defequiv}{\stackrel{\scriptscriptstyle{def}}{\equiv}}

% algorithms
\newcommand{\sig}{\mathsf{S}}

% recursive structures
\newcommand{\LS}{\mathtt{LS}}
\newcommand{\DLL}{\mathtt{DLL}}
\newcommand{\TLL}{\mathtt{TLL}}
\newcommand{\TREE}{\mathtt{TREE}}
\newcommand{\ax}{\mathtt{a}}
\newcommand{\bx}{\mathtt{b}}
\newcommand{\cx}{\mathtt{c}}
\newcommand{\dx}{\mathtt{d}}

% cardinal, degree, ...
\newcommand{\sgn}{\mbox{sgn}}
%\newcommand{\gcd}{\mbox{gcd}}
\newcommand{\lcm}{\mbox{lcm}}
\newcommand{\len}[1]{{|{#1}|}}
\newcommand{\card}[1]{{|\!|{#1}|\!|}}
\newcommand{\minimum}[2]{\mbox{min}({#1},{#2})}
\newcommand{\maximum}[2]{\mbox{max}({#1},{#2})}
\newcommand{\minset}[2]{\mbox{min}\{{#1}\ |\ {#2}\}}
\newcommand{\maxset}[2]{\mbox{max}\{{#1}\ |\ {#2}\}}
\newcommand{\divs}[1]{\mbox{div}({#1})}

% automata and languages
\newcommand{\auth}[1]{{\mathcal A}_{#1}}
\newcommand{\lang}[1]{{\mathcal L}({#1})}
\newcommand{\runs}[1]{{\mathcal R}({#1})}
\newcommand{\val}[1]{{\mathcal V}({#1})}
\newcommand{\trace}[1]{{Tr}({#1})}
\newcommand{\state}[1]{{\Sigma}({#1})}
\newcommand{\trans}[1]{{\Theta}({#1})}
\newcommand{\access}[3]{{\mathcal L}_{{#1},{#2}}({#3})}
\newcommand{\eats}{\rightarrow}
\newcommand{\Eats}[1]{\stackrel{#1}{\longrightarrow}}
\newcommand{\arrow}[2]{\xrightarrow{{\scriptstyle #1}}_{{\scriptstyle #2}}}
\newcommand{\larrow}[2]{\xleftarrow[{\scriptstyle #2}]{{\scriptstyle #1}}} 

% numbers
\newcommand{\nat}{{\bf \mathbb{N}}}
\newcommand{\zed}{{\bf \mathbb{Z}}}
\newcommand{\rat}{{\bf \mathbb{Q}}}
\newcommand{\real}{{\bf \mathbb{R}}}
\newcommand{\linzed}{{\bf \mathcal{L}\mathbb{Z}}}

% powerset
\newcommand{\pow}[1]{{\mathcal P}(#1)}
\newcommand{\pown}[2]{{\mathcal P}^{#1}(#2)}

% by definition
\newcommand{\bydef}{\stackrel{\Delta}{=}}

% projection, lifting
\newcommand{\proj}[2]{{#1}\!\!\downarrow_{{#2}}}
\newcommand{\lift}[2]{{#1}\!\!\uparrow_{{#2}}}

% forward, backward
\newcommand{\forw}[1]{\stackrel{\rightarrow}{#1}}
\newcommand{\back}[1]{\stackrel{\leftarrow}{#1}}

% equation figures
\newcommand{\eqnfigx}[1]{
\begin{minipage}{3in}
\begin{center}
\begin{eqnarray*}#1
\end{eqnarray*}
\end{center}
\end{minipage}}

\newcommand{\twoeqnfigx}[2]{
\eqnfigx{#1}
\eqnfigx{#2}
}

\newcommand{\threeeqnfigx}[3]{
\eqnfigx{#1}\\
\eqnfigx{#2}\eqnfigx{#3}
}

\newcommand{\eqnfig}[1]{
\begin{minipage}{3in}
\begin{center}
\begin{eqnarray}#1
\end{eqnarray}
\end{center}
\end{minipage}}

\newcommand{\twoeqnfig}[2]{
\eqnfig{#1}
\eqnfig{#2}
}

% transitions and SOS rules
\newcommand{\storetrans}[1]{\stackrel{#1}{\leadsto}}
\newcommand{\storetransx}[1]{\stackrel{#1}{\leadsto^*}}
\newcommand{\storetransprime}[3]{{#1} \storetrans{#2} {#3}}
\newcommand{\gammatrans}[1]{\stackrel{#1}{\hookrightarrow}}
\newcommand{\gammatransprime}[3]{{#1} \gammatrans{#2} {#3}}
\newcommand{\gammatransx}[1]{\stackrel{#1}{\hookrightarrow^*}}
%\newcommand{\trans}[3]{{#1} \stackrel{#2}{\longrightarrow} {#3}}
\newcommand{\run}[3]{{#1} \stackrel{#2}{\Longrightarrow} {#3}}
\newcommand{\sosrule}[2]{\frac{\begin{array}{c}#1\end{array}}{#2}}
\newcommand{\hoare}[3]{\{#1\}\ {\bf #2}\ \{#3\}}

\newcommand{\derivone}[3]{\mbox{\infer[#3]{#2}{#1}}}
\newcommand{\derivtwo}[4]{\mbox{\infer[#4]{#3}{{#1} & {#2}}}}
\newcommand{\derivdot}[4]{\mbox{\infer[#4]{#3}
{\begin{array}{l} {#1} \\ \vdots \\ {#2} \end{array}}}}

% comments, paragraphs
\newcommand{\mypar}[1]{\vspace*{\baselineskip}\noindent\emph{#1}}
\renewcommand{\paragraph}[1]{\noindent{\bf #1}}

% theories
\newcommand{\arith}{\langle \zed, +, \cdot, 0, 1 \rangle}
\newcommand{\hilbert}{\langle \zed, +, \cdot, 0, 1 \rangle^\exists}
\newcommand{\presbg}{\langle \zed, \geq, +, 0, 1 \rangle}
% \newcommand{\divides}{\langle \zed, +, |, 0, 1 \rangle}
\newcommand{\lipshitz}{\langle \zed, \geq, +, |, 0, 1 \rangle^\exists}
\newcommand{\dioph}[1]{\mathfrak{D}(#1)}

\let\Asterisk\undefined
\newcommand{\Asterisk}{\mathop{\scalebox{1.7}{\raisebox{-0.2ex}{$\ast$}}}}%

% proofs
\newif\ifLongVersion\LongVersiontrue
%\newcommand{\proof}[1]{\noindent {\em Proof}: {#1}}
%\newcommand{\qed}{\hfill$\Box$}
%\newcommand{\inproof}[1]{\ifLongVersion \proof{#1} \fi}
%\newcommand{\outproof}[2]{\ifLongVersion \else \vspace*{\baselineskip}\noindent Proof of {#1}: \proof{#2} \fi}
%\renewcommand{\subsubsection}[1]{\noindent{\bf #1}}

% multisorted first-order logic
\newcommand{\sorts}[1]{#1^\mathrm{S}}
\newcommand{\funs}[1]{#1^\mathrm{F}}
\newcommand{\ssorts}[1]{#1^\mathrm{s}}
\newcommand{\sfuns}[1]{#1^\mathrm{f}}
\newcommand{\mods}{\mathbf{I}}
\newcommand{\modsof}[2]{\sem{#1}_{#2}}
\newcommand{\iffeq}{\Leftrightarrow}

% separation logic
\newcommand{\symhp}{\Pi}
\newcommand{\symhs}{\Theta}
\newcommand{\locsi}{\mathsf{L}}
\newcommand{\heaps}{\mathsf{Heaps}}
\newcommand{\teq}{\approx}
\newcommand{\Spatial}{\Sigma}
\newcommand{\Emp}{\mathbf{emp}}
\newcommand{\PointsTo}[2]{ {#1 \mapsto #2} }
\newcommand{\PointsToL}[3]{ {#2 \mapsto^{#1} #3} }
\newcommand{\I}{\mathcal{I}}
\newcommand{\J}{\mathcal{J}}
\newcommand{\herb}{\mathcal{H}}
\newcommand{\tinyherb}{{\scriptscriptstyle\mathcal{H}}}
\newcommand{\locs}{\mathsf{Loc}}
\newcommand{\data}{\mathsf{Data}}
\newcommand{\nil}{\mathsf{nil}}
\newcommand{\emp}{\mathsf{emp}}
\newcommand{\wand}{
 \mathrel{\mbox{$\hspace*{-0.03em}\mathord{-}\hspace*{-0.66em}
 \mathord{-}\hspace*{-0.36em}\mathord{*}$\hspace*{-0.005em}}}} % {\multimap}
\newcommand{\seplog}{\mathsf{SL}}
\newcommand{\tinyseplog}{\mathsf{\scriptscriptstyle{SL}}}
\newcommand{\seplogl}[1]{\mathsf{SL}\parens{#1}}
\newcommand{\tterm}{\mathsf{t}}
\newcommand{\uterm}{\mathsf{u}}
\newcommand{\vterm}{\mathsf{v}}
\newcommand{\wterm}{\mathsf{w}}
\newcommand{\xterm}{\mathsf{x}}
\newcommand{\yterm}{\mathsf{y}}
\newcommand{\zterm}{\mathsf{z}}
\newcommand{\nextp}{\mathsf{next}}
\newcommand{\datap}{\mathsf{data}}
\newcommand{\lterm}{\ell}
\newcommand{\pto}{\mathsf{pt}}
\newcommand{\sets}{\mathsf{Set}}
\newcommand{\lab}{\triangleleft}
\newcommand{\labd}[1]{#1\!\!\Downarrow}
\newcommand{\thsep}{\mathsf{Sep}}
\newcommand{\fv}[2]{\mathrm{FV}^{#1}(#2)}
\newcommand{\terms}[1]{\mathrm{Pt}(#1)}
\newcommand{\bounds}[2]{\mathrm{Bnd}(#1,#2)}
\newcommand{\dom}{\mathrm{dom}}
\newcommand{\img}{\mathrm{img}}
\newcommand{\size}[1]{\mathrm{Size}(#1)}
\newcommand{\vars}{\mathsf{Var}}
\newcommand{\preds}{\mathsf{\Pi}}
\newcommand{\T}{\mathcal{T}}
\newcommand{\X}{\mathcal{X}}
\newcommand{\Y}{\mathcal{Y}}
\newcommand{\sys}{\mathcal{S}}
\newcommand{\lseq}{\Gamma}
\newcommand{\rseq}{\Delta}
\newcommand{\psys}{\mathcal{P}}
\newcommand{\cf}{\mathcal{F}}
\newcommand{\sw}{\leq_{\mathrm{sw}}}
\newcommand{\subst}[1]{\mathrm{VIS}(#1)}
\newcommand{\DS}{\mathfrak{D}^{\mathsf{\scriptscriptstyle{s}}}}

\newcommand{\infname}[1]{\begin{array}{c}\text{ ($#1$) }\\\\\end{array}}

\newcommand{\subtree}{\sqsubset}
\newcommand{\subtreeq}{\sqsubseteq}
\newcommand{\suptree}{\sqsupset}

\newcommand{\ta}{\mathrm{TA}}
\newcommand{\LU}{\mathrm{LU}}
\newcommand{\RU}{\mathrm{RU}}
\newcommand{\RD}{\mathrm{RD}}
\newcommand{\RI}{\wedge\mathrm{R}}
\newcommand{\SP}{\mathrm{SP}}
\newcommand{\ID}{\mathrm{ID}}
\newcommand{\AX}{\mathrm{AX}}
\newcommand{\Aone}{\mathrm{A1}}
\newcommand{\Atwo}{\mathrm{A2}}
\newcommand{\perms}{\mathsf{Perm}}

\newcommand{\Label}[1]{ \llbracket #1 \rrbracket_{L} }
\DeclareMathOperator{\Disjoint}{disjoint}
\newcommand{\parens}[1]{ \left( #1 \right) }
\newcommand{\Set}[1]{ \left\{ #1 \right\} }
\DeclareMathOperator{\CardOp}{card}
\newcommand{\Card}[1]{ \CardOp\parens{#1} }
\DeclareMathOperator{\Wit}{witness}
\DeclareMathOperator{\Choice}{Choice}
\DeclareMathOperator{\Explain}{explain}

\newcommand{\funcsolve}{\small\mathsf{solve}}
\newcommand{\funcsmtsolve}{\small\mathsf{solve\_rec}}

% \newcommand{\lan}{\mathbf{L}}
\newcommand{\purify}[1]{\lfloor #1 \rfloor}
\newcommand{\purifyrec}[1]{\purify{#1}^\ast}
\newcommand{\unpurify}[1]{\lceil #1 \rceil}
\newcommand{\quants}[1]{\mathrm{Q}(#1)}
\newcommand{\solvesl}[1]{\mathsf{solve_{SL(#1)}}}
\newcommand{\solveslt}[1]{\mathsf{solve_{#1}}}
\newcommand{\lblf}{\mathsf{lbl}}
\newcommand{\ltrue}{\top}
\newcommand{\lfalse}{\bot}
\newcommand{\ord}[2]{\prec_{#1,#2}}
\newcommand{\qvar}{x}
\newcommand{\Int}{\mathsf{Int}}
\newcommand{\Data}{\mathsf{Data}}
\newcommand{\Bool}{\mathsf{Bool}}
\newcommand{\booli}{\mathbb{B}}
\newcommand{\true}{\mathtt{true}}
\newcommand{\false}{\mathtt{false}}
\newcommand{\Form}{\mathsf{Form}}
\newcommand{\euf}{\mathsf{E}}
\newcommand{\lia}{\mathsf{LIA}}
\newcommand{\euflia}{\mathsf{ELIA}}
\newcommand{\tinylia}{\mathsf{\scriptscriptstyle{LIA}}}
\newcommand{\ul}{\mathsf{UL}}
\newcommand{\ulk}{\mathsf{UL}^{\!\!\scriptscriptstyle{K}}}
\newcommand{\il}{\mathsf{IL}}
\newcommand{\ulid}{\mathsf{ULID}}
\newcommand{\theory}{\mathbb{T}}

% automata
\newcommand{\A}{\mathcal{A}}
\newcommand{\Post}[1]{\mathsf{Post}_{#1}}
\newcommand{\Accept}[1]{\mathsf{Acc}_{#1}}
\newcommand{\AbsPost}[1]{\abs{\mathsf{Post}}_{#1}}
\newcommand{\AbsAccept}[1]{\abs{\mathsf{Acc}}_{#1}}
\newcommand{\Inv}{\mathsf{I}}
\newcommand{\abs}[1]{{#1}^\sharp}
\newcommand{\Art}{\mathcal{T}}
\newcommand{\SubArt}{\mathcal{S}}
\newcommand{\worklist}{\mathtt{WorkList}}
\newcommand{\rootNode}{\mathtt{r}}
\newcommand{\pivot}[1]{\Psi(#1)}
\newcommand{\impact}{\textsc{Impact}}

\newcommand{\sparagraph}[1]{\smallskip
\noindent 
{\bf #1}\ }


%%%%%%%%%%%%%%%%%%%%%%%%%%%%%%%%%%%%%%%%%%%%%%%%%%%%%%%%%%%%%%%%%%%%%%%%%%%%%%%
\begin{document}
%%%%%%%%%%%%%%%%%%%%%%%%%%%%%%%%%%%%%%%%%%%%%%%%%%%%%%%%%%%%%%%%%%%%%%%%%%%%%%%

\title{The Impact of Alternation}

\author{Radu Iosif \and Xiao Xu}

\institute{Verimag/CNRS/Universit\'e de Grenoble Alpes}

\maketitle

\begin{abstract}
\end{abstract}

\section{Introduction}

\section{Preliminaries}

A \emph{signature} $\sig = (\ssorts{\sig},\sfuns{\sig})$ consists of a
set $\ssorts{\sig}$ of \emph{sort symbols} and a set $\sfuns{\sig}$ of
sorted \emph{function symbols}. To simplify the presentation, we
assume w.l.o.g. that $\ssorts{\Sigma}$ consists of two distinct sorts
$\Data$ and $\Bool$ and each function symbol $f \in \sfuns{\sig}$ has
$\#(f) \geq 0$ arguments of sort $\Data$ and return value $\sigma(f)
\in \ssorts{\sig}$. If $\#(f)=0$ then $f$ is a \emph{constant}. We
consider the constants $\mathbf{0}$ of sort $\Data$ and $\top,\bot$ of
sort $\Bool$, denoting the truth values true and false.

We consider an infinite countable set of \emph{variables} $\vars$,
where each $x \in \vars$ has an associated sort $\sigma(x)$. Terms are
defined recursively: a term $t$ of sort $\sigma(t)=S$ is a variable $x
\in \vars$ where $\sigma(x)=S$, or $f(t_1,\ldots,t_{\#(f)})$ where
$t_1,\ldots,t_{\#(f)}$ are terms of sort $\Data$ and $\sigma(f)=S$. An
\emph{atom} is a term of sort $\Bool$ or an equality $t \teq s$
between two terms of sort $\Data$. A \emph{literal} is an atom or its
negation. A \emph{formula} is a combination of atoms using disjunction
$\vee$, conjunction $\wedge$ and negation $\neg$. We denote by
$\fv{S}{\phi}$ the set of variables of sort $S$ in $\phi$ and write
$\fv{}{\phi}$ for $\bigcup_{S \in \set{\Data,\Bool}}
\fv{S}{\phi}$. For a variable $x \in \fv{}{\phi}$ and a term $t$ such
that $\sigma(t) = \sigma(x)$, we denote by $\phi[t/x]$ the result of
replacing each occurrence of $x$ by $t$. For indexed sets
$\vec{t}=\set{t_1,\ldots,t_n}$ and $\vec{x}=\set{x_1,\ldots,x_n}$, we
write $\phi[\vec{t}/\vec{x}]$ for the formula obtained by
simultaneously replacing $x_i$ with $t_i$ in $\phi$, for all
$i=1,\ldots,n$. A formula is in positive normal form (PNF) if negation
occurs only within literals and in disjunctive normal form (DNF) if it
is a disjunction of conjunctions of literals $\bigvee_{i=1}^N
\bigwedge_{j=1}^{M_i} \ell_{ij}$.

An \emph{interpretation} $\I$ maps\begin{inparaenum}[(1)]
\item the sort $\Data$ into a non-empty set $\Data^\I$, 
%
\item the sort $\Bool$ into the set $\Bool^\I$ containing the boolean
  values, and
%
\item each function symbol $f$ into a total function $f^\I :
  \left(\Data^\I\right)^{\#(f)} \rightarrow \sigma(f)^I$, or an element of
  $\sigma(f)^I$ when $\#(f)=0$.
\end{inparaenum}
Given an interpretation $\I$, a \emph{valuation} $\nu$ maps each
variable $x \in \vars$ into an element $\nu(x) \in \sigma(x)^\I$. For
a term $t$, we denote by $t^\I_\nu$ the value obtained by replacing
each function symbol $f$ by its interpretation $f^\I$ and each
variable $x$ by its valuation $\nu(x)$. For a formula $\phi$, we write
$\I,\nu \models \phi$ if the formula obtained by replacing each term
$t$ in $\phi$ by the value $t^\I_\nu$ is logically equivalent to true.
A formula $\phi$ is \emph{satisfiable} in the interpretation $\I$ if
there exists a valuation $\nu$ such that $\I,\nu \models \phi$. Given
formulae $\varphi$ and $\psi$, we say that \emph{$\phi$ entails
  $\psi$}, denoted $\phi \models^\I \psi$ iff $\I,\nu \models \varphi$
implies $\I,\nu \models \psi$, for each valuation $\nu$, and $\phi
\eqiff^\I \psi$ iff $\phi \models^\I \psi$ and $\psi \models^\I \phi$.
We omit mentioning the interpretation $\I$ when it is clear from the
context.

\section{Alternating Data Automata}

In the rest of this section we fix an interpretation $\I$ and a finite
alphabet $\Sigma$ of \emph{input events}. Given a finite set $\vec{x}
\subset \vars$ of variables of sort $\Data$, let $\vec{x} \mapsto
\Data^\I$ be the set of valuations of the variables $\vec{x}$ and
$\Sigma[\vec{x}] = \Sigma \times (\vec{x} \mapsto \Data^\I)$ be the
set of \emph{data symbols}. A \emph{data word} (or, simply, a word) is
a finite sequence $(a_1,\nu_1)(a_2,\nu_2) \ldots (a_n,\nu_n)$ of data
symbols, where $a_1,\ldots,a_n \in \Sigma$ and $\nu_1,\ldots,\nu_n :
\vec{x} \rightarrow \Data^\I$ are valuations. We denote by
$\Sigma[\vec{x}]^*$ the set of data words over $\vec{x}$. This
definition extends the classical notion of words from a finite
alphabet with a possibly infinite alphabet $\Sigma[\vec{x}]$. Clearly,
when $\Data^\I$ is sufficiently large or infinite, we can map the
elements of $\Sigma$ into designated elements of $\Data^\I$ and use a
special variable to encode the input events. However, keeping $\Sigma$
explicit in the following simplifies several technical points below,
without cluttering the presentation.

Given sets of variables $\vec{b},\vec{x} \subset \vars$ of sort
$\Bool$ and $\Data$, respectively, we denote by
$\Form^+(\vec{b},\vec{x})$ the set of formulae $\phi$ such that
$\fv{\Bool}{\phi} \subseteq \vec{b}$, $\fv{\Data}{\phi} \subseteq
\vec{x}$ and each variable from $\vec{b}$ occurs under an even number
of negations. 

An \emph{alternating data automaton} (or, simply, and automaton) is a
tuple $\A = \tuple{\vec{x},Q,\iota,F,\Delta}$: \begin{compactitem}
%
\item $\vec{x} \subset \vars$ is a finite set of variables of sort
  $\Data$,
%
\item $Q$ is a finite set of variables of sort $\Bool$, also called
  \emph{states},
%
\item $\iota \in \Form^+(Q,\emptyset)$ is a formula describing the
  \emph{initial configurations},
%
\item $F \subseteq Q$ is a set of \emph{final} states, and
%
\item $\Delta : Q \times \Sigma \rightarrow
  \Form^+(Q,\overline{\vec{x}}\cup\vec{x})$ is a \emph{transition
    function}, where $\overline{\vec{x}}$ is the set $\{\overline{x}
  \mid x \in \vec{x}\}$.
\end{compactitem}
Intuitively, $\overline{\vec{x}}$ track the previous and $\vec{x}$ the
current values of the variables of $\A$. Formally, let $\vec{x}_k =
\set{x_k \mid x \in \vec{x}}$, $k\geq0$, be a time-stamped set of
variables. For an input even $a \in \Sigma$ and a formula $\phi$ we
write $\Delta^k(\phi,a)$ for the formula obtained from $\phi$ by
simultaneously replacing each state $q \in \fv{\Bool}{\phi}$ by the
formula
$\Delta(q,a)[\vec{x}_k/\overline{\vec{x}},\vec{x}_{k+1}/\vec{x}]$, for
all $k\geq0$.

Given a word $w = (a_1,\nu_1)(a_2,\nu_2) \ldots (a_n,\nu_n)$, the
\emph{run} of $\A$ over $w$ is the sequence of formulae
$\phi_0,\phi_1,\ldots,\phi_n$, where $\phi_0 \equiv \iota$ and, for
all $i > 0$, we have $\phi_k \equiv \Delta^k(\phi_{k-1},a_k)$. We
slightly abuse notation and write $\Delta(\iota,a_1,\ldots,a_n)$ for
the formula $\phi_n(\vec{x}_0,\ldots,\vec{x}_n)$ in the
following. $\A$ \emph{accepts} $w$ iff $\I,\nu \models \phi_n$, where
$\nu$ is the valuation that maps:\begin{inparaenum}[(1)]
\item each $x \in \vec{x}_0$ to $\mathbf{0}$, 
%
\item each $x \in \vec{x}_k$ to $\nu_k(x)$, for all $k\in[1,n]$, 
%
\item each $q \in \fv{\Bool}{\phi_n} \cap F$ to $\top$, and 
%
\item each $q \in \fv{\Bool}{\phi_n} \setminus F$ to $\bot$.
\end{inparaenum}
The language of $\A$, denoted as $L(\A)$, is the set of words accepted
by $\A$. 

In this section we tackle the following problems: \begin{compactitem}
\item \emph{boolean closure}: given automata $\A_1$ and $\A_2$, both
  with the same set of variables $\vec{x}$, does there exist automata
  $\A_\cup$, $\A_\cap$ and $\overline{\A}$ such that $L(\A_\cup) =
  \A_1 \cup \A_2$, $L(A_\cap) = \A_1 \cap \A_2$ and $L(\overline{\A_1})
  = \Sigma[\vec{x}]^* \setminus L(\A_1)$ ?
%
\item \emph{emptiness}: given an automaton $\A$, is it the case that
  $L(\A) = \emptyset$ ?
\end{compactitem}

It is well known that other problems, such as \emph{universality}
(given an automaton $\A$ with variables $\vec{x}$, does $L(\A) =
\Sigma[\vec{x}]^*$) and \emph{inclusion} (given automata $\A_1$ and
$\A_2$ with the same set of variables, does $L(\A_1) \subseteq
L(\A_2)$ ?) can be reduced to the above problems. Observe furthermore
that we do not consider cases in which the sets of variables in the
two automata differ. An interesting problem in this case is: given
automata $\A_1$ and $\A_2$, with variables $\vec{x}_1$ and
$\vec{x}_2$, respectively, such that $\vec{x}_1 \subseteq \vec{x}_2$,
does $L(\A_1) \subseteq \proj{L(\A_2)}{\vec{x}_1}$, where
$\proj{L(\A_2)}{\vec{x}_1}$ is the projection of the set of words
$L(\A_2)$ onto the variables $\vec{x}_1$? This problem is considered
as future work.

\subsection{Boolean Closure}

For a formula $\phi \in \Form^+(Q,\vec{x})$, with no negated
occurrences of the boolean variables in $Q$, we define the
formula $\overline{\phi} \in \Form^+(Q,\vec{x})$ recursively on
the structure of $\phi$:
\[\begin{array}{lclclclc}
\overline{\phi_1 \vee \phi_2} & \equiv & \overline{\phi_1} \wedge \overline{\phi_2} && 
\overline{\phi_1 \wedge \phi_2} & \equiv & \overline{\phi_1} \vee \overline{\phi_2} \\
\overline{\phi} & \equiv & \phi & \text{ if $\phi \in Q$}\hspace*{1cm} &
\overline{\phi} & \equiv & \neg\phi & \text{if $\phi \not\in Q$ is an atom}
\end{array}\]
If $\phi$ does not contain variables from $Q$, then $\overline{\phi}$
and $\neg\phi$ are equivalent, and in general:

\begin{proposition}\label{prop:overline}
  Given a formula $\phi \in \Form^+(Q,\vec{x})$ and a valuation $\nu$
  mapping each $q \in Q$ a value $\nu(q) \in \Bool^\I$ and each $x \in
  \vec{x}$ to a value $\nu(x) \in \Data^\I$, let $\nu'$ be the
  valuation that assigns each $q \in Q$ the value $\neg\nu(q)$ and
  each $x \in \vec{x}$ the value $\nu(x)$. Then we have 
  $\I,\nu \models \phi \iff \I,\nu' \not\models \overline{\phi}$. 
\end{proposition}
\proof{Immediate, by induction on the structure of $\phi$. \qed}

In the following let $\A_1 = \tuple{\vec{x},Q_1,\iota_1,F_1,\Delta_1}$
and $\A_2 = \tuple{\vec{x},Q_2,\iota_2,F_2,\Delta_2}$, where
w.l.o.g. we assume that $Q_1 \cap Q_2 = \emptyset$. We define $\A_\cup
= \tuple{\vec{x},Q_1\cup Q_2,\iota_1 \vee \iota_2,F_1\cup
  F_2,\Delta_1\cup\Delta_2}$, $\A_\cap = \tuple{\vec{x},Q_1\cup
  Q_2,\iota_1 \wedge \iota_2, F_1\cup F_2, \Delta_1\cup\Delta_2}$ and
$\overline{\A_1} = \langle \vec{x},Q_1,\iota_1,Q_1\setminus
F_1,\overline{\Delta_1} \rangle$, where $\overline{\Delta_1}(q,a)
\equiv \overline{\Delta_1(q,a)}$, for all $q \in Q_1$ and $a \in
\Sigma$. The following lemma shows the correctness of these
definitions: 

\begin{lemma}\label{lemma:closure}
  For any automata $\A_1 = \tuple{\vec{x},Q_1,\iota_1,F_1,\Delta_1}$
  and $\A_2 = \tuple{\vec{x},Q_2,\iota_2,F_2,\Delta_2}$ such that $Q_1
  \cap Q_2 = \emptyset$, we have $L(\A_\cup) = L(\A_1) \cup L(\A_2)$,
  $L(\A_\cap) = L(\A_1) \cap L(\A_2)$ and $L(\overline{\A_1}) =
  \Sigma[\vec{x}]^* \setminus L(\A_1)$.
\end{lemma}
\proof{ We prove $L(\A_\cup) = L(\A_1) \cup L(\A_2)$ first, the proof
  for $\A_\cap$ being analogous. Let $w = (a_1,\nu_1) \ldots
  (a_n,\nu_n) \in L(\A_\cup)$ be a word, where $n=0$ corresponds to
  the empty word. We prove by induction on $n\geq0$ that
  $\Delta(\iota_1 \vee \iota_2,a_1\ldots a_n) \eqiff
  \Delta(\iota_1,a_1\ldots a_n) \vee \Delta(\iota_2,a_1\ldots
  a_n)$. The case $n=0$ follows from the definition of the initial
  configuration of $\A_\cup$. For the inductive step $n>0$,
  $\Delta(\iota_1 \vee \iota_2,a_1\ldots a_n)$ is obtained from
  $\Delta(\iota_1 \vee \iota_2,a_1\ldots a_{n-1})$ by replacing each
  variable $q \in \fv{\Bool}{\Delta(\iota_1 \vee \iota_2,a_1\ldots
    a_{n-1})}$ with
  $\Delta(q,a_n)[\vec{x}_{n-1}/\overline{\vec{x}},\vec{x}_n/\vec{x}]$,
  denoted $\Delta^n(\Delta(\iota_1 \vee \iota_2,a_1\ldots
  a_{n-1}),a_n)$. Since by induction hypothesis, $\Delta(\iota_1 \vee
  \iota_2,a_1\ldots a_{n-1}) \eqiff \Delta(\iota_1,a_1\ldots a_{n-1})
  \vee \Delta(\iota_2,a_1\ldots a_{n-1})$, we obtain: 
  \[\begin{array}{rcl}
  \Delta^n(\Delta(\iota_1 \vee \iota_2,a_1\ldots a_{n-1}),a_n) & \iff & 
  \Delta^n(\Delta(\iota_1,a_1\ldots a_{n-1}),a_n) \vee
  \Delta^n(\Delta(\iota_2,a_1\ldots a_{n-1}),a_n) \\
  & \iff & \Delta(\iota_1,a_1\ldots a_{n}) \vee \Delta(\iota_2,a_1\ldots
  a_{n}) \enspace.
  \end{array}\] 
  To prove $L(\overline{\A_1}) = \Sigma[\vec{x}]^* \setminus L(\A_1)$,
  let $w = (a_1,\nu_1) \ldots (a_n,\nu_n)$ be a word and show that
  $\overline{\Delta}(\iota_1,a_1 \ldots a_n) =
  \overline{\Delta(\iota_1,a_1 \ldots a_n)}$ by induction on
  $n\geq0$. The case $n=0$ is immediate, because $\fv{}{\iota_1}
  \subseteq Q$ and thus $\overline{\iota_1} = \iota_1$. For the case
  $n>0$, we compute:
  \[\begin{array}{rcll}
  \overline{\Delta}^n(\overline{\Delta}(\iota_1,a_1\ldots a_{n-1}),a_n) & \eqiff & \overline{\Delta}^n(\overline{\Delta(\iota_1,a_1\ldots a_{n-1})},a_n) & \text{ by induction hypothesis} \\
  & \eqiff & \overline{\Delta(\iota_1,a_1\ldots,a_n)} & \text{ by the definition of $\overline{\phi}$ .}
  \end{array}\]
  Let $\nu,\nu' : Q \cup \bigcup_{i=0}^n\vec{x}_i$ be two valuations
  such that: \begin{compactitem}
  \item $\nu(q) = \top$ and $\nu'(q) = \bot$, for each $q \in F$, 
    %
  \item $\nu(q) = \bot$ and $\nu'(q) = \top$, for each $q \in Q \setminus F$, 
    %
  \item $\nu(x) = \nu'(x) = \mathbf{0}$, for each $x \in \vec{x}_0$, 
    %
  \item $\nu(x) = \nu'(x) = \nu_i(x)$, for each $x \in \vec{x}_i$ and each $i
    \in [1,n]$.
  \end{compactitem}
  By Proposition \ref{prop:overline}, we have $\I,\nu \models
  \Delta(\iota_1,a_1 \ldots a_n) \iff \I,\nu' \not\models
  \overline{\Delta(\iota_1,a_1 \ldots a_n)} \iff \I,\nu' \not\models
  \overline{\Delta}(\iota_1,a_1 \ldots a_n)$. Thus $w \in L(\A_1) \iff
  w \not\in L(\overline{\A_1})$, i.e. $L(\overline{\A_1}) =
  \Sigma[\vec{x}]^* \setminus L(\A_1)$. \qed}

\section{Predicate Abstraction for Emptiness}

Unlike the boolean closure properties, showed to be effectively
decidable (Lemma \ref{lemma:closure}), the emptiness problem is
undecidable, even in very simple cases. For instance, if $\Data^\I$ is
the set of positive integers, an ADA can simulate an Alternating
Vector Addition System with States (AVASS) \cite{LazicSchmitz14} using
only atoms $x = \overline{x} + k$, for $k \in \nat$ and $x \teq 0$,
with the classical interpretation of function symbols on
integers. Since reachability of a configuration $(q,\nu)$, where $q$
is a control state and $\nu$ a valuation of the variables, is
undecidable, the emptiness problem is undecidable for ADA, in general. 

We thus turn to semi-algorithms for this problem and develop methods
that are supposed to be both: \begin{compactenum}
\item \emph{sound}: given an automaton $\A$, if $L(\A) \neq\emptyset$,
  the procedure will terminate and return a word $w \in L(\A)$), and
%
\item \emph{complete relative to termination}: if the procedure terminates
  without returning such a word, then $L(\A) = \emptyset$. 
\end{compactenum}


%%%%%%%%%%%%%%%%%%%%%%%%%%%%%%%%%%%%%%%%%%%%%%%%%%%%%%%%%%%%%%%%%%%%%%%%%%%%%%%
\bibliographystyle{abbrv} \bibliography{refs}
%%%%%%%%%%%%%%%%%%%%%%%%%%%%%%%%%%%%%%%%%%%%%%%%%%%%%%%%%%%%%%%%%%%%%%%%%%%%%%%

%%%%%%%%%%%%%%%%%%%%%%%%%%%%%%%%%%%%%%%%%%%%%%%%%%%%%%%%%%%%%%%%%%%%%%%%%%%%%%%
\end{document}
%%%%%%%%%%%%%%%%%%%%%%%%%%%%%%%%%%%%%%%%%%%%%%%%%%%%%%%%%%%%%%%%%%%%%%%%%%%%%%%
