%\documentclass{llncs}
\documentclass[acmsmall]{acmart}

%\usepackage[usenames,dvipsnames]{color}

%\let\iint\undefined 
%\let\iiint\undefined 
%\let\iiiint\undefined 
%\let\idotsint\undefined

\usepackage{amsmath} 
\usepackage{amsxtra} 
\usepackage{amssymb}
%\usepackage{mathabx}
\usepackage{mathtools}
\usepackage{textcomp}
\usepackage{xifthen}
%\usepackage{silence}
%\WarningsOff

\usepackage{latexsym}
%\usepackage{pslatex}
%\usepackage{epsfig}
%\usepackage{wrapfig}
%\usepackage{graphics}
\usepackage{enumerate}
%\usepackage{cancel}

\usepackage{paralist}

%\usepackage{ listings }

%\usepackage{algorithm}
%\usepackage{algorithmicx}
%\usepackage[noend]{algpseudocode}
%\let\Asterisk\undefined

\usepackage{xparse}
\usepackage{pdfsync}

\pagestyle{empty}

%needed macros from the coopy-pasted preliminaries
%%%%%%%%%%%%%%%%%%%%%%%%%%%%%%%%%%%%%%%%%%%%%%%%%%%%%%%%%%%%%%%%%
\newcommand{\nat}{{\mathbb N}}
\newcommand{\domain}{\mathcal {D}}
%\newcommand{\set}[1]{\left\{ #1 \right\}}
\newcommand{\true}{\mathtt{true}}
\newcommand{\false}{\mathtt{false}}
\newcommand{\tuple}[1]{\left\langle #1 \right\rangle}
\newcommand{\theo}{\mathsf{Th}}
\newcommand{\thd}{\theo(\mathcal{D})}
\newcommand{\modelsthd}{\models_{\thd}}
\def\proj{\mathbin{\downarrow}}

\DeclareDocumentCommand\vect{ m o o }{%
    {% 
        \IfNoValueT {#2} {\bar #1} 
        \IfNoValueF {#2} {%
			\IfNoValueT	{#3} {\bar #1_{#2}}%
			%\IfNoValueF {#3} {^{(#2,#3)}}%
			\IfNoValueF {#3} {\bar #1^{(#2,#3)}}%
		}%
    }%
}

\newcommand{\set}{\vect}
%\DeclareDocumentCommand\set{ g g }{%
%    {%	
%		\IfNoValueT {#2} {\{\bar #1\}}%
%       	\IfNoValueF {#2} {\{\bar #1_{(#2)}\}}%
%    }%
%}

\newcommand{\subst}[3]{#1[#2/#3]}%substitute in #1 occurences of #2 by #3
\newcommand{\Nonempty}{\mathit{Nonemp}}
\newcommand{\Universal}{\mathit{Univ}}
\newcommand{\Nonuniversal}{\mathit{Nonuniv}}
%\newcommand{\Empty}{\mathit{Emp}}
\newcommand{\Language}[1]{\mathit{Lang_{#1}}}
\newcommand{\Path}[1]{\mathit{Path_{#1}}}
\newcommand{\Run}[1]{\mathit{Run_{#1}}}
\newcommand{\Prefix}[1]{\mathit{Pref_{#1}}}
\newcommand{\Suffix}[1]{\mathit{Suff_{#1}}}
\newcommand{\States}[1]{\mathit{States_{#1}}}
\newcommand{\Configurations}[1]{\mathit{Confs_{#1}}}
\newcommand{\Configuration}[1]{\mathit{Conf_{#1}}}

%\newcommand{\arrow}[2]{\xrightarrow{_{_{#1}}}_{{\scriptstyle #2}}}
\newcommand{\arrow}[2]{\xrightarrow{{\scriptstyle #1}}_{{\scriptstyle #2}}}
%\newcommand{\arrow}[2]{\rightarrow_{{#1}}^{{\scriptstyle #2}}}
\def\projobs{\mathbin{\downarrow_{\Xobs}}}

\newcommand{\viz}{x}
\newcommand{\inviz}{y}
\newcommand{\lang}{\mathcal {L}}
\newcommand{\dvalue}{v}
\newcommand{\dval}{\dvalue}

\newcommand{\Xobs}{\vect{x}}
\newcommand{\Xinv}{\vect{y}}
\newcommand{\Xinvz}{\vect{z}}

\newcommand{\vx}{\vect{x}}
\newcommand{\vy}{\vect{y}}
\newcommand{\vz}{\vect{z}}

\newcommand{\initial}{I}
\newcommand{\final}{F}
\newcommand{\accepting}{F}
\newcommand{\transition}{\tau}

\newcommand{\fdef}{\mathrel{:=}}
%%%%%%%%%%%%%%%%%%%%%%%%%%%%%%%%%%%%%%%%%%%%%%%%%%%%%%%%%%%%%%%%%


%%\newtheorem{definition}{Definition}
%\newtheorem{lemma}{Lemma}
%\newtheorem{theorem}{Theorem}
%\newtheorem{proposition}{Proposition}
%\newtheorem{corollary}{Corollary}
%\newtheorem{conjecture}{Conjecture}
%\newtheorem{claim}{Claim}
%\newtheorem{property}{Property}
\newtheorem{fact}{Fact}

\newcommand*\circled[1]{\tikz[baseline=(char.base)]{
            \node[shape=circle,draw,inner sep=0.5pt] (char) {#1};}}

% Marius commands
\newcommand{\Z}{{\mathbb Z}}
\newcommand{\N}{{\mathbb N}}
\newcommand{\relmat}[1] { \{\!\!\{#1\}\!\!\}}
\newcommand{\matrel}[1] { [\!\![#1]\!\!]}
\newcommand{\key}[1]{\underline{#1}}

% complexity 
\newcommand{\np}{$\mathsf{NP}$}
\newcommand{\pspace}{$\mathsf{PSPACE}$~}
\newcommand{\sigmatwop}{$\mathsf{\Sigma}_2^p$}
\newcommand{\pitwop}{$\mathsf{\Pi}_2^p$}

% semantic stuff
\newcommand{\bi}[1]{{}^\omega\!{#1}^\omega}
\newcommand{\rbr}{{\bf ]\!]}}
\newcommand{\lbr}{{\bf [\![}}
\newcommand{\sem}[1]{\lbr #1 \rbr}
\newcommand{\model}[2]{\sem{#1}_{#2}}
\newcommand{\lfp}[1]{\mbox{fix}\ {#1}}
\newcommand{\wpr}{\widetilde{pre}}
\newcommand{\cceil}[1]{\lfloor #1 \rfloor + 1}
\newcommand{\set}[1]{\left\{ #1 \right\}}
\newcommand{\tuple}[1]{\left\langle #1 \right\rangle}
\renewcommand{\vec}[1]{\mathbf #1}
\newcommand{\tup}[1]{\overline #1}
\newcommand{\ite}[3]{\mathsf{ite}(#1,#2,#3)}
\newcommand{\tupite}[2]{\mathsf{ite}(#1,#2)}

% alias
\newcommand{\alias}{\Diamond}
\newcommand{\may}[2]{{\langle #1 \rangle}{#2}}
\newcommand{\must}[2]{{\lbrack #1 \rbrack}{#2}}
\newcommand{\defd}[1]{\delta({#1})}

% definitions
\newcommand{\defequal}{\stackrel{\scriptscriptstyle{def}}{=}}
\newcommand{\defequiv}{\stackrel{\scriptscriptstyle{def}}{\equiv}}

% algorithms
\newcommand{\sig}{\mathsf{S}}

% recursive structures
\newcommand{\LS}{\mathtt{LS}}
\newcommand{\DLL}{\mathtt{DLL}}
\newcommand{\TLL}{\mathtt{TLL}}
\newcommand{\TREE}{\mathtt{TREE}}
\newcommand{\ax}{\mathtt{a}}
\newcommand{\bx}{\mathtt{b}}
\newcommand{\cx}{\mathtt{c}}
\newcommand{\dx}{\mathtt{d}}

% cardinal, degree, ...
\newcommand{\sgn}{\mbox{sgn}}
%\newcommand{\gcd}{\mbox{gcd}}
\newcommand{\lcm}{\mbox{lcm}}
\newcommand{\len}[1]{{|{#1}|}}
\newcommand{\card}[1]{{|\!|{#1}|\!|}}
\newcommand{\minimum}[2]{\mbox{min}({#1},{#2})}
\newcommand{\maximum}[2]{\mbox{max}({#1},{#2})}
\newcommand{\minset}[2]{\mbox{min}\{{#1}\ |\ {#2}\}}
\newcommand{\maxset}[2]{\mbox{max}\{{#1}\ |\ {#2}\}}
\newcommand{\divs}[1]{\mbox{div}({#1})}

% automata and languages
\newcommand{\auth}[1]{{\mathcal A}_{#1}}
\newcommand{\lang}[1]{{\mathcal L}({#1})}
\newcommand{\runs}[1]{{\mathcal R}({#1})}
\newcommand{\val}[1]{{\mathcal V}({#1})}
\newcommand{\trace}[1]{{Tr}({#1})}
\newcommand{\state}[1]{{\Sigma}({#1})}
\newcommand{\trans}[1]{{\Theta}({#1})}
\newcommand{\access}[3]{{\mathcal L}_{{#1},{#2}}({#3})}
\newcommand{\eats}{\rightarrow}
\newcommand{\Eats}[1]{\stackrel{#1}{\longrightarrow}}
\newcommand{\arrow}[2]{\xrightarrow{{\scriptstyle #1}}_{{\scriptstyle #2}}}
\newcommand{\larrow}[2]{\xleftarrow[{\scriptstyle #2}]{{\scriptstyle #1}}} 

% numbers
\newcommand{\nat}{{\bf \mathbb{N}}}
\newcommand{\zed}{{\bf \mathbb{Z}}}
\newcommand{\rat}{{\bf \mathbb{Q}}}
\newcommand{\real}{{\bf \mathbb{R}}}
\newcommand{\linzed}{{\bf \mathcal{L}\mathbb{Z}}}

% powerset
\newcommand{\pow}[1]{{\mathcal P}(#1)}
\newcommand{\pown}[2]{{\mathcal P}^{#1}(#2)}

% by definition
\newcommand{\bydef}{\stackrel{\Delta}{=}}

% projection, lifting
\newcommand{\proj}[2]{{#1}\!\!\downarrow_{{#2}}}
\newcommand{\lift}[2]{{#1}\!\!\uparrow_{{#2}}}

% forward, backward
\newcommand{\forw}[1]{\stackrel{\rightarrow}{#1}}
\newcommand{\back}[1]{\stackrel{\leftarrow}{#1}}

% equation figures
\newcommand{\eqnfigx}[1]{
\begin{minipage}{3in}
\begin{center}
\begin{eqnarray*}#1
\end{eqnarray*}
\end{center}
\end{minipage}}

\newcommand{\twoeqnfigx}[2]{
\eqnfigx{#1}
\eqnfigx{#2}
}

\newcommand{\threeeqnfigx}[3]{
\eqnfigx{#1}\\
\eqnfigx{#2}\eqnfigx{#3}
}

\newcommand{\eqnfig}[1]{
\begin{minipage}{3in}
\begin{center}
\begin{eqnarray}#1
\end{eqnarray}
\end{center}
\end{minipage}}

\newcommand{\twoeqnfig}[2]{
\eqnfig{#1}
\eqnfig{#2}
}

% transitions and SOS rules
\newcommand{\storetrans}[1]{\stackrel{#1}{\leadsto}}
\newcommand{\storetransx}[1]{\stackrel{#1}{\leadsto^*}}
\newcommand{\storetransprime}[3]{{#1} \storetrans{#2} {#3}}
\newcommand{\gammatrans}[1]{\stackrel{#1}{\hookrightarrow}}
\newcommand{\gammatransprime}[3]{{#1} \gammatrans{#2} {#3}}
\newcommand{\gammatransx}[1]{\stackrel{#1}{\hookrightarrow^*}}
%\newcommand{\trans}[3]{{#1} \stackrel{#2}{\longrightarrow} {#3}}
\newcommand{\run}[3]{{#1} \stackrel{#2}{\Longrightarrow} {#3}}
\newcommand{\sosrule}[2]{\frac{\begin{array}{c}#1\end{array}}{#2}}
\newcommand{\hoare}[3]{\{#1\}\ {\bf #2}\ \{#3\}}

\newcommand{\derivone}[3]{\mbox{\infer[#3]{#2}{#1}}}
\newcommand{\derivtwo}[4]{\mbox{\infer[#4]{#3}{{#1} & {#2}}}}
\newcommand{\derivdot}[4]{\mbox{\infer[#4]{#3}
{\begin{array}{l} {#1} \\ \vdots \\ {#2} \end{array}}}}

% comments, paragraphs
\newcommand{\mypar}[1]{\vspace*{\baselineskip}\noindent\emph{#1}}
\renewcommand{\paragraph}[1]{\noindent{\bf #1}}

% theories
\newcommand{\arith}{\langle \zed, +, \cdot, 0, 1 \rangle}
\newcommand{\hilbert}{\langle \zed, +, \cdot, 0, 1 \rangle^\exists}
\newcommand{\presbg}{\langle \zed, \geq, +, 0, 1 \rangle}
% \newcommand{\divides}{\langle \zed, +, |, 0, 1 \rangle}
\newcommand{\lipshitz}{\langle \zed, \geq, +, |, 0, 1 \rangle^\exists}
\newcommand{\dioph}[1]{\mathfrak{D}(#1)}

\let\Asterisk\undefined
\newcommand{\Asterisk}{\mathop{\scalebox{1.7}{\raisebox{-0.2ex}{$\ast$}}}}%

% proofs
\newif\ifLongVersion\LongVersiontrue
%\newcommand{\proof}[1]{\noindent {\em Proof}: {#1}}
%\newcommand{\qed}{\hfill$\Box$}
%\newcommand{\inproof}[1]{\ifLongVersion \proof{#1} \fi}
%\newcommand{\outproof}[2]{\ifLongVersion \else \vspace*{\baselineskip}\noindent Proof of {#1}: \proof{#2} \fi}
%\renewcommand{\subsubsection}[1]{\noindent{\bf #1}}

% multisorted first-order logic
\newcommand{\sorts}[1]{#1^\mathrm{S}}
\newcommand{\funs}[1]{#1^\mathrm{F}}
\newcommand{\ssorts}[1]{#1^\mathrm{s}}
\newcommand{\sfuns}[1]{#1^\mathrm{f}}
\newcommand{\mods}{\mathbf{I}}
\newcommand{\modsof}[2]{\sem{#1}_{#2}}
\newcommand{\iffeq}{\Leftrightarrow}

% separation logic
\newcommand{\symhp}{\Pi}
\newcommand{\symhs}{\Theta}
\newcommand{\locsi}{\mathsf{L}}
\newcommand{\heaps}{\mathsf{Heaps}}
\newcommand{\teq}{\approx}
\newcommand{\Spatial}{\Sigma}
\newcommand{\Emp}{\mathbf{emp}}
\newcommand{\PointsTo}[2]{ {#1 \mapsto #2} }
\newcommand{\PointsToL}[3]{ {#2 \mapsto^{#1} #3} }
\newcommand{\I}{\mathcal{I}}
\newcommand{\J}{\mathcal{J}}
\newcommand{\herb}{\mathcal{H}}
\newcommand{\tinyherb}{{\scriptscriptstyle\mathcal{H}}}
\newcommand{\locs}{\mathsf{Loc}}
\newcommand{\data}{\mathsf{Data}}
\newcommand{\nil}{\mathsf{nil}}
\newcommand{\emp}{\mathsf{emp}}
\newcommand{\wand}{
 \mathrel{\mbox{$\hspace*{-0.03em}\mathord{-}\hspace*{-0.66em}
 \mathord{-}\hspace*{-0.36em}\mathord{*}$\hspace*{-0.005em}}}} % {\multimap}
\newcommand{\seplog}{\mathsf{SL}}
\newcommand{\tinyseplog}{\mathsf{\scriptscriptstyle{SL}}}
\newcommand{\seplogl}[1]{\mathsf{SL}\parens{#1}}
\newcommand{\tterm}{\mathsf{t}}
\newcommand{\uterm}{\mathsf{u}}
\newcommand{\vterm}{\mathsf{v}}
\newcommand{\wterm}{\mathsf{w}}
\newcommand{\xterm}{\mathsf{x}}
\newcommand{\yterm}{\mathsf{y}}
\newcommand{\zterm}{\mathsf{z}}
\newcommand{\nextp}{\mathsf{next}}
\newcommand{\datap}{\mathsf{data}}
\newcommand{\lterm}{\ell}
\newcommand{\pto}{\mathsf{pt}}
\newcommand{\sets}{\mathsf{Set}}
\newcommand{\lab}{\triangleleft}
\newcommand{\labd}[1]{#1\!\!\Downarrow}
\newcommand{\thsep}{\mathsf{Sep}}
\newcommand{\fv}[2]{\mathrm{FV}^{#1}(#2)}
\newcommand{\terms}[1]{\mathrm{Pt}(#1)}
\newcommand{\bounds}[2]{\mathrm{Bnd}(#1,#2)}
\newcommand{\dom}{\mathrm{dom}}
\newcommand{\img}{\mathrm{img}}
\newcommand{\size}[1]{\mathrm{Size}(#1)}
\newcommand{\vars}{\mathsf{Var}}
\newcommand{\preds}{\mathsf{\Pi}}
\newcommand{\T}{\mathcal{T}}
\newcommand{\X}{\mathcal{X}}
\newcommand{\Y}{\mathcal{Y}}
\newcommand{\sys}{\mathcal{S}}
\newcommand{\lseq}{\Gamma}
\newcommand{\rseq}{\Delta}
\newcommand{\psys}{\mathcal{P}}
\newcommand{\cf}{\mathcal{F}}
\newcommand{\sw}{\leq_{\mathrm{sw}}}
\newcommand{\subst}[1]{\mathrm{VIS}(#1)}
\newcommand{\DS}{\mathfrak{D}^{\mathsf{\scriptscriptstyle{s}}}}

\newcommand{\infname}[1]{\begin{array}{c}\text{ ($#1$) }\\\\\end{array}}

\newcommand{\subtree}{\sqsubset}
\newcommand{\subtreeq}{\sqsubseteq}
\newcommand{\suptree}{\sqsupset}

\newcommand{\ta}{\mathrm{TA}}
\newcommand{\LU}{\mathrm{LU}}
\newcommand{\RU}{\mathrm{RU}}
\newcommand{\RD}{\mathrm{RD}}
\newcommand{\RI}{\wedge\mathrm{R}}
\newcommand{\SP}{\mathrm{SP}}
\newcommand{\ID}{\mathrm{ID}}
\newcommand{\AX}{\mathrm{AX}}
\newcommand{\Aone}{\mathrm{A1}}
\newcommand{\Atwo}{\mathrm{A2}}
\newcommand{\perms}{\mathsf{Perm}}

\newcommand{\Label}[1]{ \llbracket #1 \rrbracket_{L} }
\DeclareMathOperator{\Disjoint}{disjoint}
\newcommand{\parens}[1]{ \left( #1 \right) }
\newcommand{\Set}[1]{ \left\{ #1 \right\} }
\DeclareMathOperator{\CardOp}{card}
\newcommand{\Card}[1]{ \CardOp\parens{#1} }
\DeclareMathOperator{\Wit}{witness}
\DeclareMathOperator{\Choice}{Choice}
\DeclareMathOperator{\Explain}{explain}

\newcommand{\funcsolve}{\small\mathsf{solve}}
\newcommand{\funcsmtsolve}{\small\mathsf{solve\_rec}}

% \newcommand{\lan}{\mathbf{L}}
\newcommand{\purify}[1]{\lfloor #1 \rfloor}
\newcommand{\purifyrec}[1]{\purify{#1}^\ast}
\newcommand{\unpurify}[1]{\lceil #1 \rceil}
\newcommand{\quants}[1]{\mathrm{Q}(#1)}
\newcommand{\solvesl}[1]{\mathsf{solve_{SL(#1)}}}
\newcommand{\solveslt}[1]{\mathsf{solve_{#1}}}
\newcommand{\lblf}{\mathsf{lbl}}
\newcommand{\ltrue}{\top}
\newcommand{\lfalse}{\bot}
\newcommand{\ord}[2]{\prec_{#1,#2}}
\newcommand{\qvar}{x}
\newcommand{\Int}{\mathsf{Int}}
\newcommand{\Data}{\mathsf{Data}}
\newcommand{\Bool}{\mathsf{Bool}}
\newcommand{\booli}{\mathbb{B}}
\newcommand{\true}{\mathtt{true}}
\newcommand{\false}{\mathtt{false}}
\newcommand{\Form}{\mathsf{Form}}
\newcommand{\euf}{\mathsf{E}}
\newcommand{\lia}{\mathsf{LIA}}
\newcommand{\euflia}{\mathsf{ELIA}}
\newcommand{\tinylia}{\mathsf{\scriptscriptstyle{LIA}}}
\newcommand{\ul}{\mathsf{UL}}
\newcommand{\ulk}{\mathsf{UL}^{\!\!\scriptscriptstyle{K}}}
\newcommand{\il}{\mathsf{IL}}
\newcommand{\ulid}{\mathsf{ULID}}
\newcommand{\theory}{\mathbb{T}}

% automata
\newcommand{\A}{\mathcal{A}}
\newcommand{\Post}[1]{\mathsf{Post}_{#1}}
\newcommand{\Accept}[1]{\mathsf{Acc}_{#1}}
\newcommand{\AbsPost}[1]{\abs{\mathsf{Post}}_{#1}}
\newcommand{\AbsAccept}[1]{\abs{\mathsf{Acc}}_{#1}}
\newcommand{\Inv}{\mathsf{I}}
\newcommand{\abs}[1]{{#1}^\sharp}
\newcommand{\Art}{\mathcal{T}}
\newcommand{\SubArt}{\mathcal{S}}
\newcommand{\worklist}{\mathtt{WorkList}}
\newcommand{\rootNode}{\mathtt{r}}
\newcommand{\pivot}[1]{\Psi(#1)}
\newcommand{\impact}{\textsc{Impact}}

\newcommand{\sparagraph}[1]{\smallskip
\noindent 
{\bf #1}\ }



%%%%%%%%%%%%%%%%%%%%%%%%%%%%%%%%%%%%%%%%%%%%%%%%%%%%%%%%%%%%%%%%%%%%%%%%%%%%%%%
\begin{document}
%%%%%%%%%%%%%%%%%%%%%%%%%%%%%%%%%%%%%%%%%%%%%%%%%%%%%%%%%%%%%%%%%%%%%%%%%%%%%%%

%\title{Predicate Abstraction for Trace Inclusion}

%\author{Luk\'{a}\v{s} Hol\'{i}k\inst{1} \and Radu Iosif\inst{2} \and Adam Rogalewicz\inst{1} \and
%Tom\'{a}\v{s}~Vojnar\inst{1}}
  
%\institute{
%  FIT, Brno University of Technology, IT4Innovations Centre of Excellence, 
%  Czech Republic
%  \and 
%  University Grenoble Alpes, CNRS, VERIMAG, Grenoble, France
%}
  
% \maketitle
% \titlepage


\begin{abstract}
A sketch of a subset construction for transition systems with hidden variables.
\end{abstract}

%%%%%%%%%%%%%%%%%%%%%%%%%%%%%%%%%%%%%%%%%%%%%%%%%%%%%%%%%%%%%%%%%%%%%
%\section{Introduction}
%%%%%%%%%%%%%%%%%%%%%%%%%%%%%%%%%%%%%%%%%%%%%%%%%%%%%%%%%%%%%%%%%%%%%

%%%%%%%%%%%%%%%%%%%%%%%%%%%%%%%%%%%%%%%%%%%%%%%%%%%%%%%%%%%%%%%%%%%%%
\section{Transition Systems}
%%%%%%%%%%%%%%%%%%%%%%%%%%%%%%%%%%%%%%%%%%%%%%%%%%%%%%%%%%%%%%%%%%%%%

Vectors $x_1,x_2,\ldots,x_k,k\geq 0$ of indexed variables are denoted by $\vect{x}$ and we use $\vect{x}[k]$ to denote the particular vector $x_1,\ldots,x_k$.
%
We will sometimes abuse the notation and write $\vect{x}$ to interchangeably denote 
the vector and also the set $\{x_1,\ldots,x_k\}$ of variables appearing in it.
%
The primed variant $x_1',\ldots,x_k'$ of the vector $\vect{x}$ is denoted by $\vect{x}'$.
%
%Let $\nat$ denote the set of non-negative integers including zero. 
%
%We write $\false$ and $\true$ to denote the two boolean constants. 
%
Let $\theo$ denote the set of all syntactically correct formulae of some first order theory. 
%
A variable $x$
is said to be \emph{free} in a~formula $\phi\in\theo$
iff it does not occur under the scope of a~quantifier.
We write $\phi(\vect{x})$ to denote that the variables of $\vect{x}$ are free in $\phi$.
%
Given a vector $\vect \omega = \vect \omega[k]$ of formulae in $\theo$, 
we denote by $\subst{\phi}{\vect x}{\vect \omega}$ the formula that arises from $\phi$ by substituting all occurrences of $x_i,1\leq i \leq k$, by the formula $\omega_i$.  
%
%Let $\vect{x} = \vect{x}[k]$ and
%
Let $\thd$ be an interpretation of the first order theory $\theo$ with a possibly infinite domain $\domain$. 
%
A \emph{valuation} $\nu : \set{x} \rightarrow \domain$ is an
assignment of the variables in a vector $\vect{x}$ with values from
$\domain$. We denote by $\domain^{\set{x}}$ the set of such
valuations. 
%
For a formula $\phi(\vect{x})\in\theo$, we denote by $\nu
\modelsthd \phi$ that $\nu$ satisfies $\phi$ under the interpretation $\thd$. 

\paragraph{Transition system.} 
%
%Let $\vect{x} = \vect{x}[k]$ be a vector of visible and $\vect{y}=\vect{y}[\ell]$ a vector of invisible variables.
Let $\vect{x}$ be a vector of visible and $\vect{y}$ a vector of invisible variables.
%
A transition system is a triple $T = (\initial(\Xobs,\Xinv),\transition(\Xobs,\Xinv,\Xobs', \Xinv'),\accepting(\Xobs,\Xinv))$ where $\initial$ is the initial formula, 
$\transition(\Xobs,\Xinv,\Xobs', \Xinv')$ is the transition formula, 
and $\accepting(\Xobs,\Xinv)$ is the final formula.
%
We say that a valuation $\nu'$ is 
a~\emph{successor} of $\nu$ if and only if $(\nu\cup\nu') \modelsthd
\transition$. 
%denoted by $\nu
%\arrow{\transition}{} \nu'$. 
%, from where we omit the $\transition$ when no confusion may arise. 
%
%
A \emph{run} of $\transition$ is a sequence
of valuations $\pi : \nu_0, \ldots, \nu_n$ with $\nu_0\models\initial$ and $\nu_i$ a successor of $\nu_{i-1}$ for all $1\leq i \leq n$. 
It is \emph{accepting} if also $\nu_n\models\accepting$.
Each run corresponds to a \emph{trace}, which is a finite sequence
$w(\pi)=\nu_0\proj_{\Xobs}, \ldots,
\nu_n\proj_{\Xobs}$, where $\nu_i\proj_{\Xobs}$ denotes the restriction of $\nu_i$ to $\Xobs$.
%
The \emph{language} of $T$ is the set of accepted traces 
$\lang(T) = \{w(\pi)\mid \pi \text{ is an accepting run of } T\}$.


%%%%%%%%%%%%%%%%%%%%%%%%%%%%%%%%%%%%%%%%%%%%%%%%%%%%%%%%%%%%%%%%%%%%%
\section{Emptiness and Universality of Transition Systems}
%%%%%%%%%%%%%%%%%%%%%%%%%%%%%%%%%%%%%%%%%%%%%%%%%%%%%%%%%%%%%%%%%%%%%
For simplicity, let us assume that the transition system $T$ has only one visible variable $x$ and only one invisible variable $y$ (this is without loss of generality since $\domain$ may contain tuples).
A trace is then just a sequence $d_0,\ldots, d_k$ of elements of the domain.
%
We will elaborate on the problems of emptiness and universality of $T$, that is, the problems of deciding whether $\lang(T) = \emptyset$ and  whether $\lang(T) = \domain^*$.

For a formula $\phi(x,y,x',y')$ over variables $\vect{z}$ and their primed variants $\vect z'$, we write $\phi_i,i\geq 0$ to denote the formula $\subst{\phi}{x,y,x',y'}{x_i,y_i,x_{i+1},y_{i+1}}$ where the unprimed variables are indexed by $i$ and the primed by $i+1$.
%

All runs of the length $k\geq 0$ are described by the formula 
$$
\Run k(\vect x[k],\vect y[k]) \fdef
\initial_0(x_0,y_0) \land \bigwedge_{i=1}^{k} \tau_{i-1}(x_{i-1},y_{i-1},x_{i},y_{i})
$$
The language of $T$ would correspond to the union over all $k\geq 0$ of the satisfying assignments of formulae $\Language k$  that describe traces of the accepting runs of the length $k$:  
$$
\Language k(\vect x[k]) \fdef  \exists \vect y[k]:\Run k(\vect x[k],\vect y[k]) \land \final(x_k,y_k) \ \ .
$$
%
We have that $T$ is nonempty iff for some $k\geq 0$, $\Language k$ is satisfiable,
 and that $T$ is universal iff $\Language k$ is valid for all $k\geq 0$.

\newpage
%%%%%%%%%%%%%%%%%%%%%%%%%%%%%%%%%%%%%%%%%%%%%%%%%%%%%%%%%%%%%%%%%%%%%
\subsection{Emptiness}
%%%%%%%%%%%%%%%%%%%%%%%%%%%%%%%%%%%%%%%%%%%%%%%%%%%%%%%%%%%%%%%%%%%%%
As an exercise, before we elaborate on universality checking, we discuss language emptiness.  
Emptiness corresponds to unsatisfiability of the formula
$$
\Nonempty \fdef
\bigvee_{k\geq 0} \exists \vect x[k]:\Language k(\vect x[k])
=
\bigvee_{k\geq 0} \exists \vect x[k]\,\exists \vect y[k]:\Run k(\vect x[k],\vect y[k]) \land \final_k(x_{k},y_k)\ \ . 
$$
%
For any $0\geq k$, the $k$th disjunct is equivalent to 
$$
\exists xy: \subst{(\exists \vect x[k-1]\vect y[k-1]:\Run {k}(\vect x[k],\vect y[k]))}{x_k,y_k}{x,y}
\land 
\final(x,y)
$$
where the inner formula represents the set of states reached after $k$ steps.
Let us denote it 
$$\States k(x,y)\fdef \subst{(\exists \vect x[k-1]\vect y[k-1]:\Run {k}(\vect x[k],\vect y[k]))}{x_k,y_k}{x,y}\ \ .$$ $\Nonempty$ is therefore equivalent to
$\exists xy:\bigvee_{k\geq 0} \States k(x,y) \land \final(x,y)$. To test emptiness, it is enough to compute a formula representing states reachable at all distances and test existence of an accepting state.
To compute all reachable states, we use that 
$$
\States k \equiv \subst{(\exists xy:\States {k-1}(x,y)
\land 
\transition(x,y,x',y'))}{x',y'}{x,y}
\ \ .
$$
That is, reachable states at the distance $k$ can be computed as the image wrt. the transition relation of those reached at the distance $k-1$.  
%
So a formula that describes states including the initial ones and stable under the transition relation describes all reachable states. E.g., the formula
$$
\States {\leq n} \fdef \bigvee_{k=0}^n \States k
$$ 
such that 
$$
%\States {\leq n}  \equiv \States {\leq n} \lor \subst{(\exists x,y: \States {n}\land \transition)}{x',y'}{x,y}\ \ .
\States {\leq n}  \equiv \States {\leq n} \lor \States {n+1}\ \ .
$$

\newpage
%%%%%%%%%%%%%%%%%%%%%%%%%%%%%%%%%%%%%%%%%%%%%%%%%%%%%%%%%%%%%%%%%%%%%
\subsection{Universality}
%%%%%%%%%%%%%%%%%%%%%%%%%%%%%%%%%%%%%%%%%%%%%%%%%%%%%%%%%%%%%%%%%%%%%
We will now sketch a basic universality check. 
As in the case of emptiness, it is incomplete due to the infinite data domain, but additionally also due to the need of working with unbounded histories. 
%
Universality of $T$ corresponds to the validity of the formula
$$
\Universal \fdef
\bigwedge_{k\geq 0} \forall \vect x[k]:\Language k(\vect x[k])
=
\bigwedge_{k\geq 0} \forall \vect x[k]\,\exists \vect y[k]:\Run k(\vect x[k],\vect y[k]) \land \final_k(x_{k},y_k) 
$$
(the difference from the formula $\Nonempty$ is the leading $\bigwedge$ and the universal quantification over the visible trace). 
%
Let 
$$
\Configurations k(\vect x[k],x,y) \fdef \exists \vect y[k]:\Run k \land x = x_k \land y = y_k  
$$
be a formula which associates a visible trace $\vect x[k]$ of the length $k$ with the resulting values of $x$ and $y$.
Fixing a particular trace $\vect \dval [k]\in\domain^k$ results in a so called \emph{configuration}, 
a constraint over $x$ and $y$ (with $x$ fixed to the last value $\dval_k$ of the trace): 
$$
\subst{\Configurations k}{\vect x[k]}{\vect \dval[k]}(x,y)\ \ .
$$
One may hence understand $\Configurations k$ as a function $\lambda \vect x[k].\,\Configurations k$ from traces to configurations. 
%
%
We may rewrite the conjunction $\Universal$ as 
$$
\Universal 
\equiv
\bigwedge_{k\geq 0} \forall \vect x[k]\,\exists y:\Configurations k(\vect x[k],x,y) \land \final(x,y) 
$$
That is, every configuration must admit $y$ to satisfy $\final$.
Notice that this does not require to test all, but only the smallest (strongest) configurations.
%
The task is therefore to compute and test all smallest configurations $\subst{\Configurations {k}}{\vect x[k]}{\vect \dval[k]}$. 
To compute a representation of all smallest configurations, we use the fact that configurations at the distance $k+1$ can be obtained as successors of those at the distance $k$ as follows. 
For every visible trace $\vect \dval [k+1]$,
$$
\subst{\Configurations {k+1}}{\vect x[k+1]}{\vect \dval[k+1]} \equiv 
\subst{(\exists{xy}:\subst{\Configurations {k}}{\vect x[k]}{\vect \dval[k]} \land \tau)}{x',y'}{x,y}\land x = \dval_{k+1}\ \ .
$$
The set of all configurations can be hence computed as the least fixpoint of the transition relation that includes the initial configurations.
%
Notice that the smallest configurations at the distance $k+1$ are successors of the smallest configurations at the distance $k+1$.
%
That is, all smallest configurations are found within the distance $k$, i.e.
$$
\bigwedge_{m\geq 0} \forall \vect x[m]:\Configurations {m} \rightarrow \bigvee_{\ell\leq k} \exists \vect x[\ell]: \Configurations {\ell}
$$
%if and only if the distance $k+1$ does not hold any any new configurations, i.e 
if and only if the configurations found at the distance $k+1$ are only the same or larger, i.e.
$$
\forall \vect x[k+1]:\Configurations {k+1} \rightarrow \bigvee_{\ell\leq k} \exists \vect x[\ell]: \Configurations {\ell}\ \ .
$$



%%%%%%%%%%%%%%%%%%%%%%%%%%%%%%%%%%%%%%%%%%%%%%%%%%%%%%%%%%%%%%%%%%%%%
%\bibliographystyle{splncs03}
%\bibliography{ref}
%%%%%%%%%%%%%%%%%%%%%%%%%%%%%%%%%%%%%%%%%%%%%%%%%%%%%%%%%%%%%%%%%%%%%

%%%%%%%%%%%%%%%%%%%%%%%%%%%%%%%%%%%%%%%%%%%%%%%%%%%%%%%%%%%%%%%%%%%%%%%%%%%%%%%
\end{document}
%%%%%%%%%%%%%%%%%%%%%%%%%%%%%%%%%%%%%%%%%%%%%%%%%%%%%%%%%%%%%%%%%%%%%%%%%%%%%%%
