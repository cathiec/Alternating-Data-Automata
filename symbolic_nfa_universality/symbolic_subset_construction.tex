%\documentclass{llncs}
\documentclass[acmsmall]{acmart}

%\usepackage[usenames,dvipsnames]{color}

%\let\iint\undefined 
%\let\iiint\undefined 
%\let\iiiint\undefined 
%\let\idotsint\undefined

\usepackage{amsmath} 
\usepackage{amsxtra} 
\usepackage{amssymb}
%\usepackage{mathabx}
\usepackage{mathtools}
\usepackage{textcomp}
\usepackage{xifthen}
%\usepackage{silence}
%\WarningsOff

\usepackage{latexsym}
%\usepackage{pslatex}
%\usepackage{epsfig}
%\usepackage{wrapfig}
%\usepackage{graphics}
\usepackage{enumerate}
%\usepackage{cancel}

\usepackage{paralist}

%\usepackage{ listings }

%\usepackage{algorithm}
%\usepackage{algorithmicx}
%\usepackage[noend]{algpseudocode}
%\let\Asterisk\undefined

\usepackage{xparse}
\usepackage{pdfsync}

\pagestyle{empty}

%needed macros from the coopy-pasted preliminaries
%%%%%%%%%%%%%%%%%%%%%%%%%%%%%%%%%%%%%%%%%%%%%%%%%%%%%%%%%%%%%%%%%
\newcommand{\nat}{{\mathbb N}}
\newcommand{\domain}{\mathcal {D}}
%\newcommand{\set}[1]{\left\{ #1 \right\}}
\newcommand{\true}{\mathtt{true}}
\newcommand{\false}{\mathtt{false}}
\newcommand{\tuple}[1]{\left\langle #1 \right\rangle}
\newcommand{\theo}{\mathsf{Th}}
\newcommand{\thd}{\theo(\mathcal{D})}
\newcommand{\modelsthd}{\models_{\thd}}
\def\proj{\mathbin{\downarrow}}

\DeclareDocumentCommand\vect{ m o o }{%
    {% 
        \IfNoValueT {#2} {\bar #1} 
        \IfNoValueF {#2} {%
			\IfNoValueT	{#3} {\bar #1_{#2}}%
			%\IfNoValueF {#3} {^{(#2,#3)}}%
			\IfNoValueF {#3} {\bar #1^{(#2,#3)}}%
		}%
    }%
}

\newcommand{\set}{\vect}
%\DeclareDocumentCommand\set{ g g }{%
%    {%	
%		\IfNoValueT {#2} {\{\bar #1\}}%
%       	\IfNoValueF {#2} {\{\bar #1_{(#2)}\}}%
%    }%
%}

\newcommand{\subst}[3]{#1[#2/#3]}%substitute in #1 occurences of #2 by #3
\newcommand{\Nonempty}{\mathit{Nonemp}}
\newcommand{\Universal}{\mathit{Univ}}
\newcommand{\Nonuniversal}{\mathit{Nonuniv}}
%\newcommand{\Empty}{\mathit{Emp}}
\newcommand{\Language}[1]{\mathit{Lang_{#1}}}
\newcommand{\Path}[1]{\mathit{Path_{#1}}}
\newcommand{\Run}[1]{\mathit{Run_{#1}}}
\newcommand{\Prefix}[1]{\mathit{Pref_{#1}}}
\newcommand{\Suffix}[1]{\mathit{Suff_{#1}}}
\newcommand{\States}[1]{\mathit{States_{#1}}}
\newcommand{\Configurations}[1]{\mathit{Confs_{#1}}}
\newcommand{\Configuration}[1]{\mathit{Conf_{#1}}}

%\newcommand{\arrow}[2]{\xrightarrow{_{_{#1}}}_{{\scriptstyle #2}}}
\newcommand{\arrow}[2]{\xrightarrow{{\scriptstyle #1}}_{{\scriptstyle #2}}}
%\newcommand{\arrow}[2]{\rightarrow_{{#1}}^{{\scriptstyle #2}}}
\def\projobs{\mathbin{\downarrow_{\Xobs}}}

\newcommand{\viz}{x}
\newcommand{\inviz}{y}
\newcommand{\lang}{\mathcal {L}}
\newcommand{\dvalue}{v}
\newcommand{\dval}{\dvalue}

\newcommand{\Xobs}{\vect{x}}
\newcommand{\Xinv}{\vect{y}}
\newcommand{\Xinvz}{\vect{z}}

\newcommand{\vx}{\vect{x}}
\newcommand{\vy}{\vect{y}}
\newcommand{\vz}{\vect{z}}

\newcommand{\initial}{I}
\newcommand{\final}{F}
\newcommand{\accepting}{F}
\newcommand{\transition}{\tau}

\newcommand{\fdef}{\mathrel{:=}}
%%%%%%%%%%%%%%%%%%%%%%%%%%%%%%%%%%%%%%%%%%%%%%%%%%%%%%%%%%%%%%%%%


%\include{commands}


%%%%%%%%%%%%%%%%%%%%%%%%%%%%%%%%%%%%%%%%%%%%%%%%%%%%%%%%%%%%%%%%%%%%%%%%%%%%%%%
\begin{document}
%%%%%%%%%%%%%%%%%%%%%%%%%%%%%%%%%%%%%%%%%%%%%%%%%%%%%%%%%%%%%%%%%%%%%%%%%%%%%%%

%\title{Predicate Abstraction for Trace Inclusion}

%\author{Luk\'{a}\v{s} Hol\'{i}k\inst{1} \and Radu Iosif\inst{2} \and Adam Rogalewicz\inst{1} \and
%Tom\'{a}\v{s}~Vojnar\inst{1}}
  
%\institute{
%  FIT, Brno University of Technology, IT4Innovations Centre of Excellence, 
%  Czech Republic
%  \and 
%  University Grenoble Alpes, CNRS, VERIMAG, Grenoble, France
%}
  
% \maketitle
% \titlepage


\begin{abstract}
A sketch of a subset construction for transition systems with hidden variables.
\end{abstract}

%%%%%%%%%%%%%%%%%%%%%%%%%%%%%%%%%%%%%%%%%%%%%%%%%%%%%%%%%%%%%%%%%%%%%
%\section{Introduction}
%%%%%%%%%%%%%%%%%%%%%%%%%%%%%%%%%%%%%%%%%%%%%%%%%%%%%%%%%%%%%%%%%%%%%

%%%%%%%%%%%%%%%%%%%%%%%%%%%%%%%%%%%%%%%%%%%%%%%%%%%%%%%%%%%%%%%%%%%%%
\section{Transition Systems}
%%%%%%%%%%%%%%%%%%%%%%%%%%%%%%%%%%%%%%%%%%%%%%%%%%%%%%%%%%%%%%%%%%%%%

Vectors $x_1,\ldots,x_k,k\in\nat$ of indexed variables are denoted by $\vect{x}$ and we use $\vect{x}[k]$ to denote the particular vector $x_1,\ldots,x_k$.
%
We will sometimes abuse the notation and write $\vect{x}$ to interchangeably denote 
the vector and also the set $\{x_1,\ldots,x_k\}$ of variables appearing in it.
%
The primed variant $x_1',\ldots,x_k'$ of the vector $\vect{x}$ is denoted by $\vect{x}'$.
%
Let $\nat$ denote the set of non-negative integers including zero. 
%
We write $\false$ and $\true$ to denote the two boolean constants. 
%
Let $\theo$ denote the set of all syntactically correct formulae of some first order theory. 
%
A variable $x$
is said to be \emph{free} in a~formula $\phi\in\theo$
iff it does not occur under the scope of a~quantifier.
We write $\phi(\vect{x})$ to denote that the variables of $\vect{x}$ are free in $\phi$.
%
Given a vector $\vect \omega = \vect \omega[k]$ of formulae in $\theo$, 
we denote by $\subst{\phi}{\vect x}{\vect \omega}$ the formula that arises from $\phi$ by substituting all occurrences of $x_i,1\leq i \leq k$, by the formula $\omega_i$.  
%


%Let $\vect{x} = \vect{x}[k]$ and
%
Let $\thd$ be an interpretation of the first order theory $\theo$ with a possibly infinite domain $\domain$. 
%
A \emph{valuation} $\nu : \set{x} \rightarrow \domain$ is an
assignment of the variables in a vector $\vect{x}$ with values from
$\domain$. We denote by $\domain^{\set{x}}$ the set of such
valuations. 
%
For a formula $\phi(\vect{x})\in\theo$, we denote by $\nu
\modelsthd \phi$ that $\nu$ satisfies $\phi$ under the interpretation $\thd$. 

\paragraph{Transition system.} 
%
%Let $\vect{x} = \vect{x}[k]$ be a vector of visible and $\vect{y}=\vect{y}[\ell]$ a vector of invisible variables.
Let $\vect{x}$ be a vector of visible and $\vect{y}$ a vector of invisible variables.
%
A transition system is a triple $T = (\initial(\Xobs,\Xinv),\transition(\Xobs,\Xinv,\Xobs', \Xinv'),\accepting(\Xobs,\Xinv))$ where $\initial$ is the initial formula, 
$\transition(\Xobs,\Xinv,\Xobs', \Xinv')$ is the transition formula, 
and $\accepting(\Xobs,\Xinv)$ is the final formula.
%
We say that a valuation $\nu'$ is 
a~\emph{successor} of $\nu$ if and only if and $(\nu\cup\nu') \modelsthd
\transition$, denoted by $\nu
\arrow{\transition}{} \nu'$. 
%, from where we omit the $\transition$ when no confusion may arise. 


A \emph{run} of $\transition$ is a sequence
of valuations $\pi : \nu_0 \arrow{\transition}{}  \cdots \arrow{\transition}{} \nu_n$ with $\nu_0\models\initial$. 
It is accepting if also $\nu_n\models\accepting$.
Each run corresponds to a \emph{trace}, which is a finite sequence
$w(\pi)=\nu_0\proj_{\Xobs}, \ldots,
\nu_n\proj_{\Xobs}$, where $\nu_i\proj_{\Xobs}{}$ denotes the restriction of $\nu_i$ to $\Xobs$.
%
The \emph{language} of $T$ is the set of accepted traces 
$\lang(T) = \{w(\pi)\mid \pi \text{ is an accepting run of } T\}$.


%%%%%%%%%%%%%%%%%%%%%%%%%%%%%%%%%%%%%%%%%%%%%%%%%%%%%%%%%%%%%%%%%%%%%
\section{Emptiness and Universality of Transition Systems}
%%%%%%%%%%%%%%%%%%%%%%%%%%%%%%%%%%%%%%%%%%%%%%%%%%%%%%%%%%%%%%%%%%%%%
For simplicity, let us assume wlog. a transition system $T$ with only one visible variable $x$ and only one invisible variable $y$.
A trace is then just a sequence $d_0,\ldots, d_k$ of elements of the domain.
%
We will elaborate on the problems of emptiness and universality of $T$, that is, the problems of deciding whether $\lang(T) = \emptyset$ and  whether $\lang(T) = \domain^*$.

For a formula $\phi(x,y,x',y')$ over variables $\vect{z}$ and their primed variants $\vect z'$, we write $\phi_i,i\geq 0$ to denote the formula $\subst{\phi}{x,y,x',y'}{x_i,y_i,x_{i+1},y_{i+1}}$ where the unprimed variables are indexed by $i$ and the primed by $i+1$.
%

All runs of the length $k\geq 0$ are described by the formula 
$$
\Run k(\vect x[k],\vect y[k]) \fdef
\initial_0(x_0,y_0) \land \bigwedge_{i=1}^{k} \tau_{i-1}(x_{i-1},y_{i-1},x_{i},y_{i})
$$
The language of $T$ would correspond to the union over all $k\geq 0$ of the satisfying assignments of formulae $\Language k$  that describe traces of the accepting runs of the length $k$:  
$$
\Language k(\vect x[k]) \fdef  \exists \vect y[k]:\Run k(\vect x[k],\vect y[k]) \land \final(x_k,y_k) \ \ .
$$
%
We have that $T$ is nonempty iff for some $k\geq 0$, $\Language k$ is satisfiable,
 and that $T$ is universal iff $\Language k$ is valid for all $k\geq 0$.

\subsection{Emptiness}
As an exercise, before we elaborate on universality checking, we discuss language emptiness.  
Emptiness corresponds to unsatisfiability of the formula
$$
\Nonempty \fdef
\bigvee_{k\geq 0} \exists \vect x[k]:\Language k(\vect x[k])
=
\bigvee_{k\geq 0} \exists \vect x[k]\,\exists \vect y[k]:\Run k(\vect x[k],\vect y[k]) \land \final_k(x_{k},y_k)\ \ . 
$$
%
For any $0\geq k$, the $k$th disjunct is equivalent to 
$$
\exists xy: \subst{(\exists \vect x[k-1]\vect y[k-1]:\Run {k}(\vect x[k],\vect y[k]))}{x_k,y_k}{x,y}
\land 
\final(x,y)
$$
where the inner formula represents the set of states reached after $k$ steps.
Let us denote it $\States k(x,y)$. $\Nonempty$ is therefore equivalent to
$\exists xy:\bigvee_{k\geq 0} \States k(x,y) \land \final(x,y)$, that is, it is enough to compute a formula representing states reachable at all distances and test existence of an accepting one.
To compute all reachable states, we use that 
$$
\States k \equiv \subst{(\exists xy:\States {k-1}(x,y)
\land 
\transition(x,y,x',y'))}{x',y'}{x,y}
\ \ .
$$
That is, reachable states at the distance $k$ can be computed from those reached at the distance $k-1$ by applying the transition relation.  
%
So it is enough to compute a formula that describes reached states that is stable under the transition relation:
$$\States {\leq n} \fdef \bigvee_{k=0}^n \States k$$ 
such that 
$$\States {\leq n}  \equiv \States {\leq n} \lor \subst{(\exists x,y: \States {n}\land \transition)}{x',y'}{x,y}\ \ .$$

\newpage
\subsection{Universality}
We will snow ketch a basic universality check. 
As in the case of emptiness, it is incomplete due to the infinite data domain, but additionally also due to the need of working with unbounded histories. 
%
Universality of $T$ corresponds to the validity the formula
$$
\Universal \fdef
\bigwedge_{k\geq 0} \forall \vect x[k]:\Language k(\vect x[k])
=
\bigwedge_{k\geq 0} \forall \vect x[k]\,\exists \vect y[k]:\Run k(\vect x[k],\vect y[k]) \land \final_k(x_{k},y_k) 
$$
(the difference from the formula $\Nonempty$ is the leading $\bigwedge$ and the universal quantification over the visible variables). 
%
Let 
$$
\Configurations k(\vect x[k],x,y) \fdef \exists \vect y[k]:\Run k \land x = x_k \land y = y_k  
$$
be a formula which associates a visible trace $\vect x[k]$ of the length $k$ with the resulting values of $x$ and $y$.
Fixing a particular trace $\vect \dval [k]$ results in a so called \emph{configuration}, 
a constraint over $x$ and $y$ with $x$ fixed at the last value $\dval_k$: 
$$
\subst{\Configurations k}{\vect x[k]}{\vect \dval[k]}(x,y)\ \ .
$$
One may hence understand $\Configurations k$ as a function $\lambda \vect x[k].\,\Configurations k$ from traces to configurations. 
%
%
We may rewrite conjunction $\Universal$ as 
$$
\Universal 
\equiv
\bigwedge_{k\geq 0} \forall \vect x[k]\,\exists y:\Configurations k(\vect x[k],x,y) \land \final(x,y) 
$$
That is, every configuration must have $y$ to satisfy $\final$.
Notice that the universality test does not require all but only the smallest (strongest) configurations.

The task is therefore to compute and test all configurations $\subst{\Configurations {k}}{\vect x[k]}{\vect \dval[k]}$. 
To compute a representation of all configurations, we use the fact that configurations at the distance $k+1$ can be obtained as successors of those at the distance $k$ as follows. 
For every visible trace $\vect \dval [k+1]$,
$$
\subst{\Configurations {k+1}}{\vect x[k+1]}{\vect \dval[k+1]} \equiv 
\subst{(\exists{xy}:\subst{\Configurations {k}}{\vect x[k]}{\vect \dval[k]} \land \tau)}{x',y'}{x,y}\land x = \dval_{k+1}\ \ .
$$
The set of all configurations can be hence computed as the least fixpoint of the transition relation that includes the initial configurations.
%
Notice that the lowest configurations at the distance $k+1$ are successors of the lowest configurations at the distance $k+1$.
%
That is, all smallest configurations are found within the distance $k$, i.e.
$$
\bigwedge_{\ell\geq 0} \forall \vect x[\ell]:\Configurations {\ell} \rightarrow \bigvee_{m\leq k} \exists \vect x[m]: \Configurations {m}
$$
%if and only if the distance $k+1$ does not hold any any new configurations, i.e 
if and only if there are no new or smaller configurations at the distance $k+1$, i.e 
$$
\forall \vect x[k+1]:\Configurations {k+1} \rightarrow \bigvee_{\ell\leq k} \exists \vect x[\ell]: \Configurations {\ell}\ \ .
$$



%%%%%%%%%%%%%%%%%%%%%%%%%%%%%%%%%%%%%%%%%%%%%%%%%%%%%%%%%%%%%%%%%%%%%
%\bibliographystyle{splncs03}
%\bibliography{ref}
%%%%%%%%%%%%%%%%%%%%%%%%%%%%%%%%%%%%%%%%%%%%%%%%%%%%%%%%%%%%%%%%%%%%%

%%%%%%%%%%%%%%%%%%%%%%%%%%%%%%%%%%%%%%%%%%%%%%%%%%%%%%%%%%%%%%%%%%%%%%%%%%%%%%%
\end{document}
%%%%%%%%%%%%%%%%%%%%%%%%%%%%%%%%%%%%%%%%%%%%%%%%%%%%%%%%%%%%%%%%%%%%%%%%%%%%%%%
