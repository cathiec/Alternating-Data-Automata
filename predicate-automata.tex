\documentclass{llncs}

\usepackage[usenames,dvipsnames]{color}

\usepackage{latexsym}
\usepackage{amsxtra} 
\usepackage{amssymb}
\usepackage{amsmath}
\usepackage{pslatex}
% \usepackage{epsfig}
\usepackage{wrapfig}
\usepackage{paralist}
\usepackage{stmaryrd}
\usepackage{txfonts}
\usepackage{framed}
\usepackage{makecell}
\usepackage{url}
\usepackage{tikz}
\usetikzlibrary{automata,positioning, calc}
\usepackage[inline,shortlabels]{enumitem}

\usepackage{proof}

\usepackage{algorithm}
\usepackage{algorithmicx}
\usepackage[noend]{algpseudocode}

\usepackage[draft]{commenting}

\pagestyle{plain}

\include{commands-lncs}

%%%%%%%%%%%%%%%%%%%%%%%%%%%%%%%%%%%%%%%%%%%%%%%%%%%%%%%%%%%%%%%%%%%%%%%%%%%%%%%
\begin{document}
%%%%%%%%%%%%%%%%%%%%%%%%%%%%%%%%%%%%%%%%%%%%%%%%%%%%%%%%%%%%%%%%%%%%%%%%%%%%%%%

\title{From Predicate Automata to Alternating Data Automata}

\author{Radu Iosif}
\institute{CNRS, Verimag \\
Radu.Iosif@univ-grenoble-alpes.fr}
\maketitle

\section{Definitions}

Let $Q$ be a finite set of predicates, where for each $q \in Q$, we
denote by $\#(q)\geq0$ its arity. We denote by $Q_0$ the subset of $Q$
consisting of predicates of arity zero. We consider an infinite
countable set $\vars$ of first-order variables, ranging over
$\nat$. For a tuple of variables $\vec{x} = \tuple{x_1,\ldots,x_n}$,
we define a one-to-one corresponding tuple $\overline{\vec{x}} =
\tuple{\overline{x_1},\ldots,\vec{x_n}}$. We associate each predicate
$q \in Q$ a tuple of variables $\vec{x}_q = \tuple{x_{q,1}, \ldots,
  x_{q,\#(q)}}$, called \emph{arguments}, such that, for any two $p,q
\in Q$, if $p \neq q$ then any two of $\vec{x}_p$,
$\overline{\vec{x}}_p$, $\vec{x}_q$ and $\overline{\vec{x}}_q$ have no
elements in common, respectively.

An \emph{atom} is a predicate $q(\vec{x}_q)$,
$q(\overline{\vec{x}}_q)$, $x \teq y$ or $x \not\teq y$, for $x,y \in
\vars$ and a \emph{formula} is a positive boolean combination of
atoms. We denote by $\Form(Q,\vec{x})$ the set of formulae with
predicates from $Q$ and free variables from $\vec{x}$. A \emph{ground
  formula} is a formula in which each predicate occurs as
$q(n_1,\ldots,n_{\#(q)})$, for some $n_1,\ldots,n_{\#(q)} \in \nat$. 

\subsection{Predicate Automata}

We denote by $\varepsilon$ the empty sequence and by $\Sigma^*$ the
set of finite sequences of input symbols.

\begin{definition}
  A predicate automaton (PA) is a tuple $\A =
  \tuple{\Sigma,Q,\iota,F,\Delta}$, where: \begin{compactitem}
  \item $\Sigma$ is a finite input alphabet,
  \item $Q$ is the set of predicates,
  \item $\iota \in \Form(Q,\emptyset)$ is a ground formula describing
    the initial configurations,
  \item $F \subseteq Q$ is a set of accepting predicates,
  \item $\Delta : Q \times \Sigma \rightarrow
    \Form(Q,\vec{x}\cup\overline{\vec{x}})$ is a transition function.
  \end{compactitem}
\end{definition}
We write transition rules as follows: 
\[q(\vec{x}_q) \arrow{\alpha,x_0}{} \Delta(q,\alpha) 
\text{, where $\fv{}{\Delta(q,\alpha)} \subseteq \bigcup \{\vec{x}_p
  \mid p \text{ occurs in } \Delta(q,\alpha)\} \cup
  \overline{\vec{x}}_q \cup \set{x_0}$}\] Intuitively, $\vec{x}_p$
denotes the current values of the arguments $\vec{x}_p$ of $p$, where
$p$ is any predicate that occurs in $\Delta(q,\alpha)$, $x_0$ is the
input value associated with the symbol $\alpha$ read by the automaton
and $\overline{\vec{x}}_q$ are the past values of the arguments of
$q$. We denote by $\Delta(\phi,\alpha)$ the formula obtained from
$\phi$ by replacing each occurrence of $q(\vec{x}_q)$ in $\phi$ by
$\Delta(q,\alpha)$.

\begin{example}
For example, instead of \(q(i,j) \arrow{\alpha,k}{} i = k \wedge j
\neq k \wedge p(k,j) \wedge q(i,k)\), we write: \[q(x_{q,1},x_{q,2})
\arrow{\alpha,x_0}{} \overline{x}_{q,1} = x_0 \wedge
\overline{x}_{q,2} \neq x_0 \wedge x_{p,1} = x_0 \wedge x_{p,2} =
\overline{x}_{q,2} \wedge p(x_{p,1},x_{p,2}) \wedge x_{q,1} =
\overline{x}_{q,1} \wedge x_{q,2} = x_0 \wedge q(x_{q,1},x_{q,2})\] We
assume that this preprocessing is already done for PA. \hfill\qed
\end{example}

A \emph{configuration} of $\A$ is a ground formula. We abuse notation
and blur the distinction between the terms denoting natural numbers
and their interpretations. The transition relation is a subset of
$\Form(Q,\emptyset) \times \Sigma \times \Form(Q,\emptyset)$ denoted
as \(C \stackrel{\alpha,n}{\Longrightarrow} C'\),
where: \begin{compactitem}
\item $C = \phi\sigma$, for a formula $\phi \in \Form(Q,\vec{x})$ and
  a substitution $\sigma : \vec{x} \rightarrow \nat$, and
%
\item $C' = \Delta(q,\alpha)\sigma'$, where $\sigma' :
  \fv{}{\Delta(q,\alpha)} \rightarrow \nat$ is a substitution defined
  as follows:
  \[\sigma'(y) = \left\{\begin{array}{ll}
  \sigma(x) & \text{if $y = \overline{x}$} \\
  n & \text{if $y = x_0$} \\
  \text{arbitrary $m \in \nat$} & \text{otherwise}
  \end{array}\right.\]
\end{compactitem}
Observe that, because of the arbitrary choices of current values for
the arguments, a configuration can have infinitely many outgoing
transitions. However, since we identify configurations that are
logically equivalent, a false configuration is a sink with only false
successors, for any pair $(\alpha,n) \in \Sigma \times \nat$. 

A word $w = (\alpha_1,n_1) \ldots (\alpha_k,n_k) \in (\Sigma \times
\nat)^*$ is \emph{accepted} by $\A$ iff there exists a sequence of
configurations $C_k, \ldots, C_0$ such that $C_k = \iota$, $C_{i+1}
\stackrel{\alpha_i,n_i}{\Longrightarrow} C_i$ for all $i \in [0,n-1]$
and $C_0 \in \Form(F,\emptyset)$ contains only accepting
predicates. The language of $\A$ is the set $L(\A)$ of accepted words.

\subsection{Alternating Data Automata}

Given a finite set $\vec{x} \subset \vars$ of variables, let $\vec{x}
\mapsto \nat$ be the set of valuations of the variables $\vec{x}$ and
$\Sigma[\vec{x}] = \Sigma \times (\vec{x} \mapsto \nat)$ be the set of
\emph{data symbols}. A \emph{data word} (word in the sequel) is a
finite sequence $(a_1,\nu_1)(a_2,\nu_2) \ldots (a_n,\nu_n)$ of data
symbols, where $a_1,\ldots,a_n \in \Sigma$ and $\nu_1,\ldots,\nu_n :
\vec{x} \rightarrow \nat$ are valuations. We denote by
$\Sigma[\vec{x}]^*$ the set of data words over $\vec{x}$.

\begin{definition}
  An alternating data automaton (ADA) is a tuple $\A =
  \tuple{\Sigma,\vec{x},Q_0,\iota,F,\Delta}$,
  where: \begin{compactitem}
    \item $\Sigma$ is a finite input alphabet,
    \item $\vec{x}$ is a finite set of variables, ranging over $\nat$,
    \item $Q_0$ is a set of predicates of arity zero, called \emph{states}
    \item $\iota \in \Form(Q_0,\emptyset)$ is a boolean combination of states,
    \item $F \subseteq Q_0$ is a set of final states,
    \item $\Delta : Q_0 \times \Sigma \rightarrow
      \Form(Q_0,\vec{x}\cup\overline{\vec{x}})$ is a transition function.
  \end{compactitem}
\end{definition}
In each formula $\Delta(q,a)$ describing a transition rule, the
variables $\overline{\vec{x}}$ track the previous and $\vec{x}$ the
current values of the variables of $\A$. Observe that the initial
values of the variables are left unconstrained, as the initial
configuration does not contain free data variables.

Let $\vec{x}_k = \set{x_k \mid x \in \vec{x}}$, for any $k\geq0$, be a
set of time-stamped variables. For an input symbol $\alpha \in \Sigma$
and a formula $\phi \in \Form(Q_0,\vec{x}\cup\overline{\vec{x}})$, we
write $\Delta^k(\phi,a)$ for the formula obtained from $\phi$ by
simultaneously replacing each $q \in Q_0$ occurring in $\phi$ by the
formula
$\Delta(q,a)[\vec{x}_k/\overline{\vec{x}},\vec{x}_{k+1}/\vec{x}]$,
$k\geq0$. Given a word $w = (a_1,\nu_1)(a_2,\nu_2) \ldots
(a_n,\nu_n)$, the \emph{run} of $\A$ over $w$ is the sequence of
formulae: \[\phi_0(Q) \Rightarrow \phi_1(Q,\vec{x}_0 \cup \vec{x}_1)
\Rightarrow \ldots \Rightarrow \phi_n(Q,\vec{x}_0 \cup \ldots \cup
\vec{x}_n)\] where $\phi_0 \equiv \iota$ and, for all $k\in[1,n]$, we
have $\phi_k \equiv \Delta^k(\phi_{k-1},a_k)$. Next, we slightly abuse
notation and write $\Delta(\iota,a_1,\ldots,a_n)$ for the formula
$\phi_n(\vec{x}_0,\ldots,\vec{x}_n)$ above. We say that $\A$
\emph{accepts} $w$ iff $\I,\nu \models \Delta(\iota,a_1,\ldots,a_n)$,
for some valuation $\nu$ that maps:\begin{compactitem}
%
\item each $x \in \vec{x}_k$ to $\nu_k(x)$, for all $k\in[1,n]$, 
%
\item each $q \in F$ to $\top$ and 
%
\item each $q \in Q_0 \setminus F$ to $\bot$.
\end{compactitem}
The language of $\A$ is the set $L(\A)$ of words from
$\Sigma[\vec{x}]^*$ accepted by $\A$.

\section{Translation of PA into ADA}

Given a predicate automaton $\A_\PA = \tuple{\Sigma, Q_\PA, \iota_\PA,
  F_\PA, \Delta_\PA}$, we define an alternating data automaton
$\A_\ADA = \tuple{\Sigma, \vec{x}_\ADA, Q_\ADA, \iota_\ADA, F_\ADA,
  \Delta_\ADA}$ as follows: \begin{compactitem}
\item $\vec{x}_\ADA = \set{x \in \vars \mid x \in \fv{}{\Delta_\PA(q,\alpha)}, q \in Q_\PA, \alpha \in \Sigma}$, 
%
\item $Q_\ADA = \set{\overline{q} \mid q \in Q_\PA}$, where $\#(\overline{q}) = 0$, for all $\overline{q} \in Q_\ADA$,
%
\item $\iota_\ADA$ is obtained from $\iota_\PA$ by replacing each
  occurrence of a ground term $q(n_1,\ldots,n_{\#(q)})$ from
  $\iota_\PA$ with $\overline{q}$,
%
\item $F_\ADA = \set{\overline{q} \mid q \in F_\PA}$, and 
%
\item for each transition rule \(q(k_1,\ldots,k_{\#(q)}) \arrow{\alpha,x_0}{}
  \Delta_\PA(q,\alpha)\) \begin{compactitem}
  \item if \(k_1,\ldots,k_q\) is the tuple of variables $\vec{x}_q$
    then \(\Delta_\ADA(q,\alpha)\) is obtained from
    \(\Delta_\PA(q,\alpha)\) by replacing each occurrence of a
    predicate $q(\vec{x}_q)$ with $\overline{q}$, and
  %
\item else, if \(k_1,\ldots,k_{\#(q)} \in \nat\), we add the
  conjunction $\bigwedge_{i=1}^{\#(q)} \overline{x}_i = k_i$ to
  \(\Delta_\ADA(q,\alpha)\).
  \end{compactitem}
  Nothing else is in $\Delta_\ADA$. 
\end{compactitem}

\begin{example}
The PA transition rule: 
\[q(x_{q,1},x_{q,2})
\arrow{\alpha,x_0}{} \overline{x}_{q,1} = x_0 \wedge
\overline{x}_{q,2} \neq x_0 \wedge x_{p,1} = x_0 \wedge x_{p,2} =
\overline{x}_{q,2} \wedge p(x_{p,1},x_{p,2}) \wedge x_{q,1} =
\overline{x}_{q,1} \wedge x_{q,2} = x_0 \wedge q(x_{q,1},x_{q,2})\]
becomes the following ADA transition rule:
\[\overline{q} \arrow{\alpha}{} \overline{x}_{q,1} = x_0 \wedge
\overline{x}_{q,2} \neq x_0 \wedge x_{p,1} = x_0 \wedge x_{p,2} =
\overline{x}_{q,2} \wedge \overline{p} \wedge x_{q,1} =
\overline{x}_{q,1} \wedge x_{q,2} = x_0 \wedge \overline{q}\]
\end{example}

\begin{claim}
  $L(\A_\PA) \neq \emptyset$ if and only if $L(\A_\ADA) \neq \emptyset$.
\end{claim}

%%%%%%%%%%%%%%%%%%%%%%%%%%%%%%%%%%%%%%%%%%%%%%%%%%%%%%%%%%%%%%%%%%%%%%%%%%%%%%%
\end{document}
%%%%%%%%%%%%%%%%%%%%%%%%%%%%%%%%%%%%%%%%%%%%%%%%%%%%%%%%%%%%%%%%%%%%%%%%%%%%%%%



