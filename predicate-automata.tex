\documentclass{llncs}

\usepackage[usenames,dvipsnames]{color}

\usepackage{latexsym}
\usepackage{amsxtra} 
\usepackage{amssymb}
\usepackage{amsmath}
\usepackage{pslatex}
% \usepackage{epsfig}
\usepackage{wrapfig}
\usepackage{paralist}
\usepackage{stmaryrd}
\usepackage{txfonts}
\usepackage{framed}
\usepackage{makecell}
\usepackage{url}
\usepackage{tikz}
\usetikzlibrary{automata,positioning, calc}
\usepackage[inline,shortlabels]{enumitem}

\usepackage{proof}

\usepackage{algorithm}
\usepackage{algorithmicx}
\usepackage[noend]{algpseudocode}

\usepackage[draft]{commenting}

\pagestyle{plain}

%\newtheorem{definition}{Definition}
%\newtheorem{lemma}{Lemma}
%\newtheorem{theorem}{Theorem}
%\newtheorem{proposition}{Proposition}
%\newtheorem{corollary}{Corollary}
%\newtheorem{conjecture}{Conjecture}
%\newtheorem{claim}{Claim}
%\newtheorem{property}{Property}
\newtheorem{fact}{Fact}

\newcommand*\circled[1]{\tikz[baseline=(char.base)]{
            \node[shape=circle,draw,inner sep=0.5pt] (char) {#1};}}

% Marius commands
\newcommand{\Z}{{\mathbb Z}}
\newcommand{\N}{{\mathbb N}}
\newcommand{\relmat}[1] { \{\!\!\{#1\}\!\!\}}
\newcommand{\matrel}[1] { [\!\![#1]\!\!]}
\newcommand{\key}[1]{\underline{#1}}

% complexity 
\newcommand{\np}{$\mathsf{NP}$}
\newcommand{\pspace}{$\mathsf{PSPACE}$~}
\newcommand{\sigmatwop}{$\mathsf{\Sigma}_2^p$}
\newcommand{\pitwop}{$\mathsf{\Pi}_2^p$}

% semantic stuff
\newcommand{\bi}[1]{{}^\omega\!{#1}^\omega}
\newcommand{\rbr}{{\bf ]\!]}}
\newcommand{\lbr}{{\bf [\![}}
\newcommand{\sem}[1]{\lbr #1 \rbr}
\newcommand{\model}[2]{\sem{#1}_{#2}}
\newcommand{\lfp}[1]{\mbox{fix}\ {#1}}
\newcommand{\wpr}{\widetilde{pre}}
\newcommand{\cceil}[1]{\lfloor #1 \rfloor + 1}
\newcommand{\set}[1]{\{ #1 \}}
\newcommand{\tuple}[1]{\langle #1 \rangle}
\renewcommand{\vec}[1]{\mathbf #1}
\newcommand{\tup}[1]{\overline #1}
\newcommand{\ite}[3]{\mathsf{ite}(#1,#2,#3)}
\newcommand{\tupite}[2]{\mathsf{ite}(#1,#2)}
\newcommand{\minsem}[1]{\sem{#1}^\mu}

% alias
\newcommand{\alias}{\Diamond}
\newcommand{\may}[2]{{\langle #1 \rangle}{#2}}
\newcommand{\must}[2]{{\lbrack #1 \rbrack}{#2}}
\newcommand{\defd}[1]{\delta({#1})}

% definitions
\newcommand{\isdef}{\stackrel{\scriptscriptstyle{\mathsf{def}}}{=}}
\renewcommand{\iff}{\Leftrightarrow}

% algorithms
\newcommand{\sig}{\mathsf{S}}
\newcommand{\stamp}[2]{{#1}^{\scriptscriptstyle{({#2})}}}
\newcommand{\overstamp}[2]{{#1}^{\scriptscriptstyle{\langle{#2}\rangle}}}

% recursive structures
\newcommand{\LS}{\mathtt{LS}}
\newcommand{\DLL}{\mathtt{DLL}}
\newcommand{\TLL}{\mathtt{TLL}}
\newcommand{\TREE}{\mathtt{TREE}}
\newcommand{\ax}{\mathtt{a}}
\newcommand{\bx}{\mathtt{b}}
\newcommand{\cx}{\mathtt{c}}
\newcommand{\dx}{\mathtt{d}}

% cardinal, degree, ...
\newcommand{\sgn}{\mbox{sgn}}
%\newcommand{\gcd}{\mbox{gcd}}
\newcommand{\lcm}{\mbox{lcm}}
\newcommand{\len}[1]{{|{#1}|}}
\newcommand{\card}[1]{{|\!|{#1}|\!|}}
\newcommand{\minimum}[2]{\mbox{min}({#1},{#2})}
\newcommand{\maximum}[2]{\mbox{max}({#1},{#2})}
\newcommand{\minset}[2]{\mbox{min}\{{#1}\ |\ {#2}\}}
\newcommand{\maxset}[2]{\mbox{max}\{{#1}\ |\ {#2}\}}
\newcommand{\divs}[1]{\mbox{div}({#1})}

% automata and languages
\newcommand{\auth}[1]{{\mathcal A}_{#1}}
\newcommand{\lang}[1]{{\mathcal L}({#1})}
\newcommand{\runs}[1]{{\mathcal R}({#1})}
\newcommand{\val}[1]{{\mathcal V}({#1})}
\newcommand{\trace}[1]{{Tr}({#1})}
\newcommand{\state}[1]{{\Sigma}({#1})}
\newcommand{\trans}[1]{{\Theta}({#1})}
\newcommand{\access}[3]{{\mathcal L}_{{#1},{#2}}({#3})}
\newcommand{\eats}{\rightarrow}
\newcommand{\Eats}[1]{\stackrel{#1}{\longrightarrow}}
\newcommand{\arrow}[2]{\xrightarrow{{\scriptscriptstyle #1}}_{{\scriptstyle #2}}}
\newcommand{\larrow}[2]{\xleftarrow[{\scriptscriptstyle #2}]{{\scriptstyle #1}}} 
\newcommand{\darrow}[2]{\stackrel{\scriptscriptstyle #1}{\Longrightarrow}_{\!\!\scriptscriptstyle #2}}
\newcommand{\harrow}[2]{\stackrel{\scriptscriptstyle #1}{\leadsto}_{\!\!\scriptscriptstyle #2}}

% numbers
\newcommand{\nat}{{\bf \mathbb{N}}}
\newcommand{\zed}{{\bf \mathbb{Z}}}
\newcommand{\rat}{{\bf \mathbb{Q}}}
\newcommand{\real}{{\bf \mathbb{R}}}
\newcommand{\linzed}{{\bf \mathcal{L}\mathbb{Z}}}

% powerset
\newcommand{\pow}[1]{{\mathcal P}(#1)}
\newcommand{\pown}[2]{{\mathcal P}^{#1}(#2)}

% by definition
\newcommand{\bydef}{\stackrel{\Delta}{=}}

% projection, lifting
\newcommand{\proj}[2]{{#1}\!\!\downarrow_{{#2}}}
\newcommand{\lift}[2]{{#1}\!\!\uparrow_{{#2}}}

% forward, backward
\newcommand{\forw}[1]{\stackrel{\rightarrow}{#1}}
\newcommand{\back}[1]{\stackrel{\leftarrow}{#1}}

% equation figures
\newcommand{\eqnfigx}[1]{
\begin{minipage}{3in}
\begin{center}
\begin{eqnarray*}#1
\end{eqnarray*}
\end{center}
\end{minipage}}

\newcommand{\twoeqnfigx}[2]{
\eqnfigx{#1}
\eqnfigx{#2}
}

\newcommand{\threeeqnfigx}[3]{
\eqnfigx{#1}\\
\eqnfigx{#2}\eqnfigx{#3}
}

\newcommand{\eqnfig}[1]{
\begin{minipage}{3in}
\begin{center}
\begin{eqnarray}#1
\end{eqnarray}
\end{center}
\end{minipage}}

\newcommand{\twoeqnfig}[2]{
\eqnfig{#1}
\eqnfig{#2}
}

% transitions and SOS rules
\newcommand{\storetrans}[1]{\stackrel{#1}{\leadsto}}
\newcommand{\storetransx}[1]{\stackrel{#1}{\leadsto^*}}
\newcommand{\storetransprime}[3]{{#1} \storetrans{#2} {#3}}
\newcommand{\gammatrans}[1]{\stackrel{#1}{\hookrightarrow}}
\newcommand{\gammatransprime}[3]{{#1} \gammatrans{#2} {#3}}
\newcommand{\gammatransx}[1]{\stackrel{#1}{\hookrightarrow^*}}
%\newcommand{\trans}[3]{{#1} \stackrel{#2}{\longrightarrow} {#3}}
\newcommand{\run}[3]{{#1} \stackrel{#2}{\Longrightarrow} {#3}}
\newcommand{\sosrule}[2]{\frac{\begin{array}{c}#1\end{array}}{#2}}
\newcommand{\hoare}[3]{\{#1\}\ {\bf #2}\ \{#3\}}

\newcommand{\derivone}[3]{\mbox{\infer[#3]{#2}{#1}}}
\newcommand{\derivtwo}[4]{\mbox{\infer[#4]{#3}{{#1} & {#2}}}}
\newcommand{\derivdot}[4]{\mbox{\infer[#4]{#3}
{\begin{array}{l} {#1} \\ \vdots \\ {#2} \end{array}}}}

% comments, paragraphs
\newcommand{\mypar}[1]{\vspace*{\baselineskip}\noindent\emph{#1}}
\renewcommand{\paragraph}[1]{\noindent{\bf #1}}

% theories
\newcommand{\arith}{\langle \zed, +, \cdot, 0, 1 \rangle}
\newcommand{\hilbert}{\langle \zed, +, \cdot, 0, 1 \rangle^\exists}
\newcommand{\presbg}{\langle \zed, \geq, +, 0, 1 \rangle}
% \newcommand{\divides}{\langle \zed, +, |, 0, 1 \rangle}
\newcommand{\lipshitz}{\langle \zed, \geq, +, |, 0, 1 \rangle^\exists}
\newcommand{\dioph}[1]{\mathfrak{D}(#1)}

\let\Asterisk\undefined
\newcommand{\Asterisk}{\mathop{\scalebox{1.7}{\raisebox{-0.2ex}{$\ast$}}}}%

% proofs
\newif\ifLongVersion\LongVersiontrue
%\newcommand{\proof}[1]{\noindent {\em Proof}: {#1}}
%\newcommand{\qed}{\hfill$\Box$}
%\newcommand{\inproof}[1]{\ifLongVersion \proof{#1} \fi}
%\newcommand{\outproof}[2]{\ifLongVersion \else \vspace*{\baselineskip}\noindent Proof of {#1}: \proof{#2} \fi}
%\renewcommand{\subsubsection}[1]{\noindent{\bf #1}}

% multisorted first-order logic
\newcommand{\sorts}[1]{#1^\mathrm{S}}
\newcommand{\funs}[1]{#1^\mathrm{F}}
\newcommand{\ssorts}[1]{#1^\mathrm{s}}
\newcommand{\sfuns}[1]{#1^\mathrm{f}}
\newcommand{\sort}[1]{\sigma({#1})}
\newcommand{\mods}{\mathbf{I}}
\newcommand{\modsof}[2]{\sem{#1}_{#2}}
\newcommand{\iffeq}{\Leftrightarrow}

% separation logic
\newcommand{\symhp}{\Pi}
\newcommand{\symhs}{\Theta}
\newcommand{\locsi}{\mathsf{L}}
\newcommand{\heaps}{\mathsf{Heaps}}
\newcommand{\teq}{\approx}
\newcommand{\Spatial}{\Sigma}
\newcommand{\Emp}{\mathbf{emp}}
\newcommand{\PointsTo}[2]{ {#1 \mapsto #2} }
\newcommand{\PointsToL}[3]{ {#2 \mapsto^{#1} #3} }
\newcommand{\I}{\mathcal{I}}
\newcommand{\J}{\mathcal{J}}
\newcommand{\G}{\mathcal{G}}
\newcommand{\K}{\mathcal{K}}
\newcommand{\herb}{\mathcal{H}}
\newcommand{\tinyherb}{{\scriptscriptstyle\mathcal{H}}}
\newcommand{\locs}{\mathsf{Loc}}
\newcommand{\nil}{\mathsf{nil}}
\newcommand{\emp}{\mathsf{emp}}
\newcommand{\wand}{
 \mathrel{\mbox{$\hspace*{-0.03em}\mathord{-}\hspace*{-0.66em}
 \mathord{-}\hspace*{-0.36em}\mathord{*}$\hspace*{-0.005em}}}} % {\multimap}
\newcommand{\seplog}{\mathsf{SL}}
\newcommand{\tinyseplog}{\mathsf{\scriptscriptstyle{SL}}}
\newcommand{\seplogl}[1]{\mathsf{SL}\parens{#1}}
\newcommand{\tterm}{\mathsf{t}}
\newcommand{\uterm}{\mathsf{u}}
\newcommand{\vterm}{\mathsf{v}}
\newcommand{\wterm}{\mathsf{w}}
\newcommand{\xterm}{\mathsf{x}}
\newcommand{\yterm}{\mathsf{y}}
\newcommand{\zterm}{\mathsf{z}}
\newcommand{\nextp}{\mathsf{next}}
\newcommand{\datap}{\mathsf{data}}
\newcommand{\lterm}{\ell}
\newcommand{\pto}{\mathsf{pt}}
\newcommand{\sets}{\mathsf{Set}}
\newcommand{\lab}{\triangleleft}
\newcommand{\labd}[1]{#1\!\!\Downarrow}
\newcommand{\thsep}{\mathsf{Sep}}
\newcommand{\fv}[2]{\mathrm{FV}_{#1}(#2)}
\newcommand{\pset}[2]{\mathrm{P}^{#1}(#2)}
\newcommand{\voc}[1]{\mathrm{V}({#1})}
\newcommand{\terms}[1]{\mathrm{Pt}(#1)}
\newcommand{\bounds}[2]{\mathrm{Bnd}(#1,#2)}
\newcommand{\dom}{\mathrm{dom}}
\newcommand{\img}{\mathrm{img}}
\newcommand{\size}[1]{\mathrm{Size}(#1)}
\newcommand{\vars}{\mathsf{Var}}
\newcommand{\preds}{\mathsf{Pred}}
\newcommand{\X}{\mathcal{X}}
\newcommand{\Y}{\mathcal{Y}}
\newcommand{\sys}{\mathcal{S}}
\newcommand{\lseq}{\Gamma}
\newcommand{\rseq}{\Delta}
\newcommand{\psys}{\mathcal{P}}
\newcommand{\cf}{\mathcal{F}}
\newcommand{\sw}{\leq_{\mathrm{sw}}}
\newcommand{\subst}[1]{\mathrm{VIS}(#1)}
\newcommand{\DS}{\mathfrak{D}^{\mathsf{\scriptscriptstyle{s}}}}

\newcommand{\infname}[1]{\begin{array}{c}\text{ ($#1$) }\\\\\end{array}}

\newcommand{\subtree}{\sqsubset}
\newcommand{\subtreeq}{\sqsubseteq}
\newcommand{\suptree}{\sqsupset}

\newcommand{\ta}{\mathrm{TA}}
\newcommand{\LU}{\mathrm{LU}}
\newcommand{\RU}{\mathrm{RU}}
\newcommand{\RD}{\mathrm{RD}}
\newcommand{\RI}{\wedge\mathrm{R}}
\newcommand{\SP}{\mathrm{SP}}
\newcommand{\ID}{\mathrm{ID}}
\newcommand{\AX}{\mathrm{AX}}
\newcommand{\Aone}{\mathrm{A1}}
\newcommand{\Atwo}{\mathrm{A2}}
\newcommand{\perms}{\mathsf{Perm}}

\newcommand{\Label}[1]{ \llbracket #1 \rrbracket_{L} }
\DeclareMathOperator{\Disjoint}{disjoint}
\newcommand{\parens}[1]{ \left( #1 \right) }
\newcommand{\Set}[1]{ \left\{ #1 \right\} }
\DeclareMathOperator{\CardOp}{card}
\newcommand{\Card}[1]{ \CardOp\parens{#1} }
\DeclareMathOperator{\Wit}{witness}
\DeclareMathOperator{\Choice}{Choice}
\DeclareMathOperator{\Explain}{explain}

\newcommand{\funcsolve}{\small\mathsf{solve}}
\newcommand{\funcsmtsolve}{\small\mathsf{solve\_rec}}

% \newcommand{\lan}{\mathbf{L}}
\newcommand{\purify}[1]{\lfloor #1 \rfloor}
\newcommand{\purifyrec}[1]{\purify{#1}^\ast}
\newcommand{\unpurify}[1]{\lceil #1 \rceil}
\newcommand{\quants}[1]{\mathrm{Q}(#1)}
\newcommand{\solvesl}[1]{\mathsf{solve_{SL(#1)}}}
\newcommand{\solveslt}[1]{\mathsf{solve_{#1}}}
\newcommand{\lblf}{\mathsf{lbl}}
\newcommand{\ltrue}{\top}
\newcommand{\lfalse}{\bot}
\newcommand{\ord}[2]{\prec_{#1,#2}}
\newcommand{\qvar}{x}
\newcommand{\Int}{\mathbb{Z}}
\newcommand{\Data}{\mathbb{D}}
\newcommand{\Bool}{\mathbb{B}}
\newcommand{\true}{\mathtt{t}}
\newcommand{\false}{\mathtt{f}}
\newcommand{\euf}{\mathsf{E}}
\newcommand{\lia}{\mathsf{LIA}}
\newcommand{\euflia}{\mathsf{ELIA}}
\newcommand{\tinylia}{\mathsf{\scriptscriptstyle{LIA}}}
\newcommand{\ul}{\mathsf{UL}}
\newcommand{\ulk}{\mathsf{UL}^{\!\!\scriptscriptstyle{K}}}
\newcommand{\il}{\mathsf{IL}}
\newcommand{\ulid}{\mathsf{ULID}}
\newcommand{\theory}{\mathbb{T}}
\newcommand{\dual}[1]{{#1}^{\sim}}
\newcommand{\bigO}[1]{\mathcal{O}({#1})}

% automata
\newcommand{\A}{\mathcal{A}}
\newcommand{\Post}[1]{\mathsf{Post}({#1})}
\newcommand{\Accept}[2]{\mathsf{Acc}_{\scriptscriptstyle{#1}}({#2})}
\newcommand{\AbsPost}[1]{\abs{\mathsf{Post}}_{#1}}
\newcommand{\AbsAccept}[1]{\abs{\mathsf{Acc}}_{#1}}
\newcommand{\Inv}{\mathsf{I}}
\newcommand{\abs}[1]{{#1}^\sharp}
\newcommand{\Art}{\mathcal{T}}
\newcommand{\SubArt}{\mathcal{S}}
\newcommand{\worklist}{\mathtt{WorkList}}
\newcommand{\rootNode}{\mathtt{r}}
\newcommand{\pivot}[1]{\Psi(#1)}
\newcommand{\impact}{\textsc{Impact}}
\newcommand{\PA}{{\scriptscriptstyle{\mathsf{PA}}}}
\newcommand{\ADA}{{\scriptscriptstyle{\mathsf{ADA}}}}
\newcommand{\cube}[1]{\mathsf{c}({#1})}
\newcommand{\T}{\mathcal{T}}
\newcommand{\event}[1]{{#1}_\Sigma}
\newcommand{\data}[1]{{#1}_\Data}
\newcommand{\pathform}[1]{\Theta({#1})}
\newcommand{\accform}[1]{\Upsilon({#1})}
\newcommand{\accformA}[2]{\Upsilon_{#1}({#2})}
\newcommand{\substform}[1]{\overline{\Upsilon}({#1})}
\newcommand{\substformn}[1]{\overline{\Theta}({#1})}
\newcommand{\substformA}[2]{\overline{\Upsilon}_{#1}({#2})}        
\newcommand{\quantform}[1]{\widehat{\Upsilon}({#1})}
\newcommand{\quantformn}[1]{\widehat{\Theta}({#1})}
\newcommand{\nextf}[1]{{#1}^\prime}
\newcommand{\supsetforex}{\supseteq^{\scriptscriptstyle\forall\exists}}
\newcommand{\prefix}{\preceq}
\newcommand{\posforms}{\mathsf{Form}^+}
\newcommand{\cover}{\sqsubseteq}


%%%%%%%%%%%%%%%%%%%%%%%%%%%%%%%%%%%%%%%%%%%%%%%%%%%%%%%%%%%%%%%%%%%%%%%%%%%%%%%
\begin{document}
%%%%%%%%%%%%%%%%%%%%%%%%%%%%%%%%%%%%%%%%%%%%%%%%%%%%%%%%%%%%%%%%%%%%%%%%%%%%%%%

\title{From Predicate Automata to Alternating Data Automata}

\author{Radu Iosif}
\institute{CNRS, Verimag \\
Radu.Iosif@univ-grenoble-alpes.fr}
\maketitle

\section{Definitions}

Let $Q$ be a finite set of predicates, where for each $q \in Q$, we
denote by $\#(q)\geq0$ its arity. We denote by $Q_0$ the subset of $Q$
consisting of predicates of arity zero. We consider an infinite
countable set $\vars$ of first-order variables, ranging over
$\nat$. For a tuple of variables $\vec{x} = \tuple{x_1,\ldots,x_n}$,
we define a one-to-one corresponding tuple $\overline{\vec{x}} =
\tuple{\overline{x_1},\ldots,\vec{x_n}}$. We associate each predicate
$q \in Q$ a tuple of variables $\vec{x}_q = \tuple{x_{q,1}, \ldots,
  x_{q,\#(q)}}$, called \emph{arguments}, such that, for any two $p,q
\in Q$, if $p \neq q$ then any two of $\vec{x}_p$,
$\overline{\vec{x}}_p$, $\vec{x}_q$ and $\overline{\vec{x}}_q$ have no
elements in common, respectively.

An \emph{atom} is a predicate $q(\vec{x}_q)$,
$q(\overline{\vec{x}}_q)$, $x \teq y$ or $x \not\teq y$, for $x,y \in
\vars$ and a \emph{formula} is a positive boolean combination of
atoms. We denote by $\Form(Q,\vec{x})$ the set of formulae with
predicates from $Q$ and free variables from $\vec{x}$. A \emph{ground
  formula} is a formula in which each predicate occurs as
$q(n_1,\ldots,n_{\#(q)})$, for some $n_1,\ldots,n_{\#(q)} \in \nat$. 

\subsection{Predicate Automata}

We denote by $\varepsilon$ the empty sequence and by $\Sigma^*$ the
set of finite sequences of input symbols.

\begin{definition}
  A predicate automaton (PA) is a tuple $\A =
  \tuple{\Sigma,Q,\iota,F,\Delta}$, where: \begin{compactitem}
  \item $\Sigma$ is a finite input alphabet,
  \item $Q$ is the set of predicates,
  \item $\iota \in \Form(Q,\emptyset)$ is a ground formula describing
    the initial configurations,
  \item $F \subseteq Q$ is a set of accepting predicates,
  \item $\Delta : Q \times \Sigma \rightarrow
    \Form(Q,\vec{x}\cup\overline{\vec{x}})$ is a transition function.
  \end{compactitem}
\end{definition}
We write transition rules as follows: 
\[q(\vec{x}_q) \arrow{\alpha,x_0}{} \Delta(q,\alpha) 
\text{, where $\fv{}{\Delta(q,\alpha)} \subseteq \bigcup \{\vec{x}_p
  \mid p \text{ occurs in } \Delta(q,\alpha)\} \cup
  \overline{\vec{x}}_q \cup \set{x_0}$}\] Intuitively, $\vec{x}_p$
denotes the current values of the arguments $\vec{x}_p$ of $p$, where
$p$ is any predicate that occurs in $\Delta(q,\alpha)$, $x_0$ is the
input value associated with the symbol $\alpha$ read by the automaton
and $\overline{\vec{x}}_q$ are the past values of the arguments of
$q$. We denote by $\Delta(\phi,\alpha)$ the formula obtained from
$\phi$ by replacing each occurrence of $q(\vec{x}_q)$ in $\phi$ by
$\Delta(q,\alpha)$.

\begin{example}
For example, instead of \(q(i,j) \arrow{\alpha,k}{} i = k \wedge j
\neq k \wedge p(k,j) \wedge q(i,k)\), we write: \[q(x_{q,1},x_{q,2})
\arrow{\alpha,x_0}{} \overline{x}_{q,1} = x_0 \wedge
\overline{x}_{q,2} \neq x_0 \wedge x_{p,1} = x_0 \wedge x_{p,2} =
\overline{x}_{q,2} \wedge p(x_{p,1},x_{p,2}) \wedge x_{q,1} =
\overline{x}_{q,1} \wedge x_{q,2} = x_0 \wedge q(x_{q,1},x_{q,2})\] We
assume that this preprocessing is already done for PA. \hfill\qed
\end{example}

A \emph{configuration} of $\A$ is a ground formula. We abuse notation
and blur the distinction between the terms denoting natural numbers
and their interpretations. The transition relation is a subset of
$\Form(Q,\emptyset) \times \Sigma \times \Form(Q,\emptyset)$ denoted
as \(C \stackrel{\alpha,n}{\Longrightarrow} C'\),
where: \begin{compactitem}
\item $C = \phi\sigma$, for a formula $\phi \in \Form(Q,\vec{x})$ and
  a substitution $\sigma : \vec{x} \rightarrow \nat$, and
%
\item $C' = \Delta(q,\alpha)\sigma'$, where $\sigma' :
  \fv{}{\Delta(q,\alpha)} \rightarrow \nat$ is a substitution defined
  as follows:
  \[\sigma'(y) = \left\{\begin{array}{ll}
  \sigma(x) & \text{if $y = \overline{x}$} \\
  n & \text{if $y = x_0$} \\
  \text{arbitrary $m \in \nat$} & \text{otherwise}
  \end{array}\right.\]
\end{compactitem}
Observe that, because of the arbitrary choices of current values for
the arguments, a configuration can have infinitely many outgoing
transitions. However, since we identify configurations that are
logically equivalent, a false configuration is a sink with only false
successors, for any pair $(\alpha,n) \in \Sigma \times \nat$. 

A word $w = (\alpha_1,n_1) \ldots (\alpha_k,n_k) \in (\Sigma \times
\nat)^*$ is \emph{accepted} by $\A$ iff there exists a sequence of
configurations $C_k, \ldots, C_0$ such that $C_k = \iota$, $C_{i+1}
\stackrel{\alpha_i,n_i}{\Longrightarrow} C_i$ for all $i \in [0,n-1]$
and $C_0 \in \Form(F,\emptyset)$ contains only accepting
predicates. The language of $\A$ is the set $L(\A)$ of accepted words.

\subsection{Alternating Data Automata}

Given a finite set $\vec{x} \subset \vars$ of variables, let $\vec{x}
\mapsto \nat$ be the set of valuations of the variables $\vec{x}$ and
$\Sigma[\vec{x}] = \Sigma \times (\vec{x} \mapsto \nat)$ be the set of
\emph{data symbols}. A \emph{data word} (word in the sequel) is a
finite sequence $(a_1,\nu_1)(a_2,\nu_2) \ldots (a_n,\nu_n)$ of data
symbols, where $a_1,\ldots,a_n \in \Sigma$ and $\nu_1,\ldots,\nu_n :
\vec{x} \rightarrow \nat$ are valuations. We denote by
$\Sigma[\vec{x}]^*$ the set of data words over $\vec{x}$.

\begin{definition}
  An alternating data automaton (ADA) is a tuple $\A =
  \tuple{\Sigma,\vec{x},Q_0,\iota,F,\Delta}$,
  where: \begin{compactitem}
    \item $\Sigma$ is a finite input alphabet,
    \item $\vec{x}$ is a finite set of variables, ranging over $\nat$,
    \item $Q_0$ is a set of predicates of arity zero, called \emph{states}
    \item $\iota \in \Form(Q_0,\emptyset)$ is a boolean combination of states,
    \item $F \subseteq Q_0$ is a set of final states,
    \item $\Delta : Q_0 \times \Sigma \rightarrow
      \Form(Q_0,\vec{x}\cup\overline{\vec{x}})$ is a transition function.
  \end{compactitem}
\end{definition}
In each formula $\Delta(q,a)$ describing a transition rule, the
variables $\overline{\vec{x}}$ track the previous and $\vec{x}$ the
current values of the variables of $\A$. Observe that the initial
values of the variables are left unconstrained, as the initial
configuration does not contain free data variables.

Let $\vec{x}_k = \set{x_k \mid x \in \vec{x}}$, for any $k\geq0$, be a
set of time-stamped variables. For an input symbol $\alpha \in \Sigma$
and a formula $\phi \in \Form(Q_0,\vec{x}\cup\overline{\vec{x}})$, we
write $\Delta^k(\phi,a)$ for the formula obtained from $\phi$ by
simultaneously replacing each $q \in Q_0$ occurring in $\phi$ by the
formula
$\Delta(q,a)[\vec{x}_k/\overline{\vec{x}},\vec{x}_{k+1}/\vec{x}]$,
$k\geq0$. Given a word $w = (a_1,\nu_1)(a_2,\nu_2) \ldots
(a_n,\nu_n)$, the \emph{run} of $\A$ over $w$ is the sequence of
formulae: \[\phi_0(Q) \Rightarrow \phi_1(Q,\vec{x}_0 \cup \vec{x}_1)
\Rightarrow \ldots \Rightarrow \phi_n(Q,\vec{x}_0 \cup \ldots \cup
\vec{x}_n)\] where $\phi_0 \equiv \iota$ and, for all $k\in[1,n]$, we
have $\phi_k \equiv \Delta^k(\phi_{k-1},a_k)$. Next, we slightly abuse
notation and write $\Delta(\iota,a_1,\ldots,a_n)$ for the formula
$\phi_n(\vec{x}_0,\ldots,\vec{x}_n)$ above. We say that $\A$
\emph{accepts} $w$ iff $\I,\nu \models \Delta(\iota,a_1,\ldots,a_n)$,
for some valuation $\nu$ that maps:\begin{compactitem}
%
\item each $x \in \vec{x}_k$ to $\nu_k(x)$, for all $k\in[1,n]$, 
%
\item each $q \in F$ to $\top$ and 
%
\item each $q \in Q_0 \setminus F$ to $\bot$.
\end{compactitem}
The language of $\A$ is the set $L(\A)$ of words from
$\Sigma[\vec{x}]^*$ accepted by $\A$.

\section{Translation of PA into ADA}

Given a predicate automaton $\A_\PA = \tuple{\Sigma, Q_\PA, \iota_\PA,
  F_\PA, \Delta_\PA}$, we define an alternating data automaton
$\A_\ADA = \tuple{\Sigma, \vec{x}_\ADA, Q_\ADA, \iota_\ADA, F_\ADA,
  \Delta_\ADA}$ as follows: \begin{compactitem}
\item $\vec{x}_\ADA = \set{x \in \vars \mid x \in \fv{}{\Delta_\PA(q,\alpha)}, q \in Q_\PA, \alpha \in \Sigma}$, 
%
\item $Q_\ADA = \set{\overline{q} \mid q \in Q_\PA}$, where $\#(\overline{q}) = 0$, for all $\overline{q} \in Q_\ADA$,
%
\item $\iota_\ADA$ is obtained from $\iota_\PA$ by replacing each
  occurrence of a ground term $q(n_1,\ldots,n_{\#(q)})$ from
  $\iota_\PA$ with $\overline{q}$,
%
\item $F_\ADA = \set{\overline{q} \mid q \in F_\PA}$, and 
%
\item for each transition rule \(q(k_1,\ldots,k_{\#(q)}) \arrow{\alpha,x_0}{}
  \Delta_\PA(q,\alpha)\) \begin{compactitem}
  \item if \(k_1,\ldots,k_q\) is the tuple of variables $\vec{x}_q$
    then \(\Delta_\ADA(q,\alpha)\) is obtained from
    \(\Delta_\PA(q,\alpha)\) by replacing each occurrence of a
    predicate $q(\vec{x}_q)$ with $\overline{q}$, and
  %
\item else, if \(k_1,\ldots,k_{\#(q)} \in \nat\), we add the
  conjunction $\bigwedge_{i=1}^{\#(q)} \overline{x}_i = k_i$ to
  \(\Delta_\ADA(q,\alpha)\).
  \end{compactitem}
  Nothing else is in $\Delta_\ADA$. 
\end{compactitem}

\begin{example}
The PA transition rule: 
\[q(x_{q,1},x_{q,2})
\arrow{\alpha,x_0}{} \overline{x}_{q,1} = x_0 \wedge
\overline{x}_{q,2} \neq x_0 \wedge x_{p,1} = x_0 \wedge x_{p,2} =
\overline{x}_{q,2} \wedge p(x_{p,1},x_{p,2}) \wedge x_{q,1} =
\overline{x}_{q,1} \wedge x_{q,2} = x_0 \wedge q(x_{q,1},x_{q,2})\]
becomes the following ADA transition rule:
\[\overline{q} \arrow{\alpha}{} \overline{x}_{q,1} = x_0 \wedge
\overline{x}_{q,2} \neq x_0 \wedge x_{p,1} = x_0 \wedge x_{p,2} =
\overline{x}_{q,2} \wedge \overline{p} \wedge x_{q,1} =
\overline{x}_{q,1} \wedge x_{q,2} = x_0 \wedge \overline{q}\]
\hfill\qed
\end{example}

\begin{claim}
  $L(\A_\PA) \neq \emptyset$ if and only if $L(\A_\ADA) \neq \emptyset$.
\end{claim}

%%%%%%%%%%%%%%%%%%%%%%%%%%%%%%%%%%%%%%%%%%%%%%%%%%%%%%%%%%%%%%%%%%%%%%%%%%%%%%%
\end{document}
%%%%%%%%%%%%%%%%%%%%%%%%%%%%%%%%%%%%%%%%%%%%%%%%%%%%%%%%%%%%%%%%%%%%%%%%%%%%%%%



